\begin{question}
    \normalfont
    Find the characteristic polynomials, eigenvalues, and eigenvectors of each of the following matrices. For each eigenvalue, find its algebraic multiplicity and geometric multiplicity.
    \begin{enumerate}[(i)]
        \item
              \[
                  A= \begin{pmatrix}
                      4 & -5 \\
                      2 & -3 \\
                  \end{pmatrix}
              \]

        \item
              \[
                  B= \begin{pmatrix}
                      2 & -1 & 0 \\
                      0 & 2  & 1 \\
                      0 & 0  & 1
                  \end{pmatrix}
              \]
        \item
              \[
                  C= \begin{pmatrix}
                      1  & 3  & 3  \\
                      -3 & -5 & -3 \\
                      3  & 3  & 1  \\
                  \end{pmatrix}
              \]
    \end{enumerate}

\end{question}

\begin{proof}
    \renewcommand{\qedsymbol}{$\blacksquare$}
    \begin{enumerate}[(i)]
        \item Let $\lambda$ and $v$ be an eigenvalue and eigenvector for $A$.
              Solving for $Av=\lambda v$,
              \[
                  \begin{aligned}
                       & (A-\lambda I)v=0             \\
                      \iff
                       & \det(A-\lambda I)=0          \\
                      \iff
                       & \begin{vmatrix}
                             4-\lambda & -5         \\
                             2         & -3-\lambda
                         \end{vmatrix}=0       \\
                      \iff
                       & (4-\lambda)(-3-\lambda)+10=0 \\
                      \iff
                       & \lambda^2-\lambda-2=0        \\
                      \iff
                       & (\lambda+1)(\lambda-2)=0
                  \end{aligned}
              \]
              The characteristic polynomial for $A$ is $f(\lambda)=\lambda^2-\lambda-2$.
              Solving for $f(\lambda)=0$, we obtain $\lambda\in\{-1,2\}$, which are the eigenvalues for $A$.
              It follows that the algebraic multiplicity for both eigenvalues are 1.
              Moreover, since the geometric multiplicity cannot exceed the algebraic multiplicity but must also be nonzero, the geometric multiplicity for both eigenvalues is 1.

              When $\lambda=-1$, solving for $Av=\lambda v$,
              \[
                  \begin{aligned}
                       & (A+I)v=0                    \\
                      \iff
                       & \MatTwoTwo{5}{-5}{2}{-2}v=0
                  \end{aligned}
              \]
              Letting $v=\ColVecTwo{1}{1}$, we obtain $(A-\lambda I)v=0$, or $v=\ColVecTwo{1}{1}$ is an eigenvector corresponding to $\lambda=-1$.
              But since the geometric multiplicity is 1, it follows that the set of eigenvectors corresponding to $\lambda=-1$ is $\SpanLA(\ColVecTwo{1}{1}) -\{0\}$.

              When $\lambda=2$, solving for $Av=\lambda v$,
              \[
                  \begin{aligned}
                       & (A-2I)v=0                   \\
                      \iff
                       & \MatTwoTwo{2}{-5}{2}{-5}v=0
                  \end{aligned}
              \]
              Letting $v=\ColVecTwo{5}{2}$, we obtain $(A-\lambda I)v=0$, or $v=\ColVecTwo{5}{2}$ is an eigenvector corresponding to $\lambda=2$.
              But since the geometric multiplicity is 1, it follows that the set of eigenvectors corresponding to $\lambda=2$ is $\SpanLA(\ColVecTwo{5}{2})-\{0\}$.
        \item Let $\lambda$ and $v$ be an eigenvalue and eigenvector for $A$.
              Solving for $Av=\lambda v$,
              \[
                  \begin{aligned}
                       & (A-\lambda I)v=0                    \\
                      \iff
                       & \det(A-\lambda I)=0                 \\
                      \iff
                       & \begin{vmatrix}
                             2-\lambda & -1        & 0         \\
                             0         & 2-\lambda & 1         \\
                             0         & 0         & 1-\lambda
                         \end{vmatrix}=0   \\
                      \iff
                       & (2-\lambda)(2-\lambda)(1-\lambda)=0 \\
                  \end{aligned}
              \]
              The characteristic polynomial for $A$ is $f(\lambda)=(2-\lambda)(2-\lambda)(1-\lambda)$.
              It follows that the algebraic multiplicity for $\lambda=2$ is 2 and for $\lambda=1$ is 1.

              Let $v=\ColVecThree{v_1}{v_2}{v_3}$.
              When $\lambda=2$, solving for $Av=\lambda v$,
              \[
                  \begin{aligned}
                       & (A-2I)v=0                    \\
                      \iff
                       & \begin{pmatrix}
                            0 & -1 & 0\\
                            0 & 0 & 1\\
                            0 & 0 & -1
                       \end{pmatrix}\ColVecThree{v_1}{v_2}{v_3}=0\\
                       \iff 
                       &\ColVecThree{-v_2}{v_3}{-v_3}=0\\
                       \iff
                       &v_2=v_3=0
                  \end{aligned}
              \]
              It follows that the eigenvectors corresponding to $\lambda=2$ must have the form $v=\ColVecThree{k}{0}{0}$, where $k$ is a nonzero real number.
              Thus, the geometric multiplicity for $\lambda=2$ is 1 and the set of eigenvectors corresponding to that eigenvalue is $\SpanLA(\ColVecThree{1}{0}{0})-\{0\}$.

              When $\lambda=1$, solving for $Av=\lambda v$,
              \[
                  \begin{aligned}
                       & (A-I)v=0                   \\
                      \iff
                       & \begin{pmatrix}
                            1 & -1 & 0\\
                            0 & 1 & 1\\
                            0 & 0 & 0
                       \end{pmatrix}\ColVecThree{v_1}{v_2}{v_3}=0\\
                        \iff
                        &\ColVecThree{v_1-v_2}{v_2+v_3}{0}=0\\
                        \iff
                        &v_1=v_2=-v_3
                  \end{aligned}\\
              \]
                It follows the eigenvectors corresponding to $\lambda=1$ must have the form $v=\ColVecThree{k}{k}{-k}$, where $k$ is a nonzero real number.
                Thus, the geometric multiplicity for $\lambda=1$ is 1 and the set of eigenvectors corresponding to that eigenvalue is $\SpanLA(\ColVecThree{1}{1}{-1})-\{0\}$.
        \item The characteristic polynomial for $C$ is $f(\lambda)=-\lambda^3-3\lambda^2+4$ and the eigenvalues are $\lambda\in\{1,-2\}$.
        When $\lambda=1$, the set of eigenvectors is $\SpanLA(\ColVecThree{1}{-1}{1})-\{0\}$ and both its algebraic and geometric multiplicity is 1.
        When $\lambda=-2$, the set of eigenvectors is $\SpanLA(\ColVecThree{-1}{0}{1},\ColVecThree{-1}{1}{0})-\{0\}$ and both its algebraic and geometric multiplicity is 2.
    \end{enumerate}
    \renewcommand{\qedsymbol}{}
\end{proof}