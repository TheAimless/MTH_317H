\begin{question}
\normalfont In this problem you will prove that the determinant of an $n\times n$ matrix $A$ is the product of its eigenvalues (repeated with multiplicity), i.e. det$(A) =\lambda_1\lambda_2...\lambda_n$. You will do this in two steps:
\begin{enumerate}[(i)]
\item First, show that 
\[
\textup{det}(A-\lambda I) = (\lambda_1 - \lambda)(\lambda_2 - \lambda) \ldots (\lambda_n-\lambda),
\]
where $\lambda_1, \lambda_2,...\lambda_n$ are the eigenvalues of $A$ (repeated with multiplicity). \\ \\
\emph{Hint: Use the fact that the eigenvalues are the roots of the characteristic polynomial, plus your result from Question \ref{degree} about the coefficient of $\lambda^n$ in the characteristic polynomial.}

\item Then, plug $\lambda=0$ into your formula from part (i) to conclude that the determinant of $A$ is the product of the eigenvalues. 
\end{enumerate}
\end{question}

\begin{proof}
    \renewcommand{\qedsymbol}{$\blacksquare$}
    \begin{enumerate}[(i)]
        \item Consider the characteristic polynomial of $A$, $f(\lambda)=\det(A-\lambda I)$.
        From Problem 2.5, the polynomial $f(\lambda)$ is polynomial of degree $n$.
        Moreover, since the eigenvalues are the roots of $f(\lambda)$, $f(\lambda)$ can be written as $k(\lambda-\lambda_1)(\lambda-\lambda_2)\dots(\lambda-\lambda_n)$ for some $k\in\R$.
        Finally, since the coefficient of $\lambda^n$ in $f(\lambda)$ is $(-1)^n$, $k=(-1)^n$.
        Hence, 
        \[
            \begin{aligned}
                \det(A-\lambda I)=(-1)^n\prod_{i=1}^{n}(\lambda-\lambda_i)
                =\prod_{i=1}^{n}(-1)(\lambda-\lambda_i)
                =(\lambda_1-\lambda)(\lambda_2-\lambda)\dots(\lambda_n-\lambda)
            \end{aligned}
        \]\qed
        \item When $\lambda=0$, $\det(A)=\det(A-\lambda I)=(\lambda_1-\lambda)(\lambda_2-\lambda)\dots(\lambda_n-\lambda)=\lambda_1\lambda_2\dots\lambda_n$.
        Therefore, the determinant of $A$ is the product of its eigenvalues.\qed
    \end{enumerate}
    \renewcommand{\qedsymbol}{}
\end{proof}