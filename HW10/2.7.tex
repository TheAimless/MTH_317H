\begin{question}	
	\normalfont
Let $R$ be any polygon in the plane. Prove that it is possible to divide $R$ into triangles, all of whose vertices are vertices of $R$.
\end{question}

\begin{proof}
    \renewcommand{\qedsymbol}{$\blacksquare$}
    Suppose $R$ is convex.
    For $n\in\N$, $n\geq 3$ let 
    \[
        \begin{aligned}
            P(n)\iff \text{R can be divided into triangles whose vertices are vertices of R}
        \end{aligned}
    \]
    Base case: When $n=3$, the vertices of $R$ form a triangle itself, $R$.
    Hence, the base case holds.

    Induction step: Suppose $P(n)$ holds for some $n=k\in\N$.
    To show that $P(k+1)$ holds, let $R=A_1A_2...A_{k+1}$, where $A_1,...,A_{k+1}$ are vertices of $R$.
    By connecting the vertices $A_k$ and $A_1$ with a line segment, we obtain a new polygon $R'=A_1A_2...A_k$ that has $k$ vertices.
    Since $R$ is convex, the line segment $A_kA_1$ is fully contained inside $R$.
    Hence, the new polygon must also be convex.

    From the induction hypothesis, since $P(k)$ holds, $R'$ can be divided into triangles.
    However, since the points $A_k,A_{k+1},A_1$ are pairwise distinct, $A_kA_{k+1}A_1$ must be a triangle in the plane.
    Thus, the partition of $R'$ into triangles plus the triangle $A_kA_{k+1}A_1$ must be a partition of $R$ into triangles, all of whose vertices are vertices of $R$.
    It follows that $P(k+1)$ holds.

    From the principle of mathematical induction, it must be the case that $P(n)$ holds for all $n\in\N$, $n\geq 3$.
\end{proof}
