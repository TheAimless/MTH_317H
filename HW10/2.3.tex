\begin{question}\normalfont

Recall that a matrix $A$ is called \emph{nilpotent} if $A^k=0$ for some $k$. 

\begin{enumerate}[(i)]
\item Show that the matrix 
\[
B =\begin{pmatrix}
			2 & 2 & -2\\
			5 & 1 & -3\\
			1 & 5 & -3 \\
		\end{pmatrix}
\]
is nilpotent.

\item Find the eigenvalues of the matrix $B$. You do not need to find the associated eigenvectors. 

\item Prove that for any nilpotent matrix $A$, the only eigenvalue of $A$ is 0. 
\end{enumerate}
	
	\end{question}

\begin{proof}
    \renewcommand{\qedsymbol}{$\blacksquare$}
    \begin{enumerate}[(i)]
        \item We have 
        \[
            \begin{aligned}
                B^3=B^2B
                &=\left(\begin{pmatrix}
                    2 & 2 & -2\\
                    5 & 1 & -3\\
                    1 & 5 & -3
                \end{pmatrix}\begin{pmatrix}
                    2 & 2 & -2\\
                    5 & 1 & -3\\
                    1 & 5 & -3
                \end{pmatrix}\right)B\\
                &=\begin{pmatrix}
                    12 & -4 & -4\\
                    12 & -4 & -4\\
                    24 & -8 & -8
                \end{pmatrix}\begin{pmatrix}
                    2 & 2 & -2\\
                    5 & 1 & -3\\
                    1 & 5 & -3
                \end{pmatrix}=\begin{pmatrix}
                    0 & 0 & 0\\
                    0 & 0 & 0\\
                    0 & 0 & 0
                \end{pmatrix}
            \end{aligned}
        \]
        Hence, $B$ is nilpotent.\qed
        \item The eigenvalue of $B$ is 0.
        \item Let $A$ be a nilpotent matrix. Then, there exists $n\in\N$ such that 
        \[
            \begin{aligned}
                A^n=0
            \end{aligned}
        \]
        Let $\lambda\in\R$ be an eigenvalue for $A$ and $v$ be a nonzero eigenvector for $A$ with respect to $\lambda$.
        We will prove that for any $k\in\N$, $A^kv=\lambda^kv$ by induction.

        For $k\in\N$, let 
        \[
            \begin{aligned}
                P(k)\iff A^kv=\lambda^kv
            \end{aligned}
        \]
        Base case: When $k=1$, $A^1v=Av=\lambda v=\lambda^1v$, or $P(1)$ holds.
        
        Induction step: Suppose $P(k)$ holds for some $k=t\in\N$.
        From the induction hypothesis, $A^tv=\lambda^tv$.
        It follows that 
        \[
            \begin{aligned}
                A^{t+1}v=(AA^t)v=A(A^{t}v)=A(\lambda^tv)=\lambda^t(Av)=\lambda^t(\lambda v)=\lambda^{t+1}v
            \end{aligned}
        \]
        Hence, $P(t+1)$ holds. By the principle of mathematical induction, it follows that $A^kv=\lambda^kv$ for all $k\in\N$.
        
        Since $A^n=0$ and $A^nv=\lambda^nv$ from the proof above, it follows that $\lambda^nv=0$.
        But as $v\not=0$, $\lambda^n=0$, or $\lambda=0$.
        Therefore, for any nilpotent matrix $A$, the only eigenvalue of $A$ is 0.\qed
    \end{enumerate}
    \renewcommand{\qedsymbol}{}
\end{proof}