\begin{question}
    \normalfont
    This LP assumes that you have been introduced to and have worked with the dot product on $\R^2$ and that you are familiar with the definition of length of vectors in $\R^2$. Here are some relevant definitions.
\end{question}

\begin{DEF}
    If $u,v\in\R^2$ with $u=(u_1,u_2)$ and $v=(v_1,v_2)$, the dot product (or inner product) of u and v is denoted as "$u\cdot v$" and is defined as 
    \[
        \begin{aligned}
            u\cdot v=u_1v_1+u_2v_2
        \end{aligned}
    \]  
\end{DEF}

\begin{DEF}
    The length of a vector $v\in\R^2,v=(v_1,v_2)$, is given by 
    \[
        \begin{aligned}
            |v|=\sqrt{(v_1)^2+(v_2)^2}
        \end{aligned}
    \]
    Note that 
    \[
        \begin{aligned}
            |v|^2=v\cdot v
        \end{aligned}
    \]
\end{DEF}
We will now use, without proof, the following fact about the dot product:
\begin{lem}
    If $u,v\in\R^2$, then for $\theta\in[0,\pi]$ as the unique angle between u and v, 
    \[
        \begin{aligned}
            u\cdot v=|u||v|\cos(\theta)
        \end{aligned}
    \]
\end{lem}

\begin{DEF}
    If A is the matrix, 
    \[
        \begin{aligned}
            A=\MatTwoTwo{a}{b}{c}{d},
        \end{aligned}
    \]
    then the \text{determinant} of A is defined as 
    \begin{equation}
        \textnormal{det}(A)=ad-cb.
    \end{equation}
\end{DEF}
In many results that follow, you will use the same linear function $L:\R^2\rightarrow\R^2$, where 
\begin{equation}
    \begin{aligned}
        \text{whenever } &x=\ColVecTwo{x_1}{x_2}=x_1e_1+x_2e_2\textbf{ in the canonical basis,}\\
                        &L(x)=Ax,\text{ defined via matrix multiplication}
    \end{aligned}
\end{equation}

\textbf{Part (1).} Let $E$ be the cube in $\R^2$ that is the parallelogram whose two defining edges are $e_1$ and $e_2$ and whose diagonal is $e_1+e_2$. (I.e. the parallelogram whose vertices are $(0,0),(1,0),(1,1),(0,1)$.) Let $F$ be the parallelogram with defining edges $L(e_1)$ and $L(e_2)$ and whose diagonal is $L(e_1)+L(e_2)$. (I.e. the parallelogram whose vertices are $(0,0),L(e_1),L(e_1)+L(e_2),L(e_2).$)

Prove that 
\[
    \begin{aligned}
        \text{area of }F=|\text{det}(A)|.
    \end{aligned}
\]

\textbf{Part (2).} Use the previous part to prove that for $L$ defined in $(4)$, 
\[
    \begin{aligned}
        \text{det}(A)=0\iff\text{L is not injective.}
    \end{aligned}
\]

\begin{sol}
\begin{lem}
    \normalfont
    Let $\triangle ABC$ be a triangle, $b=\overline{\rm AB}$, $c=\overline{\rm AC}$ and $S$ be the area of $ABC$. Then, 
    \[
        \begin{aligned}
            S=\frac{1}{2}bc\sin{A}
        \end{aligned}
    \]
\end{lem}
\begin{proof}
    \renewcommand{\qedsymbol}{$\blacksquare$}
    %\begin{tikzpicture}
    %    \amongUsI{red}{cyan}
    %\end{tikzpicture}
    \[
        \begin{tikzpicture}
            \tkzDefPoints{1/4/A,0/0/B,5/0/C,2.5/2.5/H}
            \tkzDrawPolygon(A,B,C)
            \tkzDrawPoints(A,B,C,H)
            \tkzLabelPoints(B,C)
            \tkzLabelPoints[above](A)
            \tkzLabelPoints[above right](H)
            \tkzDrawSegment(B,H)
            \tkzMarkRightAngle(B,H,C)
            \tkzMarkAngle[size=0.4](B,A,C)
        \end{tikzpicture}
    \]
    Let $\overline{\rm BH}$ be the altitude of $\triangle ABC$.
    Then, $\angle BHA=90^{\circ}$, or $\triangle BHA$ is a right triangle.
    It follows that $\frac{BH}{BA}=\sin{A}$, or $b\sin{A}=\overline{\rm BH}$.
    Therefore, 
    \[
        \begin{aligned}
            \frac{1}{2}bc\sin{A}=\frac{1}{2}\overline{\rm BH}\cdot\overline{\rm AC}=S
        \end{aligned}
    \] , by the definition of the area of a triangle.
\end{proof}
\renewcommand{\qedsymbol}{$\blacksquare$}
Back to our original problem, let $v_1=L(e_1)=\ColVecTwo{a}{b}$ and $v_2=L(e_2)=\ColVecTwo{c}{d}$ for some $a,b,c,d\in\R$.
Let $v_3=\ColVecTwo{d}{-c}\in\R^2$.
It follows that $v_2\cdot v_3=cd-cd=0=|v_2||v_3|\cos{(v_2,v_3)}$.

If $|v_2|=0$, since $|v_2|=\sqrt{c^2+d^2}\geq 0$ for all $c,d\in\R$, equality only happens when $c=d=0$.
Hence, $L(e_2)=(0,0)$, or the parallelogram has two points that coincides.
It follows that the area of $F$ is 0, which is also equal to $|\det(A)|=ad-bc=0a-0b$.

Denote $\angle(u,v)$ as the angle between the vectors $u$ and $v$.
If $|v_2|\not=0$, since $|v_2|=|v_3|=\sqrt{c^2+d^2}$, it must be the case that $\cos{(v_2,v_3)}=0$.
Thus, $\angle(v_2,v_3)=90^{\circ}$ or $\angle(v_2,v_3)=-90^{\circ}$.
But since $\angle(v_1,v_3)=\angle(v_1,v_2)+\angle(v_2,v_3)$, $\angle(v_1,v_2)=\angle(v_1,v_3)-90^{\circ}$ or $\angle(v_1,v_2)=\angle(v_1,v_3)+90^{\circ}$.

We have 
\[
    \begin{aligned}
        |\det(A)|&=|ad-bc|=|v_1\cdot v_3|\\&=||v_1||v_2|\cos(v_1,v_3)|=|v_1||v_3||\cos(\angle(v_1,v_2)+\angle(v_2,v_3))|
    \end{aligned}
\]
If $\angle(v_2,v_3)=90^{\circ}$, $|\cos(\angle(v_1,v_2)+\angle(v_2,v_3))|=|\cos(\angle(v_1,v_2)+90^{\circ})|=|-\sin(v_1,v_2)|=|\sin(v_1,v_2)|$.
If $\angle(v_2,v_3)=-90^{\circ}$, $|\cos(\angle(v_1,v_2)+\angle(v_2,v_3))|=|\cos(\angle(v_1,v_2)-90^{\circ})|=|\cos(90^{\circ}-\angle(v_1,v_2))|=|\sin(v_1,v_2)|$

Thus, $|\det(A)|=|v_1||v_2||sin(v_1,v_2)|$.
However, by the lemma, $|v_1||v_2||\sin(v_1,v_2)|$ is twice the area of the triangle bound by the origin, $v_1,v_2$, which is also the area of the parallelogram $F$.
Therefore, $|\det(A)|=\text{ area of }F$.\qed

For the second part, assume $\det(A)=0$.
From part (1), it follows that the area of the parallelogram $F$ constructed as above is 0.
Thus, $L(e_1)$ and $L(e_2)$ must be scalar multiple of each other, or there exists $\lambda\in\R$ such that $L(e_2)=\lambda L(e_1)$.

Let $x_1=\lambda e_1$ and $x_2=e_2$. Then, $x_1\not=x_2$, but 
\[
    \begin{aligned}
        L(x_1)=L(\lambda e_1)=\lambda L(e_1)=L(e_2)=L(x_2)
    \end{aligned}
\]
Hence, there exists $x_1,x_2\in\R^2$ such that $x_1\not=x_2$ yet $L(x_1)=L(x_2)$. Thus, $L$ is not injective.

Conversely, suppose $L$ is not injective.
Then, there exists $x_1=\ColVecTwo{a_1}{a_2}\in\R^2,x_2=\ColVecTwo{b_1}{b_2}\in\R^2$ such that $x_1\not=x_2$ and $L(x_1)=L(x_2)$.
Since $x_1\not=x_2$, $a_1\not=a_2$ or $b_1\not=b_2$.
WLOG, suppose $b_2\not=a_2$.
Then, $b_2-a_2\not=0$.

We have: $L(x_1)=L(a_1e_1+a_2e_2)=a_1L(e_1)+a_2L(e_2)$.
Similarly, $L(x_2)=L(b_1e_1+b_2e_2)=b_1L(e_1)+b_2L(e_2)$.
Since $L(x_1)=L(x_2)$, $a_1L(e_1)+a_2L(e_2)=b_1L(e_1)+b_2L(e_2)$
, or $(a_1-b_1)L(e_1)=(b_2-a_2)L(e_2)$.
But since $b_2-a_2\not=0$, it follows that $L(e_2)=\dfrac{a_1-b_1}{b_2-a_2}L(e_1)$.
Hence, $L(e_2)$ is a scalar multiple of $L(e_1)$, or the area of $F$ must be zero.
But as the area of $F$ equals zero implies $|\det(A)|=0$.
Hence, if $L$ is not injective then $\det(A)=0$.
Therefore, 
\[
    \begin{aligned}
        \det(A)=0\iff L\text{ is not injective.}
    \end{aligned}
\]
\end{sol}
\renewcommand{\qedsymbol}{$\blacksquare$}
\qed