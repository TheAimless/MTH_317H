\begin{question}
    \normalfont
    Let $L:\R^4\rightarrow\R^2$ be the linear function defined by
    \begin{equation}
        L(\ColVecFour{x_1}{x_2}{x_3}{x_4})=\ColVecTwo{x_2+x_3-5x_4}{x_1+x_4}
    \end{equation}
    \begin{enumerate}[(i)]
        \item In the \textit{canonical basis} for both $\R^4$ and $\R^2$ (i.e. $\R^4=\spanla({e_1,e_2,e_3,e_4})$) and, with an abuse of notation, $\R^2=\spanla({e_1,e_2})$, give the unique matrix, $A$, so that 
        \[
            \begin{aligned}
                \text{whenever }&x=\ColVecFour{x_1}{x_2}{x_3}{x_4}=x_1e_1+x_2e_2+x_3e_3+x_4e_4\textbf{ in the canonical basis},\\
                &L(x)=Ax,\text{ defined via matrix multiplication.}
            \end{aligned}
        \]
        \item Without doing any computation involving $L$, give a lower bound for $\dim(\nullla(L))$. That is to say find an integer $k$ so that $\dim(\nullla(L))\geq k$ and $k>0.$
        \item Give a basis for $\nullla(L)$. Prove that your choice of basis is correct.
        \item Explain why, without calculating what is in $\range(L)$, that it must satisfy $\range(L)=\R^2$.
        \item Start with the list, $\ell={L(e_1),L(e_2),L(e_3),L(e_4)}$ and use the algorithm of LADR result 2.31 to reduce $\ell$ to a basis of $\range(L)$. You must show each step of the algorithm and justify each choice. Prove that your resulting list is indeed a basis of $\range(L)$.
        \item Find some $x_0\in\R^4$ so that 
        \[
            \begin{aligned}
                L(x_0)=\ColVecTwo{5}{3}
            \end{aligned}
        \]
        Check that indeed your choice of $x_0$ gives the desired output.
        \item List all possible solutions, $v\in\R^4$, of the equation 
        \[
            \begin{aligned}
                L(v)=\ColVecTwo{5}{3}
            \end{aligned}
        \]
    \end{enumerate}
\end{question}

\begin{proof}
    \begin{enumerate}[(i)]
        \item $A$ is given by 
        \[
            \begin{aligned}
                A = \begin{pmatrix}
                    0 & 1 & 1 & -5\\
                    1 & 0 & 0 & 1
                \end{pmatrix}
            \end{aligned}
        \]
        \item A lower bound $k$ for $\dim(\nullla(L))$ is $k=2$.
        \item Let $B=u_1,u_2$ such that $u_1=\ColVecFour{0}{1}{-1}{0}$ and $u_2=\ColVecFour{-1}{5}{0}{1}$.
        We will show that $B$ is a basis for $\nullla(L)$.
        
        Let $a_1,a_2\in\R$ such that $a_1u_1+a_2u_2=0$. We have 
        \[
            \begin{aligned}
                0
                &= a_1u_1+a_2u_2\\
                &= \ColVecFour{-a_2}{a_1+5a_2}{-a_1}{a_2}
            \end{aligned}
        \]
        , which reduces to
        \begin{align*}
            &\begin{cases}
                -a_2&=0\\
                a_1+5a_2&=0\\
                -a_1&=0\\
                a_2&=0
            \end{cases}\\&\implies
            a_1=a_2=0
        \end{align*}
        Hence, the only solution to $a_1u_1+a_2u_2=0$ is the trivial solution, or $B$ is a linearly independent list of vectors (1).
        
        Moreover, let $u\in\nullla(L)$. Then, $u$ can be written as $u=\ColVecFour{v_1}{v_2}{v_3}{v_4}$, where $v_1,v_2,v_3,v_4\in\R$.
        It follows that $L(u)=\ColVecTwo{v_2+v_3-5v_4}{v_1+v_4}=$ 
    \end{enumerate}
    
\end{proof}