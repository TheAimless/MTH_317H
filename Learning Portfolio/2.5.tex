\begin{question}
    \normalfont
    This is a direct continuation of the previous LP. Assume that $L$ is the same function defined in (1).
    \begin{enumerate}[(i)]
        \item For $v_1,v_2,v_3,v_4$ defined below, prove that $\ell={v_1,v_2,v_3,v_4}$ is a basis for $\R^4$:
        \begin{equation}
            v_1=\ColVecFour{1}{0}{-5}{-1}, v_2=\ColVecFour{0}{1}{-1}{0}, v_3=\ColVecFour{0}{1}{0}{0}, v_4=\ColVecFour{1}{0}{0}{0}
        \end{equation}
        \item For $i=1,2,3,4$, compute the value of $L(v_i)$.
        \item (you do nothing for this part. it is just notation.) Let us define a notation to account for the fact that in this context, we have two different ways to represent any $x\in\R^4$: 
        \[
            \begin{aligned}
                x=\ColVecFour{x_1}{x_2}{x_3}{x_4}_{\{e_i\}} \text{is defined as } x=x_1e_1+x_2e_2+x_3e_3+x_4e_4
            \end{aligned}
        \]
        where ${e_i}$ are the canonical basis vectors as above and 
        \[
            \begin{aligned}
                x=\ColVecFour{a_1}{a_2}{a_3}{a_4}_{\{v_i\}} \text{is defined as } x=a_1v_1+a_2v_2+a_3v_3+a_4v_4
            \end{aligned}
        \]
        \item For $x=a_1v_1+a_2v_2+a_3v_3+a_4v_4$, compute $L(x)$. Give your answer as a canonical vector in $\R^2$, i.e. find the values of $y_1$ and $y_2$ so that 
        \[
            \begin{aligned}
                L(x)=\ColVecTwo{y_1}{y_2}
            \end{aligned}
        \]
        \item Find $a_1,...,a_4$ so that when 
        \[
            \begin{aligned}
                x=\ColVecFour{1}{1}{1}{1}_{\{e_i\}}, \text{you also have } x=a_1v_1+a_2v_2+a_3v_3+a_4v_4.
            \end{aligned}
        \]
        \item Given $x_1,...,x_4$ fixed, find $a_1,...,a_4$ depending only on $x_i$ so that 
        \[
            \begin{aligned}
                x=x_1e_1+x_2e_2+x_3e_3+x_4e_4=a_1v_1+a_2v_2+a_3v_3+a_4v_4.
            \end{aligned}
        \]
        Check that your answer is correct. That is to say that you need to find 
        \[
            \begin{aligned}
                a_1v_1+a_2v_2+a_3v_3+a_4v_4=\ColVecFour{x_1}{x_2}{x_3}{x_4}
            \end{aligned}
        \]
        or in our new notation that 
        \[
            \begin{aligned}
                \ColVecFour{a_1}{a_2}{a_3}{a_4}_{\{v_i\}}=\ColVecFour{x_1}{x_2}{x_3}{x_4}_{\{e_i\}}
            \end{aligned}
        \]
        \item Write down a matrix $B$ so that 
        \[
            \begin{aligned}
                &\text{if } x=\ColVecFour{x_1}{x_2}{x_3}{x_4},\text{ and } \ColVecFour{w_1}{w_2}{w_3}{w_4}=Bx\text{ as matrix multiplication,}\\
                &\text{then } x=w_1v_1+w_2v_2+w_3v_3+w_4v_4
            \end{aligned}
        \]
        \item Write down a matrix $\tilde{A}$ so that\\
        if $x\in\R^4$ and $x=\ColVecFour{a_1}{a_2}{a_3}{a_4}_{\{v_i\}}$ and $y=\ColVecTwo{y_1}{y_2}=\tilde{A}\ColVecFour{a_1}{a_2}{a_3}{a_4}$ as matrix multiplication, then $L(x)=y_1e_1+y_2e_2$, i.e. $L(x)=\ColVecTwo{y_1}{y_2}_{\{e_i\}}$.
    \end{enumerate}
\end{question}

\begin{proof}
    \renewcommand{\qedsymbol}{$\blacksquare$}
    \begin{enumerate}[(i)]
        \item Let $a_1,a_2,a_3,a_4\in\R$ such that $a_1v_1+a_2v_2+a_3v_3+a_4v_4=0$.
        Solving for the above equation, 
        \[
            \begin{aligned}
                \ColVecFour{a_1}{0}{-5a_1}{-a_1}+\ColVecFour{0}{a_2}{-a_2}{0}+\ColVecFour{0}{a_3}{0}{0}+\ColVecFour{a_4}{0}{0}{0}=\ColVecFour{a_1+a_4}{a_2+a_3}{-5a_1-a_2}{-a_1}=0
            \end{aligned}
        \]
        which reduces to
        \begin{align*}
            &\begin{cases}
                &a_1+a_4=0\\
                &a_2+a_3=0\\
                &-5a_1-a_2=0\\
                &-a_1=0
            \end{cases}\\\iff
            &a_1=a_2=a_3=a_4=0
        \end{align*}
        Hence, $\ell$ is a linearly independent list of vectors. 
        Combined with the fact that $\ell$ has a length of 4 and $\dim(\R^4)=4$, it follows that $\ell$ is a basis for $\R^4$.\qed
        \item We have: $L(v_1)=\ColVecTwo{0-5+5}{1-1}=0$, $L(v_2)=\ColVecTwo{1-1}{0}=0$, $L(v_3)=\ColVecTwo{1}{0}$ and $L(v_4)=\ColVecTwo{0}{1}$.
    \end{enumerate}
    \renewcommand{\qedsymbol}{}
\end{proof}