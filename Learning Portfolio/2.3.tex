\begin{question}
    \normalfont
    Let $L:\P_4\rightarrow\P_4$ be a linear function that has the following properties:
    \begin{enumerate}[(a)]
        \item For $p_0,p_1,p_2\in\P_4$ given by 
        \[
            \begin{aligned}
                p_0(x)=1,\ p_1(x)=x,\ p_2(x)=x^2,
            \end{aligned}
        \]
        $L$ has the values: 
        \[
            \begin{aligned}
                [L(p_0)](x)=x^4,\ [L(p_1)](x)=3,\ [L(p_2)](x)=x.
            \end{aligned}
        \]
        \item For $p_3(x)=x^3$ and $p_4(x)=x^4,L(p_3+p_4)=$ zero polynomial.
        \item $\dim(\nullla(L))=1$
    \end{enumerate}
    \begin{enumerate}[(i)]
        \item Characterize all possible linear functions, $L:\P_4\rightarrow\P_4$, that satisfy the above properties. That is to say, since ${p_0,...,p_4}$ is a basis for $\P_4$, explain all possible values of 
        \[
            \begin{aligned}
                L(p_3)\text{  and  }L(p_4)
            \end{aligned}
        \]
        \item Prove that your characterization is correct. That is to say, take a generic choice of one of the functions you gave in part (i) and prove that it does indeed satisfy conditions (a), (b), (c). Furthermore, take a generic choice of $L$ that satisfies (a), (b), (c) and show that it must also satisfy the conditions you have given in part (i).
        \item Give a basis for $\nullla(L)$ and a basis for $\range(L)$.
    \end{enumerate}
\end{question}

\begin{proof}
    \renewcommand{\qedsymbol}{$\blacksquare$}
    \begin{enumerate}[(i)]
        \item Let $L:\P_4\rightarrow\P_4$ such that $L$ satisfies (a); $L(p_3), L(p_4)\in\P_4$ such that $L(p_3)(x)=b_0+b_1x+b_2x^2+b_3x^3+b_4x^4$, $L(p_4)(x)=c_0+c_1x+c_2x^2+c_3x^3+c_4x^4$ and
        \begin{enumerate}[(1)]
            \item $b_i,c_i\in\R$ for all $i=0,...,4$
            \item $b_i+c_i=0$ for all $i=0,...,4$.
            \item $b_2,b_3$ are not all zero and $c_2,c_3$ are not all zero
        \end{enumerate}
        \item Let $L:P_4\rightarrow\P_4$ such that $L$ satisfies the conditions stated in (i). Then, $L$ satisfies (a).
        We have 
        \[
            \begin{aligned}
                L(p_3+p_4)(x) &= L(p_3)(x)+L(p_4)(x)\\
                &= (b_0+b_1x+b_2x^2+b_3x^3+b_4x^4)+(c_0+c_1x+c_2x^2+c_3x^3+c_4x^4)\\
                &= (b_0+c_0)+(b_1+c_1)x+(b_2+c_2)x^2+(b_3+c_3)x^3+(b_4+c_4)x^4\\
                &= 0
            \end{aligned}
        \]
        Thus, $L(p_3+p_4)=$ zero polynomial, or $L$ satisfies (b).
        
        To show that $\dim(\nullla(L))=1$, from the rank-nullity theorem it is equivalent to prove that 
        \[
            \begin{aligned}
                \dim(\range(L))=\dim(\P_4)-\dim(\nullla(L))=4
            \end{aligned}
        \]
        Consider the vectors $B=L(p_0),L(p_1),L(p_2),L(p_3)$. We will show that $B$ is a basis for $\range(L)$.

        Let 
    \end{enumerate}
\end{proof}