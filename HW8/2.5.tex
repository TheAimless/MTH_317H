\begin{question}
    \normalfont

    Let $A$ be the following matrix:
    \begin{align*}
        A =
        \begin{pmatrix}
            1 & 0 & 3 & 1 \\
            1 & 1 & 0 & 2 \\
            1 & 0 & 3 & 1
        \end{pmatrix}.
    \end{align*}

    \begin{enumerate}[(i)]
        \item Let $R$ be the matrix that is the reduced echelon form of $A$.  Write down $R$.  You do not need to show your steps.



        \item Define the linear function, $L_A:\real^4\to\real^3$, given by
              \begin{align*}
                  L_A(x) = Ax\ \ \ \text{as matrix multiplication in the canonical bases}.
              \end{align*}
              Use the reduction algorithm of LADR 2.31, starting with the ordered list,
              \begin{align*}
                  [L_A(e_1), L_A(e_2), L_A(e_3), L_A(e_4)],
              \end{align*}
              to give a basis for $\range(L_A)$.  You \textbf{do not} need to show your steps in the algorithm, just state the resulting basis.


        \item Let $b\in\real^3$ be given as
              \begin{align*}
                  b=\ColVecThree{4}{-1}{4}.
              \end{align*}
              Represent $b$ uniquely using the basis vectors for $\range(L_A)$ from part (ii).  Use this to solve for $x$, in the equation
              \begin{align*}
                  L_A(x) = b.
              \end{align*}

        \item For the same $b$ as in part (iii), use the row reduction algorithm in LADW section 2.2 to solve for one solution, $x$, in the equation
              \begin{align*}
                  Ax=b.
              \end{align*}
              (Because $\NullLA(L_A)\not=\{0\}$, there are infinitely many solutions, but you just need to give one here.)


        \item For the matrix, $R$, from part (i), define the linear function, $L_R:\real^4\to\real^3$, given by
              \begin{align*}
                  L_R(x) = Rx\ \ \ \text{as matrix multiplication in the canonical bases}.
              \end{align*}
              Why will solving for $x$ in the equation
              \begin{align*}
                  L_R(x)=&b
              \end{align*}
              not give the correct solution to $L_A(x)=b$?  Give a new vector, $z$, so that
              \begin{align*}
                  L_A(x)=b\ \ \ \iff\ \ \ L_R(x)=z.
              \end{align*}
    \end{enumerate}
\end{question}

\begin{proof}
    \renewcommand{\qedsymbol}{$\blacksquare$}
    \begin{enumerate}[(i)]
        \item We have: $R=\begin{pmatrix}
            1 & 0 & 3 & 1\\
            0 & 1 & -3 & 1\\
            0 & 0 & 0 & 0
        \end{pmatrix}$
        \item The resulting basis for $\range(L_A)$ is $B=\{L_A(e_1),L_A(e_2)\}$.
        \item We have: $b=\ColVecThree{4}{-1}{4}=\ColVecThree{4}{4}{4}-\ColVecThree{0}{5}{0}=4L_A(e_1)-5L_A(e_2)$.
        Since $L_A(e_3)=\ColVecThree{3}{0}{3}=3L_A(e_1)-3L_A(e_2)$ and $L_A(e_4)=\ColVecThree{1}{2}{1}=L_A(e_1)+L_A(e_2)$,
        for any $x=\ColVecFour{x_1}{x_2}{x_3}{x_4}\in\R^4$ satisfying $L_A(x)=b$, 
        \[
            \begin{aligned}
                L_A(x)&=x_1L_A(e_1)+x_2L_A(e_2)+x_3L_A(e_3)+x_4L_A(e_4)\\
                &=x_1L_A(e_1)+x_2L_A(e_2)+(3x_3L_A(e_1)-3x_3L_A(e_2))+(x_4L_A(e_1)+x_4L_A(e_2))\\
                &=(x_1+3x_3+x_4)L_A(e_1)+(x_2-3x_3+x_4)L_A(e_2)
            \end{aligned}
        \]
        But since $b$ is uniquely equal to $4L_A(e_1)-5L_A(e_2)$, it must be the case that $x_1+3x_3+x_4=4$ and $x_2-3x_3+x_4=-5$.
        Hence, $x=\ColVecFour{-3x_3-x_4+4}{3x_3-x_4-5}{x_3}{x_4}$ for some $x_3,x_4\in\R$.
        \item For $x=\ColVecFour{x_1}{x_2}{x_3}{x_4}$, solving for the equation $Ax=b$ given $b$ as in part (iii), 
        \begin{alignat*}{2}
            \begin{sysmatrix}{cccc|c}
                1 & 0 & 3 & 1 & 4\\
                1 & 1 & 0 & 2 & -1\\
                1 & 0 & 3 & 1 & 4
            \end{sysmatrix}
            \begin{aligned}
                &\ro{R_2-R_1}\\
                &\ro{R_3-R_1}
            \end{aligned}
            \begin{sysmatrix}{cccc|c}
                1 & 0 & 3 & 1 & 4\\
                0 & 1 & -3 & 1 & -5\\
                0 & 0 & 0 & 0 & 0
            \end{sysmatrix}
        \end{alignat*}
        It follows that 
        \begin{align*}
            \begin{cases}
                &x_1=-3x_3-x_4+4\\
                &x_2=3x_3-x_4-5\\
                &x_3,x_4\text{ are free}
            \end{cases}
        \end{align*}
        Let $x_3=x_4=0$. Then, $x_1=4$ and $x_2=-5$, meaning $x=\ColVecFour{4}{-5}{0}{0}$ is a solution to $Ax=b$.
        \item Let $z=\ColVecThree{4}{-5}{0}$. From the row reduction in part (iv), $x$ must be a solution to both augmented matrices.
        Hence, $x$ is a solution to both $Ax=b$ and $Rx=z$ (since the rightmost column of the row reduced matrix is $z$), or $L_A(x)=b\iff L_R(x)=z$.\qed
    \end{enumerate}
    \renewcommand{\qedsymbol}{}
\end{proof}
 