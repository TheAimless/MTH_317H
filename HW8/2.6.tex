\begin{question}\label{que:PrimePlusPrimePrime}
    \normalfont
    Define the vector space $V$ as
    \begin{align*}
        V = \{ p\in\P_4\ :\ p(0)=0 \}.
    \end{align*}
    Define the function, $L:V\to\P_3$ as
    \begin{align*}
        L(f) = f'+f''.
    \end{align*}

    \begin{enumerate}[(i)]
        \item Briefly observe / justify that
              \begin{align*}
                  V = \{ f\in\P_4,\ :\ f(x) = a_1x + a_2x^2 +a_3x^3 + a_4 x^4,\ \text{for some},\ a_1,\dots,a_4 \in\real\}.
              \end{align*}

        \item Compute $\NullLA(L)$.  Prove that your answer is correct.

        \item Prove that $L$ is a bijection.

        \item By a direct calculation, write down a formula for $L^{-1}$.  That is to say, given a generic $q\in \P_3$, with
              \begin{align*}
                  q(x)= b_0 + b_1x + b_2x^2 + b_3x^3,
              \end{align*}
              if $L^{-1}(q)=p$, with
              \begin{align*}
                  p(x) = a_1x + a_2x^2 +a_3x^3 + a_4 x^4,
              \end{align*}
              you need to specify the coefficients $a_1,\dots, a_4$ in terms of the given $b_0,\dots, b_3$.

              (Hint: you know that $L^{-1}(q)=p$ if and only if $q=L(p)$.  This should allow you to set up 4 equations in 4 unknowns-- the $a_i$-- and you should be able to solve directly for $a_i$ in terms of $b_i$.)

        \item Confirm that your answer is correct.  That is to say, you must demonstrate that $L(L^{-1}(q))=q$ and $L^{-1}(L(p))=p$.

    \end{enumerate}
\end{question}

\begin{proof}
    \renewcommand{\qedsymbol}{$\blacksquare$}
    \begin{enumerate}[(i)]
        \item Let $x\in V$. Since $x\in\P_4$, there exists $a_0,...,a_4\in\R$ such that $p(x)=a_0+a_1x+a_2x^2+a_3x^3+a_4x^4$ for all $x\in\R$.
        As $p(0)=0$, it must be the case that $a_0=0$. Hence, any $p\in\P_4$ must satisfy $p(x)=a_1x+a_2x^2+a_3x^3+a_4x^4$ for some $a_1,...,a_4\in\R$,
        or $V=\{f\in\P_4,\ :\ f(x) = a_1x + a_2x^2 +a_3x^3 + a_4 x^4,\ \text{for some},\ a_1,\dots,a_4 \in\real\}$.\qed
    \end{enumerate}
    \renewcommand{\qedsymbol}{}
\end{proof}