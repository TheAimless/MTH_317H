\begin{question}\label{que:PrimePlusPrimePrime}
    \normalfont
    Define the vector space $V$ as
    \begin{align*}
        V = \{ p\in\P_4\ :\ p(0)=0 \}.
    \end{align*}
    Define the function, $L:V\to\P_3$ as
    \begin{align*}
        L(f) = f'+f''.
    \end{align*}

    \begin{enumerate}[(i)]
        \item Briefly observe / justify that
              \begin{align*}
                  V = \{ f\in\P_4,\ :\ f(x) = a_1x + a_2x^2 +a_3x^3 + a_4 x^4,\ \text{for some},\ a_1,\dots,a_4 \in\real\}.
              \end{align*}

        \item Compute $\NullLA(L)$.  Prove that your answer is correct.

        \item Prove that $L$ is a bijection.

        \item By a direct calculation, write down a formula for $L^{-1}$.  That is to say, given a generic $q\in \P_3$, with
              \begin{align*}
                  q(x)= b_0 + b_1x + b_2x^2 + b_3x^3,
              \end{align*}
              if $L^{-1}(q)=p$, with
              \begin{align*}
                  p(x) = a_1x + a_2x^2 +a_3x^3 + a_4 x^4,
              \end{align*}
              you need to specify the coefficients $a_1,\dots, a_4$ in terms of the given $b_0,\dots, b_3$.

              (Hint: you know that $L^{-1}(q)=p$ if and only if $q=L(p)$.  This should allow you to set up 4 equations in 4 unknowns-- the $a_i$-- and you should be able to solve directly for $a_i$ in terms of $b_i$.)

        \item Confirm that your answer is correct.  That is to say, you must demonstrate that $L(L^{-1}(q))=q$ and $L^{-1}(L(p))=p$.

    \end{enumerate}
\end{question}

\begin{proof}
    \renewcommand{\qedsymbol}{$\blacksquare$}
    \begin{enumerate}[(i)]
        \item Let $x\in V$. Since $x\in\P_4$, there exists $a_0,...,a_4\in\R$ such that $p(x)=a_0+a_1x+a_2x^2+a_3x^3+a_4x^4$ for all $x\in\R$.
        As $p(0)=0$, it must be the case that $a_0=0$. Hence, any $p\in\P_4$ must satisfy $p(x)=a_1x+a_2x^2+a_3x^3+a_4x^4$ for some $a_1,...,a_4\in\R$,
        or $V=\{f\in\P_4,\ :\ f(x) = a_1x + a_2x^2 +a_3x^3 + a_4 x^4,\ \text{for some},\ a_1,\dots,a_4 \in\real\}$.\qed
        \item Let $f\in V$. Then, there exists $a_1,...,a_4\in\R$ such that $f(x)=a_1x+a_2x^2+a_3x^3+a_4x^4$ for all $x\in\R$.
        It follows that $f'(x)=a_1+2a_2x+3a_3x^2+4a_4x^3$ and $f''(x)=2a_2+6a_3x+12a_4x^2$ for all $x\in\R$, whence $(L(f))(x)=(a_1+2a_2)+(2a_2+6a_3)x+(3a_3+12a_4)x^2+4a_4x^3$.
        Solving for $L(f)=0$,
        \begin{align*}
            &\begin{cases}
                &a_1+a_2=0\\
                &2a_2+6a_3=0\\
                &3a_3+12a_4=0\\
                &4a_4=0
            \end{cases}\\\iff
            &a_1=a_2=a_3=a_4=0
        \end{align*}
        Hence, any $f\in V$ such that $L(f)=0$ must satisfy $f=0$, or $\nullla(L)=\{0\}$.
        \item From LADR Theorem 3.16 Section 3B, since $\nullla(L)=\{0\}$ it follows that $L$ is injective.
        Let $g\in\P_3$. Then, there exist $b_1,...,b_4$ such that $g(x)=b_1+b_2x+b_3x^2+b_4x^3$ for all $x\in\R$.
        Let $f\in\P_4$ such that 
        \[
            \begin{aligned}
                f(x)=(b_1-b_2+2b_3-6b_4)x+\dfrac{b_2-2b_3+6b_4}{2}x^2+\dfrac{b_3-3b_4}{3}x^3+\dfrac{b_4}{4}x^4
            \end{aligned}
        \]
        for all $x\in\R$. Then, 
        \[
            \begin{aligned}
                f'(x)=(b_1-b_2+2b_3-6b_4)+(b_2-2b_3+6b_4)x+(b_3-3b_4)x^2+b_4x^3
            \end{aligned}
        \]
        and 
        \[
            \begin{aligned}
                f''(x)=(b_2-2b_3+6b_4)+(2b_3-6b_4)x+3b_4x^2
            \end{aligned}
        \]
        Moreover, since the free coefficient of $f$ is zero, $f\in V$.
        It follows that $L(f)$ is defined and $(L(f))(x)=f'(x)+f''(x)=b_1+b_2x+b_3x^2+b_4x^3=g(x)$ for all $x\in\R$.
        Hence, any element in the codomain has a preimage in the domain, or $f$ must be surjective.
        Therefore, $f$ is a bijection.\qed
        
        \item Let $q\in\P_3$ such that there exist $b_0,...,b_3\in\R$ satisfying $q(x)=b_0+b_1x+b_2x^2+b_3x^3$ for all $x\in\R$. Let $L^{-1}:\P_3\rightarrow V$ be given by 
        \[
            \begin{aligned}
                (L^{-1}(q))(x)=(b_0-b_1+2b_2-6b_3)x+\dfrac{b_1-2b_2+6b_3}{2}x^2+\dfrac{b_2-3b_3}{3}x^3+\dfrac{b_3}{4}x^4
            \end{aligned}
        \]
        for all $q\in\P_3$ and $x\in\R$.
        \item Let $p\in V$ and $q\in\P_3$ be arbitrary.
        Then, there exists $a_1,...,a_4\in\R$ and $b_0,...,b_3\in\R$ such that 
        \[
            \begin{aligned}
                p(x)=a_1x+a_2x^2+a_3x^3+a_4x^4
            \end{aligned}
        \]
        and 
        \[
            \begin{aligned}
                q(x)=b_0+b_1x+b_2x^2+b_3x^3
            \end{aligned}
        \]
        for all $x\in\R$.
        We have 
        \[
            \begin{aligned}
                L(L^{-1}(q))&=L(L^{-1}(b_0+b_1x+b_2x^2+b_3x^3))\\
                &= L((b_0-b_1+2b_2-6b_3)x+\frac{b_1-2b_2+6b_3}{2}x^2+\frac{b_2-3b_3}{3}x^3+\frac{b_3}{4}x^4)\\
                &= (b_0-b_1+2b_2-6b_3)+(b_1-2b_2+6b_3)x+(b_2-3b_3)x^2+b_3x^3\\
                &+(b_1-2b_2+6b_3)+(2b_2-6b_3)x+3b_3x^2\\
                &= b_0+b_1x+b_2x^2+b_3x^3=q
            \end{aligned}
        \]
        and 
        \[
            \begin{aligned}
                L^{-1}(L(p))&=L^{-1}((a_1+2a_2)+(2a_2+6a_3)x+(3a_3+12a_4)x^2+4a_4x^3)\\
                &=((a_1+2a_2)-(2a_2+6a_3)+(6a_3+24a_4)-24a_4)\\
                &+(\frac{(2a_2+6a_3)-(6a_3+24a_4)+24a_4}{2})x+\frac{(3a_3+12a_4)-12a_4}{3}x^2+a_4x^3\\
                &= a_1+a_2x+a_3x^2+a_4x^3=p
            \end{aligned}
        \]
        Therefore, for all $p\in V$ and $q\in\P_3$, $L(L^{-1})(q)=q$ and $L^{-1}(L(p))=p$, or $L^{-1}$ is the inverse of $L$.\qed
    \end{enumerate}
    \renewcommand{\qedsymbol}{}
\end{proof}