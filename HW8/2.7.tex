\begin{question}
    \normalfont

    Assume that $V$ and $L$ are as in Question \ref{que:PrimePlusPrimePrime}. Let $p_0,\dots, p_3$ be the canonical basis polynomials for $\P_3$
    \begin{align*}
        p_0(x) = 1,\ \ \
        p_1(x) = x,\ \ \
        p_2(x) = x^2,\ \ \
        p_3(x) = x^3.
    \end{align*}
    Let the following $q_1,\dots, q_4$ be a choice of basis for $V$:
    \begin{align*}
        q_1(x) = x,\ \ \
        q_2(x) = x^2,\ \ \
        q_3(x) = x^3,\ \ \
        q_4(x) = x^4.
    \end{align*}

    \begin{enumerate}[(i)]
        \item Write down a matrix, $A$, so that in the bases for $V$ and $\P_3$ given above,
              \begin{align*}
                   & \text{if}\ \
                  p = a_1q_1+ \cdots + a_4 q_4,\ \ \text{i.e.}\ \ p=\ColVecThree{a_1}{\vdots}{a_4}_{[q_i]}, \\
                   & \ \ \text{and}\ \
                  L(p) = b_0p_0 + \cdots + b_3p_3 = \ColVecThree{b_0}{\vdots}{b_3}_{[p_i]},                 \\
                   & \text{then}
                  \ColVecThree{b_0}{\vdots}{b_3}
                  = A\ColVecThree{a_1}{\vdots}{a_4}\ \ \text{as matrix multiplication}
              \end{align*}

        \item Using the algorithm of LADW Chp 2, Sec 4, compute the inverse matrix, $A^{-1}$.  Confirm by matrix multiplication that your answer is correct.  No need to show your steps, as you have done a similar computation in Question \ref{que:MatrixInverseShowSteps}.  Just write the matrix and confirm.

        \item  For the sake of comparing methods, do this part pretending you have not done Question 3.6 yet.


              Using the matrix $A^{-1}$, write down the formula for $L^{-1}$.  Given that a linear function is uniquely determined by its action on a basis, it is OK to specify $L^{-1}$ by specifying $L^{-1}(p_i)$.  Or, you can just write down a formula for $L^{-1}(q)$ for a generic $q$.
    \end{enumerate}
\end{question}

\begin{proof}
    \renewcommand{\qedsymbol}{$\blacksquare$}
    \begin{enumerate}[(i)]
        \item 
        The matrix $A$ for $L$ with respect to the basis $\{q_1,q_2,q_3,q_4\}$ for $V$ and the canonical basis for $\P_3$ is given by
        \[
            \begin{aligned}
                A=\begin{pmatrix}
                    1 & 2 & 0 & 0\\
                    0 & 2 & 6 & 0\\
                    0 & 0 & 3 & 12\\
                    0 & 0 & 0 & 4
                \end{pmatrix}
            \end{aligned}
        \]
        \item Let $A^{-1}=\begin{pmatrix}
            1 & -1 & 2 & -6\\
            0 & \frac{1}{2} & -1 & 3\\
            0 & 0 & \frac{1}{3} & -1\\
            0 & 0 & 0 & \frac{1}{4}
        \end{pmatrix}$. Then, 
        \[
            \begin{aligned}
                AA^{-1}&=\begin{pmatrix}
                    1 & 2 & 0 & 0\\
                    0 & 2 & 6 & 0\\
                    0 & 0 & 3 & 12\\
                    0 & 0 & 0 & 4
                \end{pmatrix}
                \begin{pmatrix}
                    1 & -1 & 2 & -6\\
                    0 & \frac{1}{2} & -1 & 3\\
                    0 & 0 & \frac{1}{3} & -1\\
                    0 & 0 & 0 & \frac{1}{4}
                \end{pmatrix}\\
                &=\begin{pmatrix}
                    1 & -1+1 & 2-2 & -6+6\\
                    0 & -2+3 & -2+2 & 6-6\\
                    0 & 0 & 1 & -3+3\\
                    0 & 0 & 0 & 1
                \end{pmatrix}\\
                &=\begin{pmatrix}
                    1 & 0 & 0 & 0\\
                    0 & 1 & 0 & 0\\
                    0 & 0 & 1 & 0\\
                    0 & 0 & 0 & 1
                \end{pmatrix}
            \end{aligned}
        \]
        and 
        \[
            \begin{aligned}
                A^{-1}A&=\begin{pmatrix}
                    1 & -1 & 2 & -6\\
                    0 & \frac{1}{2} & -1 & 3\\
                    0 & 0 & \frac{1}{3} & -1\\
                    0 & 0 & 0 & \frac{1}{4}
                \end{pmatrix}
                \begin{pmatrix}
                    1 & 2 & 0 & 0\\
                    0 & 2 & 6 & 0\\
                    0 & 0 & 3 & 12\\
                    0 & 0 & 0 & 4
                \end{pmatrix}\\
                &=\begin{pmatrix}
                    1 & 2-2 & -6+6 & 24-24\\
                    0 & 1 & 3-3 & -12+12\\
                    0 & 0 & 1 & 4-4\\
                    0 & 0 & 0 & 1
                \end{pmatrix}\\
                &=\begin{pmatrix}
                    1 & 0 & 0 & 0\\
                    0 & 1 & 0 & 0\\
                    0 & 0 & 1 & 0\\
                    0 & 0 & 0 & 1
                \end{pmatrix}
            \end{aligned}
        \]
        Therefore, $AA^{-1}=A^{-1}A=id_4$, where $id_4$ is the $4\times 4$ identity matrix.\qed
        \item The matrix $A^{-1}$ is the matrix for $L^{-1}$ with respect to the canonical basis for $\P_3$ and the basis $\{q_1,q_2,q_3,q_4\}$ for $V$.
        It follows that $L^{-1}$ must satisfy 
        \[
            \begin{aligned}
                &L^{-1}(p_0)=q_1,\ &&L^{-1}(p_1)=-q_1+\frac{1}{2}q_2\\ 
                &L^{-1}(p_2)=2q_1-q_2+\frac{1}{3}q_3,\ &&L^{-1}(p_3)=-6q_1+3q_2-q_3+\frac{1}{4}q_4
            \end{aligned}
        \]\qed
    \end{enumerate}
    \renewcommand{\qedsymbol}{}
\end{proof}