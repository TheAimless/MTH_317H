\begin{question}
	\normalfont
	Assume that the list $B= q_1, \dots, q_m $ \hspace{.1cm} is a basis for $\P_4$.
	\begin{enumerate}[(i)]
		\item Is it possible that no polynomial in $B$ has degree 4? Prove your answer.
		\item Is it possible that no polynomial in $B$ has degree 2? Prove your answer.
		\item What must be the value of $m$ (the length of the list, $B$)?  Prove your answer. \\
	\end{enumerate}
\end{question}
\renewcommand{\qedsymbol}{$\blacksquare$}
\begin{proof}
    \begin{enumerate}[(i)]
        \item No. Assume the contrary. Then, there exists a solution to the equation $x^4=a_1q_1(x)+a_2q_2(x)+...+a_mq_m(x)$ for some $a_1,a_2,...,a_m\in\R$ and the coefficient for $x^4$ of $q_1,...,q_m$ is $0$.
        However, that means the coefficients of $x^4$ for $a_1q_1(x),a_2q_2(x),...,a_mq_m(x)$ must be 0, or 
        \item Let $q_1,q_2,q_3,q_4,q_5\in\P_4$ such that $q_1(x)=1,q_2(x)=x,q_3(x)=x^3-x^2,q_4(x)=x^3+x^2,q_5(x)=x^4$. Let $B=q_1,q_2,q_3,q_4,q_5$. Then, no polynomial in $B$ has degree 2. We will prove that $B$ is a basis for $P_4$.
        
        To show that $B$ is linearly independent, let $a_1,a_2,a_3,a_4,a_5\in\R$ such that $a_1q_1+a_2q_2+a_3q_3+a_4q_4+a_5q_5=\vec{0}$. It follows that 
        \[
            \begin{aligned}
                0
                &= a_1q_1(x)+a_2q_2(x)+a_3q_3(x)+a_4q_4(x)+a_5q_5(x)\\
                &= a_1+a_2x+a_3(x^3-x^2)+a_4(x^3+x^2)+a_5x^4\\
                &= a_1+a_2x+(a_4-a_3)x^2+(a_3+a_4)x^3+a_5x^4
            \end{aligned}
        \]
        , which reduces to
        \begin{align*}
            &\begin{cases}
                a_1 &= 0\\
                a_2 &= 0\\
                a_4-a_3 &= 0\\
                a_4+a_3 &= 0\\
                a_5 &= 0   
            \end{cases}\\\iff
            &a_1=a_2=a_3=a_4=a_5=0
        \end{align*}
        Thus, the only solution to $a_1q_1+a_2q_2+a_3q_3+a_4q_4+a_5q_5=\vec{0}$ is the trivial solution, or B is a linearly independent list.

        Moreover, since $\dim(\P_4)=5$ and $B$ has a length of 5, from Theorem 2.39 Section 2C LADR it follows that $B$ is a basis for $\P_4$.
        Therefore, it is possible that no polynomial in $B$ has degree 2.\qed
        \item Since $\dim(\P_4)=5$ and 
    \end{enumerate}
    \renewcommand{\qedsymbol}{}
\end{proof}