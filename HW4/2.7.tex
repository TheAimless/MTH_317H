\begin{question}\label{que:GeneralVectorSpaceNewBasisFromOld}
	\normalfont
	
	Prove that in any vector space, $V$, if the ordered list, $B=v_1, v_2, v_3, v_4$ \hspace{.1cm} is a basis for $V$, then for 
	\begin{align*}
		C= w_1, w_2, w_3, w_4 ,
	\end{align*}
	where
	\begin{align*}
		w_1=v_1+v_2,\ \  w_2=v_2+v_3,\ \  w_3=v_2,\ \  w_4=v_2 + v_4,
	\end{align*}
	$C$ is also a basis for $V$.  \\
	
	\noindent Prove this by two different methods, as follows:
	
	\begin{enumerate}[(i)]
		\item Use the definitions of linear independence and span to show that $C$ is a basis. (Hint: for span, try to show that $v_i\in\SpanLA(C)$.)
		
		\item Use the results of LADR 2.C to show that $C$ is a basis of $V$.
	\end{enumerate}
\end{question}

\begin{proof}
    \renewcommand{\qedsymbol}{$\blacksquare$}
    \begin{enumerate}[(i)]
        \item Let $a_1,a_2,a_3,a_4\in\R$ be solution to $a_1w_1+a_2w_2+a_3w_3+a_4w_4=\vec{0}$. We have 
        \[
            \begin{aligned}
                \vec{0} &= a_1w_1+a_2w_2+a_3w_3+a_4w_4\\
                        &= a_1(v_1+v_2)+a_2(v_2+v_3)+a_3v2+a_4(v_2+v_4)\\
                        &= a_1v_1+(a_1+a_2+a_3)v_2+a_2v_3+a_4v_4      
            \end{aligned}
        \]
        However, since $v_1,v_2,v_3,v_4$ is a basis for $V$, the only solution to $a_1v_1+(a_1+a_2+a_3)v_2+a_2v_3+a_4v_4=\vec{0}$ is $a_1=a_1+a_2+a_3=a_2=a_4=0$, or $a_1=a_2=a_3=a_4=0$.
        Hence, the only solution to $a_1w_1+a_2w_2+a_3w_3+a_4w_4=\vec{0}$ is the trivial solution, or $C$ is a linearly independent list (1).
        
        Let $x\in V$ be arbitrary. Since $v_1,v_2,v_3,v_4$ is a basis for $V$, there exist $a_1,a_2,a_3,a_4\in\R$ such that $a_1v_1+a_2v_2+a_3v_3+a_4v_4=x$.
        We have 
        \[
            \begin{aligned}
                &\quad \ a_1w_1+a_2w_2-(a_1+a_2-a_3+a_4)w_3+a_4w_4\\
                &= a_1(v_1+v_2)+a_3(v_2+v_3)-(a_1-a_2+a_3+a_4)w_3+a_4(v_2+v_4)\\
                &= a_1v_1+(a_1+a_3+a_4-a_1-a_3-a_4+a_2)v_2+a_3v_3+a_4v_4\\
                &= a_1v_1+a_2v_2+a_3v_3+a_4v_4\\
                &= x
            \end{aligned}
        \]
        Hence, $x$ can be written as a linear combination of $w_1,w_2,w_3,w_4$, or $x\in\SpanLA(C)$. But since $x\in V$ arbitrary, it follows that $C$ spans $V$ (2).

        From (1) and (2), it follows that $C$ is a basis for $V$.\qed
        \item We have already proven that $C$ is linearly independent in (i). Since $B$ is a basis for $V$ and $B$ has four vectors, $\dim(V)=4$. However, because $C$ also contains four vectors, from Theorem 2.39 Section 2C LADR, it follows that $C$ must be a basis for $V$.\qed
    \end{enumerate}
    \renewcommand{\qedsymbol}{}
\end{proof}