\begin{question}
	\normalfont
	
	Define the list, $\ell=x_1,x_2,x_3$ \hspace{.1cm} of vectors in $\real^3$, where 
	\begin{align*}
		x_1 = \ColVecThree{1}{2}{3}\ \ 
		x_2 = \ColVecThree{1}{2}{1}\ \
		x_3 = \ColVecThree{1}{0}{1}.
	\end{align*}
	You will prove that $\ell$ is a basis for $\real^3$ via two different methods.
	\begin{enumerate}[(i)]
		\item Prove, from the definitions of linear independence and span that $\ell$ is linearly independent and $\SpanLA(\ell)=\real^3$, hence that $\ell$ is a basis of $\real^3$.  For the proof of the span, you must use the typical set equality proof, in which you will establish that $\SpanLA(\ell)\subseteq \real^3$ and $\real^3\subseteq \SpanLA(\ell)$.  Note, for your span proof, it may be easiest to first show that the canonical basis vectors satisfy $e_1,e_2,e_3\in \SpanLA(\ell)$.
		
		\item Use some combination of the results in LADR 2.C to prove that $\ell$ is a basis for $\real^3$.
		
		\item Do you have a preference for either method?
		
	\end{enumerate}

\end{question}
\begin{proof}
    \renewcommand{\qedsymbol}{$\blacksquare$}
    \begin{enumerate}[(i)]
        \item We have:
        \begin{align*}
            \begin{cases}
                &-0.5x_1+0.5x_2+x_3
                =\ColVecThree{-0.5}{-1}{-1.5}+\ColVecThree{0.5}{1}{0.5}+\ColVecThree{1}{0}{1}
                =\ColVecThree{1}{0}{0}
                =e_1\\
                &0.5x_2-0.5x_3
                =\ColVecThree{0.5}{1}{0.5}-\ColVecThree{0.5}{0}{0.5}
                =\ColVecThree{0}{1}{0}
                =e_2\\
                &0.5x_1-0.5x_2
                =\ColVecThree{0.5}{1}{1.5}-\ColVecThree{0.5}{1}{0.5}
                =\ColVecThree{0}{0}{1}
                =e_3
            \end{cases}
        \end{align*}
        Thus, $e_1,e_2,e_3\in\SpanLA(\ell)$. However, since $e_1,e_2,e_3$ is the canonical basis of $\R^3$, any vector in $\R^3$ can be written as a linear combination of $e_1,e_2,e_3$. Hence, any vector in $\R^3$ can be written as a linear combination of $x_1,x_2,x_3$, or $\R^3\subseteq\SpanLA(\ell)$.
        In addition, because $x_1,x_2,x_3\in\R^3$, $\SpanLA(\ell)\subseteq\R^3$. Therefore, $\SpanLA(\ell)=\R^3$ (1).

        Moreover, to show that $x_1,x_2,x_3$ are linearly independent vectors, let $a,b,c\in\R$ be solution to $ax_1+bx_2+cx_3=\vec{0}$. We have 
        \[
            \begin{aligned}
                \ColVecThree{0}{0}{0}
                &= ax_1+bx_2+cx_3\\
                &= \ColVecThree{a}{2a}{3a}+\ColVecThree{b}{2b}{b}+\ColVecThree{c}{0}{c}\\
                &= \ColVecThree{a+b+c}{2a+2b}{3a+b+c}
            \end{aligned}
        \]
        , which can be reduced to
        \begin{align*}
            &\begin{cases}
                a+b+c  &= 0\\
                2a+2b  &= 0\\
                3a+b+c &= 0
            \end{cases}\\\iff
            &\begin{cases}
                a+b+c  &= 0\\
                a+b    &= 0\\
                2a     &= 0
            \end{cases}\\\iff
            &\begin{cases}
                c      &= 0\\
                a      &= 0\\
                b      &= 0
            \end{cases}
        \end{align*}
        Hence, the only solution to the equation $ax_1+bx_2+cx_3=\vec{0}$ is the trivial solution ($a=b=c=0$), or $x_1,x_2,x_3$ are linearly independent (2). 

        From (1) and (2), it follows that $\ell$ is a basis of $\R^3$.\qed
        \item We have: $\dim(\R^3)=3$. Then, for $\ell$ to be a basis for $\R^3$, $\ell$ must have a length of 3 and be linearly independent (Theorem 2.39 Section 2C LADR). However, since $\ell$ comprises three vectors and is linearly independent as proven in part (i), it follows that $\ell$ forms a basis of $\R^3$.\qed
        \item The method stated in part (ii) is preferable as the list length requirement is both easier to check than the spanning requirement.
    \end{enumerate}
    \renewcommand{\qedsymbol}{}
\end{proof}