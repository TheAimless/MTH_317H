
\begin{question}\label{que:ReductionAlgorithm}
	\normalfont
	
	
	Let $\ell=v_1,\dots, v_6$ and $s=w_1\dots,w_6$ be ordered lists of vectors in $\real^3$, where:
	\begin{align*}
		v_1 = \ColVecThree{1}{2}{3}\ \ 
		v_2 = \ColVecThree{1}{1}{1}\ \
		v_3 = \ColVecThree{3}{4}{5}\ \ 
		v_4 = \ColVecThree{2}{1}{2}\ \ 
		v_5 = \ColVecThree{0}{0}{1}\ \ 
		v_6 = \ColVecThree{1}{0}{3}
	\end{align*}
	and
	\begin{align*}
		w_1 = \ColVecThree{1}{0}{3}\ \ 
		w_2 = \ColVecThree{0}{0}{1}\ \
		w_3 = \ColVecThree{1}{2}{3}\ \ 
		w_4 = \ColVecThree{1}{1}{1}\ \ 
		w_5 = \ColVecThree{3}{4}{5}\ \ 
		w_6 = \ColVecThree{2}{1}{2}.
	\end{align*}
	
	\vspace{.2cm}
	
	Recall that the vectors in the canonical basis for $\real^3$ are
	
	\begin{align*}
		e_1 = \ColVecThree{1}{0}{0}\ \
		e_2 = \ColVecThree{0}{1}{0}\ \
		e_3 = \ColVecThree{0}{0}{1}. 
	\end{align*}
	
	\vspace{.2cm}
	\begin{enumerate}[(i)]
		\item Prove that $e_1,e_2,e_3\in\SpanLA(\ell)$.
		\item Prove that $\SpanLA(\ell)=\real^3$.
		\item Use the reduction algorithm in LADR Result 2.31 to reduce the list $\ell$ to a basis.  What is the resulting list?    For this you must show each step of the algorithm, including the calculation of whether or not a vector is in the span of the previous vectors.  Then just state the resulting list.  Give a one sentence explanation for why your answer for the span of the resulting list is what it is.
		\item Use the reduction algorithm in LADR Result 2.31 to reduce the list $s$ to a basis.  What is the resulting list?  Is it the same list as the algorithm applied to $\ell$?  Does order matter in this algorithm?  Does the ordering of the original vectors affect the span of the reduced list?
	\end{enumerate}
\end{question}

\begin{proof}
    \renewcommand{\qedsymbol}{$\blacksquare$}
    \begin{enumerate}[(i)]
        \item We have: $-3v_5+v_6=\ColVecThree{0}{0}{-3}+\ColVecThree{1}{0}{3}=\ColVecThree{1}{0}{0}=e_1, 2v_2-v_4=\ColVecThree{2}{2}{2}-\ColVecThree{2}{1}{2}=\ColVecThree{0}{1}{0}=e_2,v_5=\ColVecThree{0}{0}{1}=e_3$. Therefore, $e_1,e_2,e_3$ can be written as linear combinations of vectors in $\ell$, or $e_1,e_2,e_3\in\SpanLA(\ell)$\qed
        \item Since $e_1,e_2,e_3\in\SpanLA(\ell)$, any linear combination of $e_1,e_2,e_3$ can be written as a linear combination of vectors from $\ell$. However, since $e_1,e_2,e_3$ is also the canonical basis for $\R^3$, any vector in $\R^3$ can be written as a linear combination of $e_1,e_2,e_3$. Thus, any vector in $\R^3$ can be written as a linear combination of vectors from $\ell$, or $\R^3\subseteq\SpanLA(\ell)$. But since every vector in $\ell$ belongs to $\R^3$, $\SpanLA(\ell)\subseteq\R^3$. Therefore, $\SpanLA(\ell)=\R^3$.\qed
        \item Since $\dim(\R^3)=3$, our final list should contain 3 vectors forming a basis for $\R^3$. Following the reduction algorithm in LADR Result 2.31, we have:
        \begin{enumerate}[Step 1:]
            \item Since $v_1\not=\vec{0}$, we shall keep $v_1$ in our final list.
            \item Since the first component of $v_2$ equals to that of $v_1$ but not the second component, $v_2$ is not a scalar multiple of $v_1$. Hence, $v_2\not\in\SpanLA(v_1)$, or $v_2$ is in our final list.
            \item Since $v_1+2v_2=\ColVecThree{1}{2}{3}+\ColVecThree{2}{2}{2}=\ColVecThree{3}{4}{5}=v_3$, $v_3\in\SpanLA(v_1,v_2)$. Thus, $v_3$ is not in our final list.
            \item We will prove that $v_4\not\in\SpanLA(v_1,v_2)$. Assume the contrary. Then, there exist $a,b\in\R$ such that $av_1+bv_2=v_4$. This becomes 
            \[
                \begin{aligned}
                    \ColVecThree{2}{1}{2} &= a \ColVecThree{1}{2}{3}+b \ColVecThree{1}{1}{1}\\
                                          &= \ColVecThree{a+b}{2a+b}{3a+b}
                \end{aligned}
            \]
            , which can be reduced to
            \begin{align*}
                &\begin{cases}
                    a+b  &= 2\\
                    2a+b &= 1\\
                    3a+b &= 2\\
                \end{cases}\\\implies
                &\begin{cases}
                    a    &= -1\\
                    a    &= 1
                \end{cases}
            \end{align*}
            , which is impossible. Hence, $v_4\not\in\SpanLA(v_1,v_2)$, or $v_4$ is in our final list.
        \end{enumerate}
        Since we have chosen three vectors for our list, the algorithm ends here. Therefore, our final list is $\ell'=v_1,v_2,v_4$ and $\SpanLA(\ell')=\R^3$ since $v_1,v_2,v_4$ form a basis for $\R^3$.
        \item The final list should contain three vectors as constructed below:
        \begin{enumerate}[Step 1:]
            \item Since $w_1\not=\vec{0}$, $w_1$ belongs in our final list.
            \item Let $k\in\R$ such that $w_2=kw_1$. Then, $\ColVecThree{0}{0}{1}=k\ColVecThree{1}{0}{3}=\ColVecThree{k}{0}{3k}$
            , which reduces to
            \begin{align*}
                &\begin{cases}
                    k &= 0\\
                    3k&= 1
                \end{cases}\\\iff
                &\begin{cases}
                    k &= 0\\
                    k &= \frac{1}{3}
                \end{cases}
            \end{align*}
            , which is impossible. Therefore, $w_2\not\in\SpanLA(w_1)$, or $w_2$ belongs in our final list.
            \item Since the second component of $w_3$ is non-zero but the second components of both $w_1$ and $w_2$ are zero, there cannot exist $a,b\in\R$ such that $aw_1+bw_2=w_3$. Hence, $w_3\not\in\SpanLA(w_1,w_2)$, or $w_3$ belongs in our final list.
        \end{enumerate}
        Since we have chosen three vectors for our list, the algorithm ends here. Therefore, our final list is $s'=w_1,w_2,w_3$ and $\SpanLA(s')=\R^3$ since $w_1,w_2,w_3$ form a basis for $\R^3$.
        
    \end{enumerate}
    \renewcommand{\qedsymbol}{}
\end{proof}