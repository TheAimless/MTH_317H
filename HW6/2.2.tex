\begin{question}\label{que:LemmaGenericSols}
	\normalfont
	
	In this question, provide a proof of the following important lemma. \\
	
	\textbf{Lemma:} Assume that $L:V\to W$ is linear, that $y\in\range(L)$. If $x_0\in V$ with $L(x_0)=y$, then 
	\begin{align*}
		\forall\ x\in\{ v\in V\ :\ L(v)=y \},\ \exists\ z\in \NullLA(L),\ \ \text{with}\ x=x_0+z.
	\end{align*}
\end{question}

\begin{proof}
    \renewcommand{\qedsymbol}{$\blacksquare$}
    Let $x\in\{v\in V:L(v)=y\}$ and $z=x-x_0\in V$.
    Then, $L(x)=L(x_0)=y$.
    Since $L$ is linear, 
    \[
        \begin{aligned}
            L(z)
            &= L(x-x_0)\\
            &= L(x)-L(x_0)\\
            &= y-y = 0
        \end{aligned}
    \]
    Therefore, $z\in\nullla(L)$, or for every $x\in\{v\in V:L(v)=y\}$, there exists $z\in\nullla(L)$ such that $x=x_0+z$.
\end{proof}