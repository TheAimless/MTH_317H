\begin{question}
    \normalfont




    Let $L:\P_5\to P_4$ be the linear function given by
    \begin{align*}
        L(p) = 7p'' + 3p'.
    \end{align*}
    \begin{enumerate}[(i)]
        \item Give a list of polynomials, $\ell = \{n_1,\dots, n_m\}$ so that $\ell$ is a basis for $\NullLA(L)$.  Prove that your choice is correct.


        \item Prove that given any polynomial, $q\in \P_4$, there exists $p\in\P_5$ so that $L(p)=q$.  (Hint, think about surjectivity. You should be able to do dimension counting and use LADR result 3.22.  I would suggest \emph{not} to prove the existence of $q$ from scratch.)

        \item For a given $q$ as in the previous step, is the choice of $p$ unique?  Explain your answer.

    \end{enumerate}
\end{question}

\begin{proof}
    \renewcommand{\qedsymbol}{$\blacksquare$}
    \begin{enumerate}[(i)]
        \item Consider the vector $p_0\in\P_5$ such that $p_0(x)=1$. Let $\ell=p_0$.
              Then, $L(p_0)(x)=7(1)''+3(1)'=0$, or $p_0\in\nullla(L)$.
              Thus, $\ell$ is a linearly independent list of vectors in $\nullla(L)$ (1).

              Let $q\in\nullla(L)$ such that $q(x)=ax^5+bx^4+cx^3+dx^2+ex+f$ for some $a,b,c,d,e,f\in\R$.
              Then,
              \[
                  \begin{aligned}
                      (Lq)(x)
                       & = 7q''(x)+3q'(x)                                      \\
                       & = 7(20ax^3+12bx^2+6cx+2d)+3(5ax^4+4bx^3+3cx^2+2dx+e)  \\
                       & = 15ax^4+(140a+12b)x^3+(84b+9c)x^2+(42c+6d)x+(14d+3e)
                  \end{aligned}
              \]
              Solving for $Lq=0$,
              \begin{align*}
                   & \quad \ \begin{cases}
                                 15a      & = 0 \\
                                 140a+12b & = 0 \\
                                 84b+9c   & = 0 \\
                                 42c+6d   & = 0 \\
                                 14d+e    & = 0
                             \end{cases} \\
                   & \iff a=b=c=d=e=0
              \end{align*}
              Thus, $q(x)=f=fp_0(x)$, or $q=fp_0$ for all $q\in\nullla(L)$. Therefore, $\ell$ spans $\nullla(L)$ (2).

              From (1) and (2), it follows that $\ell$ is a basis for $\nullla(L)$.\qed
        \item Since $\ell$ is a basis for $\nullla(L)$ and $\ell$ has one vector, $\dim(\nullla(L))=1$.
              From the rank-nullity theorem, since $\dim(P_5)=\dim(\range(L))+\dim(\nullla(L))$, \begin{equation*}
                  \begin{aligned}
                      \dim(\nullla(L))=6-1=5
                  \end{aligned}
              \end{equation*}
              Let $B$ be a basis for $\range(L)$, which has $5$ vectors (since $\dim(\range(L))=5$).
              By definition, $\range(L)$ is a subspace of the codomain, which is $P_4$.
              Thus, the vectors in $B$ must be vectors in $\P_4$.

              Moreover, $B$ is a linearly independent list of vectors.
              Hence, $B$ is a linearly independent list of vectors in $P_4$ of size 5, meaning $B$ is a basis for $P_4$.

              Let $x\in\P_4$. Then, $x$ can be uniquely written as a linear combination of vectors in $B$.
              But since $B$ is also a basis for $\range(L)$, such linear combination belongs to $\range(L)$, or $x\in\range(L)$.
              Thus, $\P_4\subseteq\range(L)$.
              Combined with the fact that $\range(L)$ is a subspace of $P_4$, meaning $\range(L)\subseteq P_4$, it follows that $\range(L)=P_4$.

              Since $\range(L)=P_4$, it follows that $L$ is surjective, or for every $q\in\P_4$ there exists $p\in\P_5$ such that $L(p)=q$.\qed

        \item Let $q\in\P_4$ such that $q=0$. Define $p_1,p_2\in\P_5$ such that $p_1(x)=1$ and $p_2(x)=2$ for all $x\in\R$.
        Then, \begin{equation*}
            \begin{aligned}
                L(p_1)(x)=7(1)''+3(1)'=0
            \end{aligned}
        \end{equation*}
        and \begin{equation*}
            \begin{aligned}
                L(p_2)(x)=7(2)''+3(3)'=0
            \end{aligned}
        \end{equation*}
        Thus, the choice of $p$ is not unique given $q=0$.
    \end{enumerate}
    \renewcommand{\qedsymbol}{}
\end{proof}