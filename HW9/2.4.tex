\begin{question}
	\normalfont 
	\begin{enumerate}[(i)]
	
	\item A square matrix with entries in the real numbers $Q$ is called \emph{orthogonal} if $Q^TQ = I$. Here $Q^T$ denotes the transpose of $Q$. Prove that if $Q$ is an orthogonal matrix, then 
	\[
	\textup{det}(Q) = \pm 1. 
	\]
	
	\item A square matrix is called \emph{nilpotent} if $A^k =0$ for some positive integer $k$. Prove that for a nilpotent matrix $A$, det$(A)=0$. 
	\end{enumerate}
	\end{question}

\begin{proof}
    \renewcommand{\qedsymbol}{$\blacksquare$}
    \begin{enumerate}[(i)]
        \item Since $\det Q^T=\det Q$ from Property $3.6.10$ Section 3.4 LADW, it follows that 
        \[
            \begin{aligned}
                \det I=\det(Q^TQ)=\det(Q^T)\det(Q)=\det(Q)^2
            \end{aligned}
        \]
        or $\det(Q)^2=1$. Therefore, $\det(Q)=\pm 1$.\qed
        \item We will prove the following lemma:\\
        \textit{Lemma: }Let $A$ be an $n\times n$ matrix. Then, for any $k\in\N$, 
        \[
            \begin{aligned}
                \det(A^k)=(\det A)^k
            \end{aligned}
        \]
        \textit{Proof: } For any $k\in\N$, let $P(k)$ be defined as 
        \[
            \begin{aligned}
                P(k)\iff \det(A^k)=(\det A)^k
            \end{aligned}
        \]
        Base case: When $k=1$, $\det(A^1)=\det A=(\det A)^1$. Thus, $P(1)$ holds.

        Induction step: Assume $P(k)$ is true for some $k=t\in\N$.
        Then, $\det(A^t)=(\det A)^t$.
        When $k=t+1$, from Theorem 3.5 Section 3.3 LADW, $\det(A^tA)=\det A^t\det A$ and
        \[
            \begin{aligned}
                \det(A^{t+1})=\det(A^tA)=\det A^t\det A=(\det A)^t\det A=(\det A)^{t+1}
            \end{aligned}
        \]
        or $P(k+1)$ holds.
        From the principle of mathematical induction, it follows that 
        \[
            \begin{aligned}
                \det(A^k)=(\det A)^k
            \end{aligned}
        \]
        for all $k\in\N$.\qed

        Back to the original problem, since $A^k=0$, 
        \[
            \begin{aligned}
                0=\det(A^k)=(\det A)^k
            \end{aligned}
        \]
        by the lemma, or $\det A=0$. Therefore, for a nilpotent matrix $A$, $\det A=0$.\qed
    \end{enumerate}
    
    \renewcommand{\qedsymbol}{}
\end{proof}
