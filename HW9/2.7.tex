\begin{question}
\normalfont
Let $T:\real^3\to\real^3$ be the linear function defined as
		\begin{align*}
			T(\ColVecThree{x_1}{x_2}{x_3}) = \ColVecThree{x_1-x_2}{-x_1+2x_2-2x_3}{2x_2 + x_3}.
\end{align*}

\begin{enumerate}[(i)]

\item Let $A$ denote the matrix of the linear transformation $T$ with respect to the standard basis of $\real^3$. Find the matrix $A$, and find det($A$). \\
\item Consider the basis 
\[
\ColVecThree{1}{1}{1}, \ColVecThree{0}{1}{1},
 \ColVecThree{1}{0}{1}
\]
of $\real^3$.  Let $B$ denote the matrix of the transformation $T$ with respect to this basis for both the domain and the target. Find the matrix $B$, and find det($B$). \\
\item Considering your answers to parts (i) and (ii) above, make a conjecture about how the determinant of matrix representing a linear transformation depends on the bases of the domain and target. You do not need to prove your conjecture. 
\end{enumerate}
\end{question}

\begin{proof}
    \renewcommand{\qedsymbol}{$\blacksquare$}
    \begin{enumerate}[(i)]
        \item We have 
        \[
            \begin{aligned}
                T(e_1)=\ColVecThree{1}{-1}{0}=e_1-e_2,\
                T(e_2)=\ColVecThree{-1}{2}{2}=-e_1+2e_2+2e_3
            \end{aligned}
        \] 
        and $T(e_3)=\ColVecThree{0}{-2}{1}=-2e_2+e_3$.
        Therefore, the matrix $A$ for $T$ with respect to the standard basis of $\R^3$ is $A=\begin{pmatrix}
            1 & -1 & 0\\
            -1 & 2 & -2\\
            0 & 2 & 1
        \end{pmatrix}$
        
        We have 
        \[
            \begin{aligned}
                \det A&=(-1)^{1+1}A_{1,1}-(-1)^{1+2}A_{1,2}
                =\begin{vmatrix}
                    2 & -2\\
                    2 & 1
                \end{vmatrix}
                +\begin{vmatrix}
                    -1 & -2\\
                    0 & 1
                \end{vmatrix}\\
                &=(2+4)+(-1-0)=5
            \end{aligned}
        \]
        Therefore, $\det A=5$.
        \item Let $v_1=\ColVecThree{1}{1}{1},v_2=\ColVecThree{0}{1}{1},v_3=\ColVecThree{1}{0}{1}$.
        Let $B=\{v_1,v_2,v_3\}$ be a basis for $\R^3$.
        Then, 
        \[
            \begin{aligned}
                T(v_1)=\ColVecThree{0}{-1}{3}=-4v_1+3v_2+4v_3,\
                T(v_2)=\ColVecThree{-1}{0}{3}=-4v_1+4v_2+3v_3
            \end{aligned}
        \]
        and $T(v_3)=\ColVecThree{1}{-3}{1}=-3v_1+4v_3$.
        Hence, the matrix $B$ for $T$ with respect to $B$ is given by $B=\begin{pmatrix}
            -4 & -4 & -3\\
            3 & 4 & 0\\
            4 & 3 & 4
        \end{pmatrix}$

        Moreover, 
        \[
            \begin{aligned}
                \det B
                &=3(-1)^{2+1}B_{2,1}+4(-1)^{2+2}B_{2,2}\\
                &=-3\begin{vmatrix}
                    -4 & -3\\
                    3 & 4
                \end{vmatrix}
                +4\begin{vmatrix}
                    -4 & -3\\
                    4 & 4
                \end{vmatrix}\\
                &=-3(-16+9)+4(-16+12)=5
            \end{aligned}
        \]
        Hence, $\det B=5$.\qed
        \item Conjecture: the determinant of matrix representing a linear transformation does not depend on the bases of the domain and target.
    \end{enumerate}
    
    \renewcommand{\qedsymbol}{}
\end{proof}
