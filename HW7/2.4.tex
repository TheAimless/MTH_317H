\begin{question}
\normalfont
\noindent Without citing LADR result 3.59, show explicitly that  $\P_3$ is isomorphic to the vector space of $2\times2$ matrices with entries in the real numbers, $M_{2\times 2}(\real)$. In other words, define linear maps (and prove that they are linear)
\[
T: \P_3 \to M_{2\times2}(\real)
\]
and 
\[
S: M_{2\times2}(\real) \to \P_3
\]
such that $ST$ is the identity map on $\P_3$ and $TS$ is the identity map on $M_{2\times2}(\real)$.  \\ 


\end{question}

\begin{proof}
    \renewcommand{\qedsymbol}{$\blacksquare$}
    Let $P=\{p_0,p_1,p_2,p_3\}$ be the canonical basis for $\P_3$ such that for all $x\in\R$, $p_0(x)=1$, $p_1(x)=x$, $p_2(x)=x^2$ and $p_3(x)=x^3$.

    Let $m_1=\MatTwoTwo{1}{0}{0}{0},m_2=\MatTwoTwo{0}{1}{0}{0},m_3=\MatTwoTwo{0}{0}{1}{0},m_4=\MatTwoTwo{0}{0}{0}{1}$ and\\
    $M=\{m_1,m_2,m_3,m_4\}$. Let $c_1,c_2,c_3,c_4\in\R$ such that $c_1m_1+c_2m_2+c_3m_3+c_4m_4=0$. Then 
    \[
        \begin{aligned}
            \MatTwoTwo{0}{0}{0}{0} 
            = c_1m_1+c_2m_2+c_3m_3+c_4m_4
            = \MatTwoTwo{c_1}{c_2}{c_3}{c_4}
        \end{aligned}
    \]
    This implies that $c_1=c_2=c_3=c_4=0$, or $M$ is a linearly independent list of vectors in $M_{2\times 2}(\R)$.
    Moreover, since $\dim(M_{2\times 2}(\R))=4$ and $M$ contains four vectors, from Theorem 2.39 Section 2C LADR it follows that $M$ is a basis for $M_{2\times 2}(\R)$.
    
    Let $a,b,c,d\in\R$. 
    Let $p\in\P_3$ and $m\in M_{2\times 2}(\R)$ such that $p(x)=a+bx+cx^2+dx^3$ for all $x\in\R$ and $m=\MatTwoTwo{a}{b}{c}{d}$.
    Define $T:\P_3\to M_{2\times 2}(\R)$ and $S:M_{2\times 2}(\R)\to\P_3$ to be
    \[
        \begin{aligned}
            T(p)=m \text{ and } S(m)=p
        \end{aligned}
    \]
    for all $a,b,c,d\in\R$. Then, $T$ and $S$ are well-defined.

    To show that $T$ is linear, let $q_1,q_2\in\P_3$ such that
    $q_1(x)=a_0+a_1x+a_2x^2+a_3x^3$ and $q_2(x)=b_0+b_1x+b_2x^2+b_3x^3$ for all $x\in\R$.
    Let $\lambda\in\R$.
    By definition, it follows that $T(q_1)=\MatTwoTwo{a_0}{a_1}{a_2}{a_3}$ and $T(q_2)=\MatTwoTwo{b_0}{b_1}{b_2}{b_3}$.
    We have 
    \[
        \begin{aligned}
            T(q_1+q_2)=\MatTwoTwo{a_0+b_0}{a_1+b_1}{a_2+b_2}{a_3+b_3}=\MatTwoTwo{a_0}{a_1}{a_2}{a_3}+\MatTwoTwo{b_0}{b_1}{b_2}{b_3}=T(q_1)+T(q_2)
        \end{aligned}
    \]
    and 
    \[
        \begin{aligned}
            T(\lambda q_1)=\MatTwoTwo{\lambda a_0}{\lambda a_1}{\lambda a_2}{\lambda a_3}=\lambda \MatTwoTwo{a_0}{a_1}{a_2}{a_3}=\lambda T(q_1)
        \end{aligned}
    \]
    Thus, $T$ is a linear transformation since $T$ satisfies additivity and homogeneity.

    Similarly, let $n_1=\MatTwoTwo{a_0}{a_1}{a_2}{a_3}$, $n_2=\MatTwoTwo{b_0}{b_1}{b_2}{b_3}$ and $\lambda\in\R$ be arbitrary.
    Then, $S(n_1)=q_1\in\P_3$ and $S(n_2)=q_2\in\P_3$ such that $q_1(x)=a_0+a_1x+a_2x^2+a_3x^3$ and $q_2(x)=b_0+b_1x+b_2x^2+b_3x^3$ for all $x\in\R$.
    We have
    \[
        \begin{aligned}
            &(S(n_1+n_2))(x)=(a_0+b_0)+(a_1+b_1)x+(a_2+b_2)x^2+(a_3+b_3)x^3\\
            &=(a_0+a_1x+a_2x^2+a_3x^3)+(b_0+b_1x+b_2x^2+b_3x^3)\\
            &=(S(n_1))(x)+(S(n_2))(x)
        \end{aligned}
    \]
    and 
    \[
        \begin{aligned}
            S(\lambda n_1)(x)&=\lambda a_0+\lambda a_1x+\lambda a_2x^2+\lambda a_3x^3\\
            &=\lambda(a_0+a_1x+a_2x^2+a_3x^3)=\lambda S(n_1)
        \end{aligned}
    \]
    Hence, $S$ is a linear transformation since $S$ satisfies additivity and homogeneity.

    Consider the composite mapping $ST$ and $TS$. Since $T:\P_3\rightarrow M_{2\times 2}(\R)$ and $S:M_{2\times 2}(\R)\rightarrow\P_3$ are both linear,
    $ST\in\mathcal{L}(\P_3,\P_3)$ and $TS\in\mathcal{L}(M_{2\times 2}(\R), M_{2\times 2}(\R))$.

    Let $n\in M_{2\times 2}(\R)$. 
    Then, for $S(n)=q\in\P_3$, by the definition of $T$ it follows that $(TS)(n)=T(S(n))=T(q)=s$, or $TS\in\mathcal{L}(M_{2\times 2}(\R), M_{2\times 2}(\R))$ is the identity map on $M_{2\times 2}(\R)$.

    Similarly, let $q\in\P_3$.
    Then, for $T(q)=n\in\M_{2\times 2}(\R)$, by the definition of $S$ it follows that $(ST)(q)=S(T(q))=S(n)=q$, or $ST\in\mathcal{L}(\P_3,\P_3)$ is the identity mapping on $\P_3$.
    Therefore, $T$ is invertible, meaning $T$ is an isomorphism from $\P_3$ to $M_{2\times 2}(\R)$, or $\P_3$ is isomorphic to $M_{2\times 2}(\R)$.
\end{proof}