\begin{question}

\normalfont
(a) Consider the linear map $D: \P_3 \to \P_2$ given by differentiation. In other words $(Dp)(x) = p'(x)$. Find a basis of $\P_3$ and a basis of $\P_2$ such that the matrix of $D$ with respect to these bases is 
\[
\left( \begin{array}{rrrr}
1 & 0 & 0 & 0 \\
0 & 1 & 0 & 0 \\
0 & 0 & 1 & 0 \\
\end{array}\right)
\]


\vspace{.4cm}

\noindent (b) Suppose $V$ and $W$ are finite-dimensional vector spaces, and $T$ is a linear map $T: V \to W$. Prove that there exists a basis of $V$ and a basis of $W$ such that all entries of the matrix of $T$ with respect to these bases are 0 except that the entry in row $j$, column $j$ is 1 for all $1\leq j \leq \dim (\textup{range} T$). 

\end{question}
\begin{proof}
    \renewcommand{\qedsymbol}{$\blacksquare$}
    \begin{enumerate}[(a)]
    \item
    Let $P=\{p_0,p_1,p_2,p_3\}$ such that for every $x\in\R$, $p_0(x)=x^3$, $p_1(x)=x^2$, $p_2(x)=x$ and $p_3(x)=1$.
    Then, $P$ is the canonical basis for $\P_3$ by definition.
    Let $Q=\{q_0,q_1,q_2\}\subseteq\P_2$ such that for every $x\in\R$, $q_0(x)=3x^2$, $q_1(x)=2x$ and $q_2(x)=1$.
    Let $c_0,c_1,c_2\in\R$ be the solution to $c_0q_0+c_1q_1+c_2q_2=0$.
    Then 
    \[
        \begin{aligned}
            &c_0q_0(x)+c_1q_1(x)+c_2q_2(x)=0\\
            \Leftrightarrow\
            &3c_0x^2+2c_1x+c_2=0\\
            \Leftrightarrow\
            &3c_0=2c_1=c_2=0\\
            \Leftrightarrow\
            &c_0=c_1=c_2=0
        \end{aligned}
    \]
    Thus, the only solution to $c_0q_0+c_1q_1+c_2q_2=0$ is $c_0=c_1=c_2=0$, or $Q$ is a linearly independent list of vectors.
    Combined with the fact that $\dim(\P_2)=3$, which is equal to the length of list $Q$, it follows that $Q$ is a basis for $\P_2$.

    Consider the linear map $D$. For every $x\in\R$, 
    \[
        \begin{aligned}
            &(Dp_0)(x)=p_0'(x)=3x^2=q_0(x)+0q_1(x)+0q_2(x)\\
            &(Dp_1)(x)=p_1'(x)=2x=0q_0(x)+q_1(x)+0q_2(x)\\
            &(Dp_2)(x)=p_2'(x)=1=0q_0(x)+0q_1(x)+q_2(x)\\
            &(Dp_3)(x)=p_3'(x)=0=0q_0(x)+0q_1(x)+0q_2(x)
        \end{aligned}
    \]
    Therefore, for given $P$ and $Q$, the matrix of $D$ with respect to these bases is 
    \[
        \begin{pmatrix}
            1 & 0 & 0 & 0\\
            0 & 1 & 0 & 0\\
            0 & 0 & 1 & 0
        \end{pmatrix}    
    \]\qed
    \item Let $n=\dim(V)$ and $m=\dim(W)$.
    Let $\dim(\nullla(T))=k$ and $B_u=\{u_1,u_2,...,u_k\}$ be a basis for $\nullla(T)$.
    Since $B_u$ is a basis for $\nullla(T)$, $B_u$ is a linearly independent list in $V$.
    From Theorem 2.33 Section 2B LADR, $B_u$ can be extended to a basis for $V$.

    Let $B_V=\{v_1,...,v_{n-k},u_1,...,u_k\}$ be a basis for $V$ after extending $B_u$.
    Then, $\spanvect(B_V)=V$. From the nice lemma, we have 
    \[
        \begin{aligned}
            T(\spanvect(B_V))=\spanvect(T(v_1),...,T(v_{n-k}),T(u_1),...,T(u_k))
        \end{aligned}
    \]
    But since $T(\spanvect(B_V))=T(V)=\range(T)$ and $T(u_i)=0$ for all $i=1,...,k$, it follows that $\range(T)=\spanvect(T(v_1),...,T(v_{n-k}),0,...,0)$, or the vectors $T(v_1),...,T(v_{n-k})$ span $\range(T)$.
    Moreover, from the rank-nullity theorem, $\dim(\range(T))=\dim(V)-\dim(\nullla(T))=n-k$.
    Therefore, from Theorem 2.42 Section 2C LADR, since $T(v_1),...,T(v_{n-k})$ is a spanning list of $\range(T)$ of the right length, it follows that $T(v_1),...,T(v_{n-k})$ is a basis for $\range(T)$.

    Let $w_i=T(v_i)$ for all $i=1,...,n-k$ and $B_w=\{w_i\ |\ i=1,...,n-k\}$.
    By definition, $B_w$ is a basis for $\range(T)$.
    Since $B_w$ is a basis for $\range(T)$ and $\range(T)$ is a subspace of $W$, $B_w$ must be a linearly independent list in $W$.
    Thus, $B_w$ can be extended to a basis for $W$ from Theorem 2.33 Section 2B LADR.

    Let $B_W=\{w_1,...,w_{n-k},w_{n-k+1},...,w_m\}$ be a basis for $W$ after extending $B_w$.
    Consider the matrix $M$ for $T$ with respect to $B_V$ and $B_W$. We have
    \begin{align*}
        \begin{cases}
            &T(v_1)=w_1=1w_1+0w_2+...+0w_m\\
            &T(v_2)=w_2=0w_1+1w_2+...+0w_m\\
            &\vdots\\
            &T(v_{n-k})=w_{n-k}=0w_1+0w_2+...+1w_{n-k}+0w_{n-k+1}+...+0w_m\\
            &T(u_1)=0=0w_1+0w_2+...+0w_m\\
            &\vdots\\
            &T(u_k)=0=0w_1+0w_2+...+0w_m
        \end{cases}
    \end{align*}
    , or such matrix $M$ has all entries 0 except that the entry in row $j$, column $j$ is 1 for all $1\leq j\leq \dim(\range(T))$.\qed
    \end{enumerate}
    \renewcommand{\qedsymbol}{}
\end{proof}