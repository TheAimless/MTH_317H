\begin{question}

	\normalfont

	Define the function, $L:\P_4\to \P_2$ by the following rule:
	\begin{align*}
		\text{for}\ p(x) = a_0 + a_1x + a_2x^2 + a_3x^3 + a_4 x^4,\ \ \
		(Lp)(x) = a_2 + a_3 x + a_4 x^2.
	\end{align*}


	\begin{enumerate}[(i)]

		\item Prove that $U=\{p\in\P_4\ :\ Lp = \text{zero polynomial}\}$ is a subspace of $\P_4$. \\
		      \emph{Reminder: you cannot use results from LADR 3.B for these questions.}

		\item Give a basis for $U$. (No proof necessary.)

		\item Using the definition of an injective function (see Section 2 above), prove that $L$ is not injective.

		\item Using the definition of a surjective function (See Section 2 above), prove that $L$ is surjective.
	\end{enumerate}
\end{question}

\begin{proof}
	\renewcommand{\qedsymbol}{$\blacksquare$}
	\begin{enumerate}[(i)]
		\item We will first prove that $L$ is linear. Let $a,b\in\P_4$ and $c\in\R$ such that $a(x)=a_0+a_1x+a_2x^2+a_3x^3+a_4x^4$ and $b(x)=b_0+b_1x+b_2x^2+b_3x^3+b_4x^4$. Then,
		      \begin{equation*}
			      \begin{aligned}
				      (L(a+b))(x)
				       & = (a_0+b_0)+(a_1+b_1)x+(a_2+b_2)x^2+(a_3+b_3)x^3+(a_4+b_4)x^4     \\
				       & = (a_0+a_1x+a_2x^2+a_3x^3+a_4x^4)+(b_0+b_1x+b_2x^2+b_3x^3+b_4x^4) \\
				       & =(La)(x)+(Lb)(x)
			      \end{aligned}
		      \end{equation*}
		      and
		      \begin{equation*}
			      \begin{aligned}
				      (L(ca))(x)
				       & = ca_0+ca_1x+ca_2x^2+ca_3x^3+ca_4x^4        \\
				       & = c(a_0+a_1x+a_2x^2+a_3x^3+a_4x^4)=c(La)(x)
			      \end{aligned}
		      \end{equation*}
		      Thus, L satisfies both the additivity and homogeneity properties, or $L$ is linear.

		      To show that $U$ is a subspace of $\P_4$, we will show that the zero vector of $\P_4$ is in $U$, $U$, vector addition is closed under $U$ and scalar multiplication is closed under $U$.
		      Let $z=0+0x+0x^2+0x^3+0x^4\in\P_4$. Then, $(Lz)(x)=0+0x+0x^2=0$, or $Lz$ equals to the zero polynomial. Hence, the zero polynomial of $\P_4$ belongs to $U$ (1).

		      Let $u,v\in U$.
			  Then, $u,v\in\P_4$ and $Lu=Lv=\vec{0}$.
			  But since $\P_4$ is a vector space, $u+v\in\P_4$.
			  Combined with the fact that $L$ is linear, $L(u+v)=Lu+Lv=\vec{0}$. 
			  It follows that $L(u+v)$ equals to the zero polynomial. 
			  But since $u+v\in\P_4$ and $U$ is the set of all the vectors $p$ of $\P_4$ satisfying $Lp=\vec{0}$, it follows that $u+v\in U$.
			  Hence, vector addition is closed under $U$ (2).

			  Let $u\in U$ and $c\in\R$.
			  Then, $u\in\P_4$ and $Lu=c(Lu)=\vec{0}$.
			  But since $\P_4$ is a vector space, $cu\in\P_4$.
			  Combined with the fact that $L$ is linear, $L(cu)=c(Lu)=\vec{0}$.
			  It follows that $L(cu)$ equals to the zero polynomial.
			  But since $cu\in\P_4$ and $U$ is the set of all the vectors $p$ of $\P_4$ satisfying $Lp=\vec{0}$, it follows that $cu\in U$.
			  Hence, scalar multiplication is closed under $U$ (3).
			  
			  From (1), (2) and (3), it follows that $U$ is a subspace of $\P_4$.\qed
			  \item Let $B=b_1,b_2$ where $b_1,b_2\in\P_4$, $b_1(x)=1$ and $b_2(x)=x$. Then, $B$ is a basis for $U$.
			  \item Let $p,q\in\P_4$ such that $p(x)=1$ and $q(x)=x$. Then, $p\not=q$. But since $(Lp)(x)=0=(Lq)(x)$, $Lp=Lq$. From the definition of an injective function, if $L$ is injective then for every $u,v\in\P_4$, if $u\not=v$ then $Lu\not=Lv$. But since $p\not=q$ and $Lp=Lq$, it follows that $L$ cannot be injective.\qed
			  \item Let $v\in\P_2$ such that $v(x)=d_2+d_3x+d_4x^2$. 
			  Consider the following polynomial: $u(x)=d_2x^2+d_3x^3+d_4x^4$. Then, $u\in\P_4$ and $(Lu)(x)=d_2+d_3x+d_4x^2=v(x)$, or $v=Lu$. Therefore, for any $v\in\P_2$ there exists $u\in\P_4$ such that $Lu=v$, or L is surjective.\qed
	\end{enumerate}
	\renewcommand{\qedsymbol}{}
\end{proof}