\begin{question}\label{que:SpanIntersection}
    \normalfont

    \begin{enumerate}[(i)]
        \item Let $V$ be a vector space, and let $A$ = $\{v_1,\dots, v_m\}$ and $B=\{ w_1, \dots, w_n\}$.  Prove that
              \begin{align*}
                  \SpanLA(A\intersect B) \subseteq \SpanLA(A)\intersect\SpanLA(B).
              \end{align*}


        \item Give an example for $V=\real^3$ of two \emph{non-empty} sets $A$ and $B$ as above, in which
              \begin{align*}
                  \SpanLA(A)\intersect \SpanLA(B) \not\subseteq \SpanLA(A\intersect B).
              \end{align*}


    \end{enumerate}
\end{question}

\begin{proof}
    \begin{enumerate}[(i)]
        \renewcommand{\qedsymbol}{$\blacksquare$}
        \item Let $C=A\intersect B=\{u_1,...,u_p\}$ where $p$ is non-negative and $u_1,...,u_p\in V$.
        Then, for any $x\in\SpanLA(A\intersect B)$, $x$ can be written as $x=a_1u_1+...+a_pu_p$, with $a_1,...,a_p\in\R$. 

        Since $\{u_1,...,u_p\}=A\intersect B$, $u_1,...,u_p$ are elements of $A$. Hence, $x$ is a linear combination of vectors in $A$, or $x\in\SpanLA(A)$.
        Similarly, $x\in\SpanLA(B)$. Therefore, $x\in\SpanLA(A)\intersect\SpanLA(B)$, or $\SpanLA(A\intersect B)\subseteq\SpanLA(A)\intersect\SpanLA(B)$.\qed
        \item Let $A=\left\{ \ColVecThree{1}{1}{1}\right \}$ and $B=\left\{\ColVecThree{2}{2}{2}\right\}$. Then, $A\intersect B=\emptyset$, or $\SpanLA(A\intersect B)=\emptyset$.
        
        However, since both $\SpanLA(A)$ and $\SpanLA(B)$ are subspaces of $V$, the zero vector $z\in V$ belongs to both $\SpanLA(A)$ and $\SpanLA(B)$. Therefore, $\SpanLA(A)\intersect\SpanLA(B)\not=\emptyset=\SpanLA(A\intersect B)$, or $\SpanLA(A)\intersect\SpanLA(B)\not\subseteq\SpanLA(A\intersect B)$.
    \end{enumerate}
\end{proof}