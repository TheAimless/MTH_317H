\begin{question}
    \normalfont


    Consider the same list of vectors $v_1, v_2, v_3, v_4$ \hspace{.1cm} as in Question \ref{que:LinDepInR3}.
    \begin{enumerate}[(i)]
        \item Prove that $\SpanLA(v_1, v_2, v_3, v_4) = \SpanLA(v_1, v_2, v_3)$.

        \item Prove that $\SpanLA(v_1, v_2, v_3) = \real^3$.

              Here, you must use the basic set equality proof structure, namely to show as sets that $A=B$, you must demonstrate $A\subseteq B$ and $B\subset A$.

        \item Prove that $\SpanLA(v_1, v_2)\not=\real^3$.
    \end{enumerate}
\end{question}

\renewcommand{\qedsymbol}{$ $}
\begin{proof}
    \begin{enumerate}[(i)]
        \renewcommand{\qedsymbol}{$\blacksquare$}
        \item To show that $\SpanLA(v_1,v_2,v_3,v_4)=\SpanLA(v_1,v_2,v_3)$, we need to show that $\SpanLA(v_1,v_2,v_3,v_4)\subseteq\SpanLA(v_1,v_2,v_3)$ and $\SpanLA(v_1,v_2,v_3)\subseteq\SpanLA(v_1,v_2,v_3,v_4)$.
        
        Let $x\in\SpanLA(v_1,v_2,v_3)\subseteq\R^3$. Then, there exist $\lam_1,\lam_2,\lam_3\in\F$ such that
        \[
            \begin{aligned}
                x=\lam_1v_1+\lam_2v_2+\lam_3v_3
            \end{aligned}    
        \]
        Choose $\lam_4=0\in\F$. Then, $x=\lam_1v_1+\lam_2v_2+\lam_3v_3+\lam_4v_4$. Hence, $x\in\SpanLA(v_1,v_2,v_3,v_4)$, or $\SpanLA(v_1,v_2,v_3)\subseteq\SpanLA(v_1,v_2,v_3,v_4)$. (1)
        
        Conversely, let $x\in\SpanLA(v_1,v_2,v_3,v_4)$. Then, there exist $\lam_1,\lam_2,\lam_3,\lam_4$ such that
        \[
            \begin{aligned}
                x=\lam_1v_1+\lam_2v_2+\lam_3v_3+\lam_4v_4
            \end{aligned}
        \]  
        Let $\lam_2'=\lam_2+\lam_4\in\F$ and $\lam_3'=\lam_3+2\lam_4\in\F$. Then
        \[
            \begin{aligned}
                x &= \lam_1v_1+\lam_2v_2+\lam_3v_3+\lam_4v_4\\
                  &= \lam_1v_1+(\lam_2'-\lam_4)v_2+(\lam_3'-2\lam_4)v_3+\lam_4v_4\\
                  &= \lam_1v_1+\lam_2'v_2+\lam_3'v_3+\lam_4(-v_2-2v_3+v_4)
            \end{aligned}  
        \]
        However, since $v_4-v_2-2v_3=\vec{0}$, it follows that $x=\lam_1'v_1+\lam_2'v_2+\lam_3'v_3\in\SpanLA(v_1,v_2,v_3)$. Hence, $\SpanLA(v_1,v_2,v_3,v_4)\subseteq\SpanLA(v_1,v_2,v_3)$. (2)

        Therefore, from (1) and (2), it follows that $\SpanLA(v_1,v_2,v_3,v_4)=\SpanLA(v_1,v_2,v_3)$.\qed
        \item To show that $\SpanLA(v_1,v_2,v_3)=\R^3$, we need to prove that $\SpanLA(v_1,v_2,v_3)\subseteq\R^3$ and $\R^3\subseteq\SpanLA(v_1,v_2,v_3)$.

        From Theorem 2.7 in LADR section 2A, it follows that $\SpanLA(v_1,v_2,v_3)$ is a subspace of $\R^3$, or $\SpanLA(v_1,v_2,v_3)\subseteq\R^3$ (1).

        Conversely, let $x=\ColVecThree{x_1}{x_2}{x_3}\in\R^3$, where $x_1,x_2,x_3\in\R$. Let $\lam_1=x_2-x_3+x_1,\lam_2=\frac{x_3-x_1}{2},\lam_3=\frac{x_1-2x_2+x_3}{2}$. Then,
        \[
            \begin{aligned}
                \lam_1v_1+\lam_2v_2+\lam_3v_3 &= (x_2-x_3+x_1)\ColVecThree{1}{1}{1}+\frac{x_3-x_1}{2}\ColVecThree{1}{2}{3}+\frac{x_1-2x_2+x_3}{2}\ColVecThree{1}{0}{1}\\
                                              &= \ColVecThree{x_2-x_3+x_1+\frac{x_3-x_1}{2}+\frac{x_1+x_3}{2}-x_2}{x_2-x_3+x_1+x_3-x_1}{x_2-x_3+x_1+\frac{3(x_3-x_1)}{2}+\frac{x_1+x_3}{2}-x_2}\\
                                              &= \ColVecThree{x_1}{x_2}{x_3}=x
            \end{aligned}  
        \]
        Thus, $x\in\SpanLA(v_1,v_2,v_3)$, or $\R^3\subseteq\SpanLA(v_1,v_2,v_3)$ (2). Therefore, from (1) and (2), it follows that $\R^3=\SpanLA(v_1,v_2,v_3)$.\qed
        \item Let $x=\ColVecThree{1}{3}{4}\in\R^3$. Suppose there exist $\lam_1,\lam_2\in\R$ satisfying $\lam_1v_1+\lam_2v_2=x$. We have
        \[
            \begin{aligned}
                \ColVecThree{1}{3}{4} &= \ColVecThree{\lam_1}{\lam_1}{\lam_1}+\ColVecThree{\lam_2}{2\lam_2}{3\lam_2}\\
                                      &= \ColVecThree{\lam_1+\lam_2}{\lam_1+2\lam_2}{\lam_1+3\lam_2}
            \end{aligned}  
        \]
        This equation can be reduced to
        \begin{align*}
            &\begin{cases}
                \lam_1+\lam_2  &= 1\\
                \lam_1+2\lam_2 &= 3\\
                \lam_1+3\lam_2 &= 4
            \end{cases}\\\iff
            &\begin{cases}
                \lam_1         &= 1\\
                \lam_2         &= 2\\
                \lam_2         &= 1
            \end{cases}
        \end{align*}
        , which is impossible (since $\lam_2$ cannot be equal to both 1 and 2). Thus, there does not exist $\lam_1,\lam_2\in\R$ satisfying $\lam_1v_1+\lam_2v_2=x$, or $x\not\in\SpanLA(v_1,v_2)$. 
        
        Therefore, $\SpanLA(v_1,v_2)\not\subseteq\R^3$, or $\SpanLA(v_1,v_2)\not=\R^3$.\qed
    \end{enumerate}
\end{proof}