\begin{question}
    \normalfont


    Consider the same list of vectors $v_1, v_2, v_3, v_4$ \hspace{.1cm} as in Question \ref{que:LinDepInR3}.
    \begin{enumerate}[(i)]
        \item Prove that $\SpanLA(v_1, v_2, v_3, v_4) = \SpanLA(v_1, v_2, v_3)$.

        \item Prove that $\SpanLA(v_1, v_2, v_3) = \real^3$.

              Here, you must use the basic set equality proof structure, namely to show as sets that $A=B$, you must demonstrate $A\subseteq B$ and $B\subset A$.

        \item Prove that $\SpanLA(v_1, v_2)\not=\real^3$.
    \end{enumerate}
\end{question}

\renewcommand{\qedsymbol}{$ $}
\begin{proof}
    \begin{enumerate}[(i)]
        \renewcommand{\qedsymbol}{$\blacksquare$}
        \item To show that $\SpanLA(v_1,v_2,v_3,v_4)=\SpanLA(v_1,v_2,v_3)$, we need to show that $\SpanLA(v_1,v_2,v_3,v_4)\subseteq\SpanLA(v_1,v_2,v_3)$ and $\SpanLA(v_1,v_2,v_3)\subseteq\SpanLA(v_1,v_2,v_3,v_4)$.
        
        Let $x\in\SpanLA(v_1,v_2,v_3)$. Then, there exist $\lam_1,\lam_2,\lam_3\in\F$ such that
        \[
            \begin{aligned}
                x=\lam_1v_1+\lam_2v_2+\lam_3v_3
            \end{aligned}    
        \]
        Choose $\lam_4=0\in\F$, and $x=\lam_1v_1+\lam_2v_2+\lam_3v_3+\lam_4v_4$. Hence, $x\in\SpanLA(v_1,v_2,v_3,v_4)$, or $\SpanLA(v_1,v_2,v_3)\subseteq\SpanLA(v_1,v_2,v_3,v_4)$. (1)
        
        Conversely, let $x\in\SpanLA(v_1,v_2,v_3,v_4)$. Then, there exist $\lam_1,\lam_2,\lam_3,\lam_4$ such that
        \[
            \begin{aligned}
                x=\lam_1v_1+\lam_2v_2+\lam_3v_3+\lam_4v_4
            \end{aligned}
        \]  
        Let $\lam_2=\lam_2'-\lam_4$ and $\lam_3=\lam_3'-2\lam_4$, where $\lam_2',\lam_3'\in\F$. Then
        \[
            \begin{aligned}
                x &= \lam_1v_1+\lam_2v_2+\lam_3v_3+\lam_4v_4\\
                  &= \lam_1v_1+(\lam_2'-\lam_4)v_2+(\lam_3'-2\lam_4)v_3+\lam_4v_4\\
                  &= \lam_1v_1+\lam_2'v_2+\lam_3'v_3+\lam_4(-v_2-2v_3+v_4)
            \end{aligned}  
        \]
        However, since $v_4-v_2-2v_3=\vec{0}$, it follows that $x=\lam_1'v_1+\lam_2'v_2+\lam_3'v_3\in\SpanLA(v_1,v_2,v_3)$. Hence, $\SpanLA(v_1,v_2,v_3,v_4)\subseteq\SpanLA(v_1,v_2,v_3)$. (2)

        Therefore, from (1) and (2), it follows that $\SpanLA(v_1,v_2,v_3,v_4)=\SpanLA(v_1,v_2,v_3)$.\qed
        \item To show that $\SpanLA(v_1,v_2,v_3)=\R^3$, we need to prove that $$
    \end{enumerate}
\end{proof}