\begin{question}\label{que:SpanAgain}
    \normalfont

    Assume that  $u,v,w_1,\dots,w_n$ are all distinct elements of a vector space, $V$.  Define the lists of vectors, $B$ and $C$, as
    \begin{align*}
        B =  u, w_1,\dots, w_n \ \ \ \text{and}\ \ \
        C =  v, w_1,\dots, w_n .
    \end{align*}
    Prove the following implication:
    \begin{align*}
        \text{if}\ B\ \text{is a linearly independent list and}\  u\in\SpanLA(C),\ \ \text{then}\ \ \SpanLA(B)=\SpanLA(C).
    \end{align*}
\end{question}

\renewcommand{\qedsymbol}{$\blacksquare$}
\begin{proof}
    Let $B$ be a linearly independent list and $u\in\SpanLA(C)$. To show that $\SpanLA(B)=\SpanLA(C)$, we will prove that $\SpanLA(B)\subseteq\SpanLA(C)$ and $\SpanLA(C)\subseteq\SpanLA(B)$. 

    Let $x\in\SpanLA(B)$. Then there exist $a_0,a_1,...,a_n\in\R$ such that 
    \[
        \begin{aligned}
            x=a_0u+a_1w_1+a_2w_2+...+a_nw_n
        \end{aligned}
    \]
    However, since $u\in\SpanLA(C)$, there exist $b_0,b_1,...,b_n\in\R$ such that $u=b_0v+b_1w_1+...+b_nw_n$. We have 
    \[
        \begin{aligned}
            x &= a_0u+a_1w_1+a_2w_2+...+a_nw_2\\
              &= a_0(b_0v+b_1w_1+b_2w_2+...+b_nw_2)+a_1w_1+a_2w_2+...+a_nw_2\\
              &= a_0b_0v+(a_0b_1+a_1)w_1+(a_0b_2+a_2)w_2+...+(a_0b_n+a_n)w_n
        \end{aligned}
    \]
    Thus, $x$ can be written as a linear combination of vectors in $C$, or $x\in\SpanLA(C)$. Hence, $\SpanLA(B)\subseteq\SpanLA(C)$ (1).

    Conversely, we will prove that $\SpanLA(C)\subseteq\SpanLA(B)$ by contradiction. Assume the contrary. Then, there exists $x\in\SpanLA(C)$ such that $x\not\in\SpanLA(B)$. Write $x=a_0v+a_1w_1+...+a_nw_n$, where $a_0,a_1,...,a_n\in\R$.

    Since $u\in\SpanLA(C)$, there exist $b_0,b_1,...,b_n\in\R$ such that 
    \[
        \begin{aligned}
            u=b_0v+b_1w_1+...+b_nw_n
        \end{aligned}
    \]
    However, since $B$ is a linearly independent list, $u$ cannot be written as a linear combination of $w_1,w_2,...,w_n$, or 
    \[
        \begin{aligned}
            u\not=b_1w_1+b_2v_2+...+b_nv_n
        \end{aligned}
    \]
    Thus, $b_0\not=0$, or $\frac{a_0u}{b_0}=a_0v+\frac{a_0b_1w_1}{b_0}+\frac{a_0b_2w_2}{b_0}+...+\frac{a_0b_nw_n}{b_0}$. We have 
    \[
        \begin{aligned}
            x-\frac{a_0u}{b_0} &= (a_0v+a_1w_1+a_2w_2+...+a_nw_n)-(a_0v+\frac{a_0b_1w_1}{b_0}+\frac{a_0b_2w_2}{b_0}+...+\frac{a_0b_nw_n}{b_0})\\
                               &= 0v+(a_1-\frac{a_0b_1}{b_0})w_1+(a_2-\frac{a_0b_2}{b_0})w_2+...+(a_n-\frac{a_0b_n}{b_0})w_n
        \end{aligned}
    \]  
    , which is a linear combination of the vectors $v,w_1,w_2,...,w_n$. Hence, $x-\frac{a_0u}{b_0}\in\SpanLA(C)$. But since $\SpanLA(C)$ is a subspace of $V$, vector addition and scalar multiplication is closed under $\SpanLA(C)$. Thus, for $u\in\SpanLA(C)$, $\frac{a_0u}{b_0}\in\SpanLA(C)$ and $x=(x-\frac{a_0u}{b_0})+\frac{a_0u}{b_0}\in\SpanLA(C)$, contradicting our above assumption ($x\not\in\SpanLA(C)$). Hence, $\SpanLA(C)\subseteq\SpanLA(B)$ (2).

    Therefore, from (1) and (2), it follows that $\SpanLA(B)=\SpanLA(C)$
\end{proof}