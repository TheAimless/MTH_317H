\documentclass[12pt]{article}
\usepackage{amssymb}
\usepackage{epsfig}
\usepackage{psfrag}
\usepackage{fullpage}
\usepackage{color}
\usepackage{amsfonts,epsf}
\usepackage{amsmath,amssymb,
amscd,amsbsy, bbm, amsthm, enumerate, url}
\usepackage{wasysym}


\newtheorem{thm}{Theorem}[section]
\newtheorem{question}[thm]{Question}
\newtheorem{prop}[thm]{Proposition}
\newtheorem{lem}[thm]{Lemma}
\newtheorem{DEF}[thm]{Definition}
\newtheorem{rem}[thm]{Remark}
\newenvironment{sol}{\paragraph{Solution:}\color{red}}{\hfill$\square$}
\newenvironment{grade}{\paragraph{grading:}\color{blue}}{\hfill$\square$}



\def\real{{\mathbb R}}
\def\Natural{\mathbb{N}}
\def\Z{\mathbb{Z}}
\def\integer{\mathbb{Z}}
\def\dx{\textnormal{dx}}
\def\dy{\textnormal{dy}}
\def\dz{\textnormal{dz}}
\def\dt{\textnormal{dt}}
\def\ds{\textnormal{ds}}
\def\dw{\textnormal{dw}}
\def\Re{\textnormal{Re}}
\def\Im{\textnormal{Im}}
\def\Id{\textnormal{Id}}
\def\exp{\textnormal{exp}}
\def\dim{\textnormal{dim}}
\def\interior{\textnormal{interior}}
\def\NullLA{\textnormal{null}}
\def\SpanLA{\textnormal{span}}
\def\range{\textnormal{range}}
\def\al{\alpha}
\def\del{\delta}
\def\Del{\Delta}
\def\gam{\gamma}
\def\Gam{\Gamma}
\def\Om{\Omega}
\def\ep{\varepsilon}
\def\lam{\lambda}
\def\rational{{\mathbb Q}}
\def\integer{{\mathbb Z}}
\def\grad{\nabla}


\def\B{\mathcal B}
\def\C{\mathcal C}
\def\I{\mathcal I}
\def\M{\mathcal M}
\def\N{\mathbb N}
\def\P{\mathcal P}
\def\R{\mathbb R}
\def\T{\mathcal T}
\def\Z{\mathbb Z}

\newcommand{\abs}[1]{\left| #1 \right|}
\newcommand{\ColVecTwo}[2]{\begin{pmatrix} #1\\ #2\end{pmatrix}}
\newcommand{\ColVecThree}[3]{\begin{pmatrix} #1\\ #2\\ #3\end{pmatrix}}
\newcommand{\ColVecFour}[4]{\begin{pmatrix} #1\\ #2\\#3 \\ #4\end{pmatrix}}
\newcommand{\MatrixTwoTwo}[4]{\begin{pmatrix} #1 &#2\\ #3 &#4 \end{pmatrix}}
\newcommand{\inner}[2]{\langle #1 , #2 \rangle}
\newcommand{\norm}[1]{\left\lVert#1\right\rVert}
\newcommand{\spanvect}{\textnormal{span}}
\newcommand{\union}{\cup}
\newcommand{\Union}{\bigcup}
\newcommand{\intersect}{\cap}
\newcommand{\Intersect}{\bigcap}

\DeclareMathOperator*{\Limsup}{LIMSUP}



\pagestyle{empty}
\begin{document}

	\begin{LARGE}
	\begin{center}
		
		

		
	
	\textbf{Homework 9; Due date: 11/16/2023}
	

	(MTH 317H, Honors Linear Algebra;  Fall 2023)
	\end{center}
	\end{LARGE}
	\vspace{0.15in}
	
	
	

	
\section{Commentary}

This homework assignment explores properties and computations of the determinant.

\subsection{required reading}

\begin{itemize}
	\item LADW (Linear Algebra Done Wrong) sections 3.2, 3.3, 3.4, 4.1.
\end{itemize}




\section{Questions}

\vspace{.5cm}

\begin{question}
	\normalfont
	
\begin{enumerate}[(i)]

\item Prove or give a counterexample: for $n \times n$ matrices $A$ and $B$, 
\[
\textup{det}(A + B) = \textup{det} A + \textup{det} B.
\]	


\item If $A$ is an $n \times n$ matrix, how are the determinants det$A$ and det($5A$) related? Justify your answer.
 
 \end{enumerate}
 
\end{question}



\vspace{1cm}
\begin{question}
\normalfont
\noindent Find the determinant of each of the following matrices, showing your work. You should think carefully about which approach to the determinant is best-suited for each matrix. 

\begin{enumerate}[(i)]

\item \[A= \left( \begin{array}{cccccc}
1 & 6 & 3 & 2  & -3 & 1\\
0 & 2 & 9 & - 1 & - 9 & 3.4\\
0 & 0 & 1 & -3 & -7& 2\\
0 & 0 & 0 & 3 & -1.1 & 2\\
0 & 0 & 0 & 0 & 1& 8\\
0 & 0 & 0 & 0 & 0& 1\\
\end{array} \right)\]



\item \[B= \left( \begin{array}{cccccc}
1 & 2 & 3 & 2  & -3 & 1\\
0 & 0 & 9 & - 1 & - 9 & 3.4\\
0 & 0 & 0 & 2 & -7& 2\\
0 & 0 & 0 & 0 & -1 & 2\\
0 & 0 & 0 & 0 & 0 & 1\\
1 & 0 & 2 & 0 & 0& 1\\
\end{array} \right)\]




\item \[
C= \left( \begin{array}{cccccc}
7 & 6 & 3 & 2  & -3 & 1\\
\pi & 8 & 9 & - 1 & - 9 & 3.4\\
4 & 12 & 7 & -3 & -7& 2\\
1 & 2.2 & 1.1 & 9 & -1.1 & 2\\
3 & 42 & 0 & 89 & 0& 8\\
18 & 1 & -12 & 5 & 12& 0\\
\end{array} \right)\]
\end{enumerate}

\end{question}
\vspace{1cm}

\begin{question}

\normalfont
For each of the following matrices: \\

\noindent (a) Compute the determinant of the matrix using cofactor expansion. 

\noindent (b) Compute the determinant of the matrix again, this time using row operations to reduce the matrix to a triangular matrix. \\

\begin{enumerate}[(i)]
\item
\[
A= \begin{pmatrix}
			1 & 2 & 3 & 2\\
			0 & -1 & 7 & -2\\
			0 & 0 & 2 & 1\\
			2 & 0 & 3 & 1	
		\end{pmatrix}
\]
\item
\[
B= \begin{pmatrix}
			1 & 0 & -2 & 7\\
			-3 & 1 & 1 & -4\\
			0 & 4 & -1 & 11\\
			2 & 3 & 0 & 8	
		\end{pmatrix}
\]

\end{enumerate}
\end{question}
\vspace{1cm}


\begin{question}
	\normalfont 
	\begin{enumerate}[(i)]
	
	\item A square matrix with entries in the real numbers $Q$ is called \emph{orthogonal} if $Q^TQ = I$. Here $Q^T$ denotes the transpose of $Q$. Prove that if $Q$ is an orthogonal matrix, then 
	\[
	\textup{det}(Q) = \pm 1. 
	\]
	
	\item A square matrix is called \emph{nilpotent} if $A^k =0$ for some positive integer $k$. Prove that for a nilpotent matrix $A$, det$(A)=0$. 
	
	
	
	
	\end{enumerate}
	\end{question}



\vspace{1cm}

\begin{question}
	\normalfont We say that a matrix $A$ is \emph{similar} to a matrix $B$ if there exists a matrix $Q$ such that 
	\[
	A = Q^{-1}BQ.
	\]
	\vspace{.1cm}
	\begin{enumerate}[(i)]
	\item Prove that if matrices $A$ and $B$ are similar, both $A$ and $B$ must be square matrices, and of the same size. 
	\vspace{.1cm}
	
\item	Prove that if $A$ and $B$ are similar matrices, then det($A$) = det($B$). 
	\end{enumerate}
		
	\end{question}




\vspace{1cm}

\begin{question}	\normalfont
	Recall that for a $2\times 2$ matrix 
	\[
	\begin{pmatrix}
			a & b \\
			c & d
		\end{pmatrix}
\]
the determinant is given by
\[
\textup{det} \begin{pmatrix}
			a & b \\
			c & d
		\end{pmatrix} = ad - bc. 
\]
In this problem you will verify the following properties of the determinant in the $2\times 2$ case, using this definition. 

\begin{enumerate}[(i)]

\item \emph{Anti-symmetry}: Show that 
\[
\textup{det}\begin{pmatrix}
			b & a \\
			d & c
		\end{pmatrix} = - \textup{det}\begin{pmatrix}
			a & b \\
			c & d
		\end{pmatrix} 
\]
\item \emph{Linearity}: Show that 
\[
\textup{det}\begin{pmatrix}
			\alpha a_1 + \beta a_2 & b \\
			\alpha c_1 + \beta c_2 & d
		\end{pmatrix} = \alpha \hspace{.1cm} \textup{det}\begin{pmatrix}
			a_1 & b \\
			c_1 & d
		\end{pmatrix} + \beta  \hspace{.1cm}  \textup{det}\begin{pmatrix}
			a_2 & b \\
			c_2 & d
		\end{pmatrix}  
\]

\item \emph{Normalization}: Show that 
\[
\textup{det}\begin{pmatrix}
			1 & 0 \\
			0 & 1
		\end{pmatrix} = 1.
\]

\end{enumerate}

\end{question}
	


\vspace{1cm}

\begin{question}
\normalfont
Let $T:\real^3\to\real^3$ be the linear function defined as
		\begin{align*}
			T(\ColVecThree{x_1}{x_2}{x_3}) = \ColVecThree{x_1-x_2}{-x_1+2x_2-2x_3}{2x_2 + x_3}.
\end{align*}

\begin{enumerate}[(i)]

\item Let $A$ denote the matrix of the linear transformation $T$ with respect to the standard basis of $\real^3$. Find the matrix $A$, and find det($A$). \\

\item Consider the basis 
\[
\ColVecThree{1}{1}{1}, \ColVecThree{0}{1}{1},
 \ColVecThree{1}{0}{1}
\]
of $\real^3$.  Let $B$ denote the matrix of the transformation $T$ with respect to this basis for both the domain and the target. Find the matrix $B$, and find det($B$). \\


\item Considering your answers to parts (i) and (ii) above, make a conjecture about how the determinant of matrix representing a linear transformation depends on the bases of the domain and target. You do not need to prove your conjecture. 
\end{enumerate}





\end{question}




\vspace{1cm}


	
	\vspace{.5cm}
	
\begin{question}
\normalfont
Let $S$ be an invertible $l \times l$ matrix and $D$ be an $l \times l$ matrix. Use mathematical induction to prove 
$$
\left(S D S^{-1} \right)^n = S D^n S^{-1}, \text{ for all } n \in \mathbb{N}.
$$ 
Your inductive proof must include the following steps:
	\begin{enumerate}[(i)]
		\item identify and state what is the conditional statement, $P(n)$, for $n\in \Natural$.
		\item state which natural number you will use as the base case, and prove that the conditional statement is true for the base case.
		\item for a fixed $k\in\Natural$, state the inductive assumption.
		\item prove the inductive step.  That is to say, prove that $P(k+1)$ is also true.
		\item conclude.
	\end{enumerate}
\end{question}










%%%%%%%%%%%%%%%%%%%%%%%%%%%%%%%%%%%%%%%%%%%%%%%%%%%%%%%%%%%%%%%%%%%%%%%%%%%%%%%%%%%%%%%%%%%

%%%%%%%%%%%%%%%%%%%%%%%%%%%%%%%%%%%%%%%%%%%%%%%%%%%%%%%%%%%%%%%%%%%%%%%%%%%%%%%%%%%%%%%%%%%
\end{document}





















