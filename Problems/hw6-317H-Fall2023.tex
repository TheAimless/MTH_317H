\documentclass[12pt]{article}
\usepackage{amssymb}
\usepackage{epsfig}
\usepackage{psfrag}
\usepackage{fullpage}
\usepackage{color}
\usepackage{amsfonts,epsf}
\usepackage{amsmath,amssymb,
amscd,amsbsy, bbm, amsthm, enumerate, url}
\usepackage{wasysym}


\newtheorem{thm}{Theorem}[section]
\newtheorem{question}[thm]{Question}
\newtheorem{prop}[thm]{Proposition}
\newtheorem{lem}[thm]{Lemma}
\newtheorem{DEF}[thm]{Definition}
\newtheorem{rem}[thm]{Remark}
\newenvironment{sol}{\paragraph{Solution:}\color{red}}{\hfill$\square$}
\newenvironment{grade}{\paragraph{grading:}\color{blue}}{\hfill$\square$}



\def\real{{\mathbb R}}
\def\Natural{\mathbb{N}}
\def\Z{\mathbb{Z}}
\def\integer{\mathbb{Z}}
\def\dx{\textnormal{dx}}
\def\dy{\textnormal{dy}}
\def\dz{\textnormal{dz}}
\def\dt{\textnormal{dt}}
\def\ds{\textnormal{ds}}
\def\dw{\textnormal{dw}}
\def\Re{\textnormal{Re}}
\def\Im{\textnormal{Im}}
\def\Id{\textnormal{Id}}
\def\exp{\textnormal{exp}}
\def\dim{\textnormal{dim}}
\def\interior{\textnormal{interior}}
\def\NullLA{\textnormal{null}}
\def\SpanLA{\textnormal{span}}
\def\range{\textnormal{range}}
\def\al{\alpha}
\def\del{\delta}
\def\Del{\Delta}
\def\gam{\gamma}
\def\Gam{\Gamma}
\def\Om{\Omega}
\def\ep{\varepsilon}
\def\lam{\lambda}
\def\rational{{\mathbb Q}}
\def\integer{{\mathbb Z}}
\def\grad{\nabla}


\def\B{\mathcal B}
\def\C{\mathcal C}
\def\I{\mathcal I}
\def\M{\mathcal M}
\def\N{\mathbb N}
\def\P{\mathcal P}
\def\R{\mathbb R}
\def\T{\mathcal T}
\def\Z{\mathbb Z}

\newcommand{\abs}[1]{\left| #1 \right|}
\newcommand{\ColVecTwo}[2]{\begin{pmatrix} #1\\ #2\end{pmatrix}}
\newcommand{\ColVecThree}[3]{\begin{pmatrix} #1\\ #2\\ #3\end{pmatrix}}
\newcommand{\ColVecFour}[4]{\begin{pmatrix} #1\\ #2\\#3 \\ #4\end{pmatrix}}
\newcommand{\MatrixTwoTwo}[4]{\begin{pmatrix} #1 &#2\\ #3 &#4 \end{pmatrix}}
\newcommand{\inner}[2]{\langle #1 , #2 \rangle}
\newcommand{\norm}[1]{\left\lVert#1\right\rVert}
\newcommand{\spanvect}{\textnormal{span}}
\newcommand{\union}{\cup}
\newcommand{\Union}{\bigcup}
\newcommand{\intersect}{\cap}
\newcommand{\Intersect}{\bigcap}

\DeclareMathOperator*{\Limsup}{LIMSUP}



\pagestyle{empty}
\begin{document}

	\begin{LARGE}
	\begin{center}
		
		

		
	
	\textbf{Homework 6; Due 10/19/2023}
	

	(MTH 317H, Honors Linear Algebra;  Fall 2023)
	\end{center}
	\end{LARGE}
	\vspace{0.15in}
	
	
	


	
\section{Commentary}

In this homework you will continue working with linear functions, span, independence, and basis.



\subsection{required reading}

\begin{itemize}
	\item LADR sections 3C and 3D
	\item LADW (Linear Algebra Done Wrong) sections 2.1, 2.2.
\end{itemize}


\section{Definitions}\label{defs}
Here is some notation we will use in this homework regarding the bookkeeping for coordinates in a basis.  This is a slightly more precise variation on the definition given in LADR 3.62, as we specifically note the basis in the notation.

\begin{DEF}[variation on LADR 3.62]

Let $\B = [v_1, \dots v_m]$ be a basis for a vector space, $V$.  We say that $x\in V$ has the coordinates $a_1,\dots,a_m$ in the basis, $\B$, if 
\begin{align*}
	x = a_1 v_1 + \cdots + a_m v_m.
\end{align*}
We use the following shorthand notation for this as
\begin{align}\label{eq:Coordinates}
	x=\ColVecFour{a_1}{a_2}{\vdots}{a_m}_{\B}\ \ \text{or}\ \
	x=\ColVecFour{a_1}{a_2}{\vdots}{a_m}_{[v_i]} \ \ \ \text{is defined as}\ \ \
	x = a_1 v_1 + \cdots + a_m v_m.
\end{align}

\end{DEF}


Note, when working on $\real^m$ with the \emph{canonical} basis, $[e_1,\dots, e_m]$, we will often suppress the specific indication of the basis.  That is to say, it is implied that when no basis is noted, it means the canonical one is used, 
\begin{align*}
	x=\ColVecFour{a_1}{a_2}{\vdots}{a_m} = \ColVecFour{a_1}{a_2}{\vdots}{a_m}_{[e_i]}.
\end{align*}





\section{Questions}


\begin{question}
	\normalfont
	
	Assume that $U$ and $V$ are both subspaces of $\real^7$ and 
	\begin{align*}
		\dim(U)=\dim(V)=4.
	\end{align*} 
	\begin{align}\label{eq:UIntersectVStatement}
		\text{Prove that}\  U\intersect V \not= \{0\}. 
	\end{align}
	 
	
	
	
	You may not use any results using the sum of vector spaces.  Instead, you should show that you can prove this by more basic methods.  Specifically, please include the following substeps (or sub-results) in your proof.
	
	\begin{enumerate}
		\item Let $[u_1, u_2, u_3, u_4]$ be a basis for $U$ and $[v_1, v_2, v_3, v_4]$ be a basis for $V$.  Prove that if $U\intersect V = \{0\}$, then \hspace{.1cm} $u_1,\dots, u_4, v_1,\dots, v_4$ \hspace{.1cm} is a linearly independent list in $\real^7$.   (Hint: Suppose the list $u_1, \dots, u_4, v_1,\dots, v_4$ is a linearly dependent list and show that you get a contradiction with the fact that $U\intersect V = \{0\}$.)
		
		\item Give a proof of the original claim, (\ref{eq:UIntersectVStatement}), by way of contradiction.  Use your result from step 1 as part of the proof. 
	\end{enumerate}
\end{question}


\vspace{.5cm}



\begin{question}\label{que:LemmaGenericSols}
	\normalfont
	
	In this question, provide a proof of the following important lemma. \\
	
	\textbf{Lemma:} Assume that $L:V\to W$ is linear, that $y\in\range(L)$. If $x_0\in V$ with $L(x_0)=y$, then 
	\begin{align*}
		\forall\ x\in\{ v\in V\ :\ L(v)=y \},\ \exists\ z\in \NullLA(L),\ \ \text{with}\ x=x_0+z.
	\end{align*}
	
	(For context, what this lemma says in plain English is that as soon as you have found one solution of $L(v)=y$, called $x_0$, then all other solutions must be of form $x_0+z$, where $z$ is in the null space of $L$.)
	
\end{question}


\vspace{.5cm}


\begin{question}
	\normalfont
	
	
	
	
	Let $L:\P_5\to P_4$ be the linear function given by 
	\begin{align*}
		L(p) = 7p'' + 3p'.
	\end{align*}
	\begin{enumerate}[(i)]
		\item Give a list of polynomials, $\ell = \{n_1,\dots, n_m\}$ so that $\ell$ is a basis for $\NullLA(L)$.  Prove that your choice is correct.
		
		
		\item Prove that given any polynomial, $q\in \P_4$, there exists $p\in\P_5$ so that $L(p)=q$.  (Hint, think about surjectivity. You should be able to do dimension counting and use LADR result 3.22.  I would suggest \emph{not} to prove the existence of $q$ from scratch.)
		
		\item For a given $q$ as in the previous step, is the choice of $p$ unique?  Explain your answer.
		
	\end{enumerate}
\end{question}


\vspace{.8cm}


\begin{question}\label{que:CoordinateFunction}
	\normalfont
	
		
	Let $V$ be a vector space, $k$ a fixed positive integer, and $\ell = v_1,\dots, v_k$ a fixed list of vectors in $V$. Let $L:\real^k\to V$ be the linear function defined as
	\begin{align}\label{eq:LCoordinateMap}
		L(\ColVecFour{x_1}{x_2}{\vdots}{x_k}) = \sum_{i=1}^k x_iv_i.
	\end{align}
	
	\begin{enumerate}[(i)]
		\item Prove that if $L$ is surjective, then span($v_1,\dots, v_k) = V$.
		
		\item Prove that if $L$ is injective, then $v_1,\dots, v_k$ is a linearly independent list of vectors.
		
		\item Assume that you know that $L$ is surjective.  Let $\dim(V)=m$. State, in terms of $k$, the possible values of $m$.  Prove your answer.
		
		
	\end{enumerate}
\end{question}

\vspace{.5cm}







\noindent The next collection of questions will all refer to the same matrix and function. Define the $4\times3$ matrix $A$, as
	\begin{align}\label{eq:MatrixA}
		A = 
		 \begin{pmatrix}
		  1 & 0 & 1 \\
		  2 & 1 & 1 \\
		  2 & 0 & 2 \\
		  5 & 3 & 2 
		 \end{pmatrix},
	\end{align}
	and define the function, $L:\real^3\to\real^4$ as
	\begin{align}\label{eq:FunctionL}
		\text{whenever}\ &x = \ColVecThree{x_1}{x_2}{x_3} = x_1 e_1 + x_2 e_2 + x_3 e_3\ \ \text{\textbf{in the canonical basis}},\nonumber \\
		&L(x) = Ax,\ \ \text{defined via matrix multiplication}.
	\end{align}
	You should have read about matrix-vector multiplication in LADR 3C, but to be explicit, this is what we mean: for a matrix $B$, the matrix multiplication $Bx$ is given as:
	\begin{align*}
		\begin{pmatrix}
	     	  b_{1,1} & b_{1,2} & b_{1,3} \\
	     	  b_{2,1} & b_{2,2} & b_{2,3} \\
	     	  b_{3,1} & b_{3,2} & b_{3,3} \\
	     	  b_{4,1} & b_{4,2} & b_{4,3} \\
		\end{pmatrix}
		\begin{pmatrix}
			x_1\\
			x_2\\
			x_3
		\end{pmatrix}
		= 
		\ColVecFour{b_{1,1}x_1 + b_{1,2}x_2 + b_{1,3}x_3}
		{b_{2,1}x_1 + b_{2,2}x_2 + b_{2,3}x_3}
		{b_{3,1}x_1 + b_{3,2}x_2 + b_{3,3}x_3}
		{b_{4,1}x_1 + b_{4,2}x_2 + b_{4,3}x_3}
	\end{align*}
	
	
	\vspace{.5cm}
	
\begin{question}\label{que:RangeAndNullMatrixL}
	\normalfont
	
	Consider the function $L$ defined in (\ref{eq:FunctionL}).
	
	\begin{enumerate}[(i)]
		\item Prove that $[L(e_1), L(e_2)]$ is a basis for $\range(L)$.
		
		\item Give a basis for $\NullLA(L)$ and prove why your choice is correct.
		
		\item Let $w\in\real^4$ be given by
		\begin{align*}
			w = \ColVecFour{0}{0}{1}{0}.
		\end{align*}
		Is it possible to find some $v\in\real^3$, so that
		\begin{align*}
			L(v)=w?
		\end{align*}
		Prove your answer. (Hint: is $w\in\range(L)$?)
	\end{enumerate}
	
	
	
\end{question}
	
	
	\vspace{1cm}
	


\begin{question}
	\normalfont
	Let $L$ be defined in (\ref{eq:FunctionL}).
	
	\begin{enumerate}[(i)]
		
		
		\item Let $y\in\real^4$ be the vector, in the canonical basis, given by
		\begin{align*}
			y=\ColVecFour{4}{6}{8}{14}.
		\end{align*}
		Find scalars, $a_1$, $a_2$, so that
		\begin{align*}
			y= a_1L(e_1) + a_2 L(e_2).
		\end{align*}
		 
		\item For the same $y$ as in part (i), find an $x_0$ so that $L(x_0)=y$.
		\item For the same $y$ as in the previous parts, find all possible solutions, $x$, to the equation $L(x)=y$, in terms of a scalar, $\lam$, and some vector, $z_0$, so that $x=x_0+\lam z_0$.  In this context, state what is $z_0$.  (You should use the lemma in Question \ref{que:LemmaGenericSols} plus your answer in Question \ref{que:RangeAndNullMatrixL}.)  
	
	\end{enumerate}
\end{question}




\begin{question}
	\normalfont
	
	Here are different choices of basis for $\real^3$ and $\real^4$:
	\begin{align*}
		v_1 = \ColVecThree{1}{0}{0}\ \ \ 
		v_2 = \ColVecThree{0}{1}{0}\ \ \ 
		v_3 = \ColVecThree{1}{-1}{-1}\ \ \
		\text{is a basis of}\ \ \real^3. 
	\end{align*}
	and
	\begin{align*}
		w_1 = \ColVecFour{2}{4}{4}{10}\ \ \ 
		w_2 = \ColVecFour{0}{2}{0}{6}\ \ \ 
		w_3 = \ColVecFour{1}{0}{0}{0}\ \ \
		w_4 = \ColVecFour{0}{1}{0}{0}\ \ \
		\text{is a basis of}\ \ \real^4. 
	\end{align*}
	
	
	Write down the matrix, $C$, given by 
	\begin{align*}
			C = 
			 \begin{pmatrix}
			  c_{1,1} & c_{1,2} & c_{1,3} \\
			  c_{2,1} & c_{2,2} & c_{2,3} \\
			  c_{3,1} & c_{3,2} & c_{3,3} \\
			  c_{4,1} & c_{4,2} & c_{4,3} 
			 \end{pmatrix},
	\end{align*}
	so that when using $[v_i]$ as a basis for $\real^3$ and $[w_i]$ as a basis for $\real^4$, we have that $C$ is the matrix for $L$ in those specific bases.  That is to say, 
	\begin{align*}
		\text{if}\ \ L(\ColVecThree{y_1}{y_2}{y_3}_{[v_i]}) = \ColVecFour{z_1}{z_2}{z_3}{z_4}_{[w_i]},\ \ \ 
	\text{then}\ \ \ 
		\ColVecFour{z_1}{z_2}{z_3}{z_4}_{[w_i]} = 
	 \begin{pmatrix}
	  c_{1,1} & c_{1,2} & c_{1,3} \\
	  c_{2,1} & c_{2,2} & c_{2,3} \\
	  c_{3,1} & c_{3,2} & c_{3,3} \\
	  c_{4,1} & c_{4,2} & c_{4,3} 
	 \end{pmatrix}
	 \ColVecThree{y_1}{y_2}{y_3}_{[v_i]}
	\end{align*}
	Here are some steps that should be useful:
	\begin{enumerate}
		\item You know that $L$ is uniquely determined on any basis.  You should have already computed $L(e_1)$, $L(e_2)$, and $L(e_3)$ (or if you haven't yet, it is quick).  Now, write $v_1$, $v_2$, $v_3$ in terms of $e_1$, $e_2$, and $e_3$.  This allows you to compute $L(v_1)$, $L(v_2)$, $L(v_3)$.
		
		\item For a generic $y\in\real^3$ in the $[v_i]$ basis, it will have coordinates
		\begin{align*}
			y=\ColVecThree{y_1}{y_2}{y_3}_{[v_i]}\ \ \ \text{hence}\ \ \ 
			y=y_1 v_1 + y_2 v_2 + y_3 v_3.
		\end{align*}
		This means that your previous step allows you to easily compute $L(y)$, as you now know $L(v_i)$.
		
		\item Now, you need to find a choice (it is unique) of $z_1, z_2, z_3, z_4\in\real$ so that
		\begin{align*}
			L(y) = z_1 w_1 + z_2 w_2 + z_3 w_3 + z_4 w_4,\ \ \ \text{i.e.}\ \ \
			L(y) = \ColVecFour{z_1}{z_2}{z_3}{z_4}_{[w_i]}.
		\end{align*}
		
		\item Finally, choose the $c_{i,j}$ so that 
		\begin{align*}
			\ColVecFour{z_1}{z_2}{z_3}{z_4}_{[w_i]} = C\ColVecThree{y_1}{y_2}{y_3}_{[v_i]}\ \ \text{as matrix multiplication}.
		\end{align*}
		(Hint, most entries should be zero.)
	\end{enumerate}
	
	
	
\end{question}








%%%%%%%%%%%%%%%%%%%%%%%%%%%%%%%%%%%%%%%%%%%%%%%%%%%%%%%%%%%%%%%%%%%%%%%%%%%%%%%%%%%%%%%%%%%

%%%%%%%%%%%%%%%%%%%%%%%%%%%%%%%%%%%%%%%%%%%%%%%%%%%%%%%%%%%%%%%%%%%%%%%%%%%%%%%%%%%%%%%%%%%
\end{document}





















