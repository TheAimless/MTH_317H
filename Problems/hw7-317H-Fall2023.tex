\documentclass[12pt]{article}
\usepackage{amssymb}
\usepackage{epsfig}
\usepackage{psfrag}
\usepackage{fullpage}
\usepackage{color}
\usepackage{amsfonts,epsf}
\usepackage{amsmath,amssymb,
amscd,amsbsy, bbm, amsthm, enumerate, url}
\usepackage{wasysym}


\newtheorem{thm}{Theorem}[section]
\newtheorem{question}[thm]{Question}
\newtheorem{prop}[thm]{Proposition}
\newtheorem{lem}[thm]{Lemma}
\newtheorem{DEF}[thm]{Definition}
\newtheorem{rem}[thm]{Remark}
\newenvironment{sol}{\paragraph{Solution:}\color{red}}{\hfill$\square$}
\newenvironment{grade}{\paragraph{grading:}\color{blue}}{\hfill$\square$}



\def\real{{\mathbb R}}
\def\Natural{\mathbb{N}}
\def\Z{\mathbb{Z}}
\def\integer{\mathbb{Z}}
\def\dx{\textnormal{dx}}
\def\dy{\textnormal{dy}}
\def\dz{\textnormal{dz}}
\def\dt{\textnormal{dt}}
\def\ds{\textnormal{ds}}
\def\dw{\textnormal{dw}}
\def\Re{\textnormal{Re}}
\def\Im{\textnormal{Im}}
\def\Id{\textnormal{Id}}
\def\exp{\textnormal{exp}}
\def\dim{\textnormal{dim}}
\def\interior{\textnormal{interior}}
\def\NullLA{\textnormal{null}}
\def\SpanLA{\textnormal{span}}
\def\range{\textnormal{range}}
\def\al{\alpha}
\def\del{\delta}
\def\Del{\Delta}
\def\gam{\gamma}
\def\Gam{\Gamma}
\def\Om{\Omega}
\def\ep{\varepsilon}
\def\lam{\lambda}
\def\rational{{\mathbb Q}}
\def\integer{{\mathbb Z}}
\def\grad{\nabla}


\def\B{\mathcal B}
\def\C{\mathcal C}
\def\I{\mathcal I}
\def\M{\mathcal M}
\def\N{\mathbb N}
\def\P{\mathcal P}
\def\R{\mathbb R}
\def\T{\mathcal T}
\def\Z{\mathbb Z}

\newcommand{\abs}[1]{\left| #1 \right|}
\newcommand{\ColVecTwo}[2]{\begin{pmatrix} #1\\ #2\end{pmatrix}}
\newcommand{\ColVecThree}[3]{\begin{pmatrix} #1\\ #2\\ #3\end{pmatrix}}
\newcommand{\ColVecFour}[4]{\begin{pmatrix} #1\\ #2\\#3 \\ #4\end{pmatrix}}
\newcommand{\MatrixTwoTwo}[4]{\begin{pmatrix} #1 &#2\\ #3 &#4 \end{pmatrix}}
\newcommand{\inner}[2]{\langle #1 , #2 \rangle}
\newcommand{\norm}[1]{\left\lVert#1\right\rVert}
\newcommand{\spanvect}{\textnormal{span}}
\newcommand{\union}{\cup}
\newcommand{\Union}{\bigcup}
\newcommand{\intersect}{\cap}
\newcommand{\Intersect}{\bigcap}

\DeclareMathOperator*{\Limsup}{LIMSUP}



\pagestyle{empty}
\begin{document}

	\begin{LARGE}
	\begin{center}
		
		

		
	
	\textbf{Homework 7; Due Thursday 11/2}
	

	(MTH 317H, Honors Linear Algebra;  Fall 2023)
	\end{center}
	\end{LARGE}
	\vspace{0.15in}
	
	
	

\section{Commentary}

This assignment covers invertible linear transformations, isomorphisms, and matrices associated to a transformation. We will also start solving systems of linear equations using matrices. 

\subsection{required reading}

\begin{itemize}
	\item LADW (Linear Algebra Done Wrong) sections 2.3, 2.4, and 2.6.
\end{itemize}




\section{Questions}


\begin{question}
	\normalfont
	For each of the linear maps below, find the matrix of the linear map with respect to the indicated bases. \\
	
	
	\noindent (a) Find the matrix of the linear map $T: \real^4 \to \real^3$ defined by
	\[
	T(\ColVecFour{x_1}{x_2}{x_3}{x_4}) = \ColVecThree{x_1+x_2+x_3+x_4}{x_2-x_4}{x_1+3x_2+6x_4},
	\]
with respect to the standard bases of $\real^4$ and $\real^3$. \\

\vspace{.2cm}
\noindent (b) Find the matrix of the linear map $S: \P_3 \to \P_3$ defined by \[(Sp)(x) = 2p(x) + 3p'(x) - 4 p^{''}(x),\] with respect to the standard basis of $\P_3$. \\

\vspace{.2cm}
	
	\noindent	(c) Consider the linear map $H:\mathbb{R}^3 \rightarrow \mathbb{R}^3$ defined by $$H\left( \begin{array}{c}
v_1 \\
v_2 \\
v_3 \\
\end{array} \right)
= \left( \begin{array}{c}
 v_3\\
 v_2 - v_1\\
 2v_1 \\
 \end{array} \right)$$
\noindent Give the matrix for $H$ with respect to the following basis $\mathcal{C}$ for both the domain and the target: 

\[
\mathcal{C}= \Bigg\{\left( \begin{array}{c}
0 \\
1 \\
1\\
 \end{array}  \right), \left( \begin{array}{c}
1 \\
1 \\
0\\
 \end{array} \right), \left( \begin{array}{c}
0 \\
-1 \\
1\
 \end{array} \right)\Bigg\}.
 \]

	\vspace{.4cm}	
		
\noindent (d) Consider the linear map $J:\P_2 \rightarrow \mathbb{R}^3$ defined by 
\[
J(a_0+a_1x+a_2x^2)= \left( \begin{array}{c}
a_1 + a_0\\
a_2-2a_1 \\
a_0 \\
 \end{array} \right)
 \]
\noindent Write the matrix for $J$ with respect to the standard bases $\{1,x,x^2\}$ for $\P_2$, and $\{\textbf{e}_1,\textbf{e}_2,\textbf{e}_3\}$ for $\mathbb{R}^3$.\\ 
 \vspace{0.5cm}

\noindent (e) Given the linear map $J$ in part (d), find the matrix for $J$ with respect to the bases $\mathcal{A} = \{2,1+x, x^2\}$ for $\P_2$, and 
\[
\mathcal{B} = \Bigg\{\left( \begin{array}{c}
1 \\
0 \\
1\
 \end{array}  \right), \left( \begin{array}{c}
1 \\
-1 \\
0\\
 \end{array} \right), \left( \begin{array}{c}
0 \\
1 \\
0\\
 \end{array} \right)\Bigg\} 
 \]
 for $\mathbb{R}^3.$

 
 
\end{question}


\vspace{.5cm}

\begin{question}

\normalfont
(a) Consider the linear map $D: \P_3 \to \P_2$ given by differentiation. In other words $(Dp)(x) = p'(x)$. Find a basis of $\P_3$ and a basis of $\P_2$ such that the matrix of $D$ with respect to these bases is 
\[
\left( \begin{array}{rrrr}
1 & 0 & 0 & 0 \\
0 & 1 & 0 & 0 \\
0 & 0 & 1 & 0 \\
\end{array}\right)
\]


\vspace{.4cm}

\noindent (b) Suppose $V$ and $W$ are finite-dimensional vector spaces, and $T$ is a linear map $T: V \to W$. Prove that there exists a basis of $V$ and a basis of $W$ such that all entries of the matrix of $T$ with respect to these bases are 0 except that the entry in row $j$, column $j$ is 1 for all $1\leq j \leq \dim (\textup{range} T$). 

\end{question}






\vspace{.5cm}

\begin{question}
	\normalfont
	
	Suppose $T \in \mathcal{L}(U,V)$ and $S \in \mathcal{L}(V,W)$ are both invertible linear maps. Prove that $ST \in \mathcal{L}(U,W)$ is invertible and that $(ST)^{-1} = T^{-1}S^{-1}$. 
	
	
	\end{question}







\vspace{.5cm}

\begin{question}
\normalfont
\noindent Without citing LADR result 3.59, show explicitly that  $\P_3$ is isomorphic to the vector space of $2\times2$ matrices with entries in the real numbers, $M_{2\times 2}(\real)$. In other words, define linear maps (and prove that they are linear)
\[
T: \P_3 \to M_{2\times2}(\real)
\]
and 
\[
S: M_{2\times2}(\real) \to \P_3
\]
such that $ST$ is the identity map on $\P_3$ and $TS$ is the identity map on $M_{2\times2}(\real)$.  \\ 


\end{question}




\vspace{.5cm}

\begin{question}	\normalfont
	
For each of the following systems of equations: \\

\noindent (i) Express the system in matrix form $Ax =b$. \\

\noindent (ii) 	Use row reduction to reach {\bf reduced row echelon} form,
and then  determine all solutions to the
system.  \\ \\	
	
\noindent (a) 	
\[
\begin{array}{r}
x_1 + 2x_2 - x_3 = -1\\ 
2x_1 + 2x_2 + x_3 = 1 \\
3x_1 + 5x_2 - 2x_3 = -1 \\
\end{array}
\]

\vspace{.2cm}

\noindent (b)
\[
\begin{array}{r}
x_1 - 4x_2 - x_3 + x_4 = 3 \\
2x_1 - 8 x_2 + x_3 -4x_4 = 9 \\
-2x_1 + 8 x_2 + 2x_3 -2x_4 = -6 \\
\end{array}
\]	

\vspace{.2cm}

\noindent (c) 
\[
\begin{array}{r}
x_1 -2x_2 -x_3 = 1\\
2x_1 - 3x_2 + x_3 = 6 \\
-x_1 -5x_3 = 4 \\
\end{array}
\]


\end{question}
	
	
	\vspace{1cm}
	





\vspace{.5cm}

\begin{question}
	\normalfont
	
	Consider the following system of equations in variables $x_1,x_2,x_3,x_4$:
	\[
	\begin{array}{r}
	x_1 + 3x_2 + 2x_3 = a \\
	x_3 + 5 x_4 = b \\
	3x_1 + 9x_2 + 7x_3 + 5x_4 = c \\
	\end{array}
	\]
	
\noindent (a) Describe all triples $(a,b,c)$ for which this system will have a solution \\


\noindent (b) Find all solutions to the equation when  $(a,b,c)=(1,1,4)$. 
	
	
\end{question}


\vspace{.5cm}





\begin{question}
	\normalfont
	
	Use mathematical induction to prove the following statement:
	\begin{align*}
		\textnormal{for all}\ n\in\Natural,\ \ 
		\MatrixTwoTwo{2}{0}{-1}{1}^n
		= \MatrixTwoTwo{2^n}{0}{1-2^n}{1}.
	\end{align*}
	Note, in this class we are using the convention that $\Natural = \{1,2,3,\dots\}$, i.e. $0\not\in\Natural$.
	Also, we are using the notation for the power of a matrix that if $A$ is a square matrix, then by $A^n$, we mean $A$ multiplied by itself $n$ times; i.e.
	\begin{align*}
		A^2 = AA,\ \ A^3 = AAA,\ \ A^n = A\cdots A\ \ \textnormal{with $n$ terms in the product}.
	\end{align*}
	
	Your inductive proof must include the following steps:
	\begin{enumerate}[(i)]
		\item identify and state what is the conditional statement, $P(n)$, for $n\in \Natural$.
		\item state which natural number you will use as the base case, and prove that the conditional statement is true for the base case.
		\item for a fixed $k\in\Natural$, state the inductive assumption.
		\item prove the inductive step.  That is to say, prove that $P(k+1)$ is also true.
		\item conclude.
	\end{enumerate}
\end{question}








%%%%%%%%%%%%%%%%%%%%%%%%%%%%%%%%%%%%%%%%%%%%%%%%%%%%%%%%%%%%%%%%%%%%%%%%%%%%%%%%%%%%%%%%%%%

%%%%%%%%%%%%%%%%%%%%%%%%%%%%%%%%%%%%%%%%%%%%%%%%%%%%%%%%%%%%%%%%%%%%%%%%%%%%%%%%%%%%%%%%%%%
\end{document}





















