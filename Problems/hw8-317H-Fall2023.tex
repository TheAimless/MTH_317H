\documentclass[12pt]{article}
\usepackage{amssymb}
\usepackage{epsfig}
\usepackage{psfrag}
\usepackage{fullpage}
\usepackage{color}
\usepackage{amsfonts,epsf}
\usepackage{amsmath,amssymb,
amscd,amsbsy, bbm, amsthm, enumerate, url}
\usepackage{wasysym}


\newtheorem{thm}{Theorem}[section]
\newtheorem{question}[thm]{Question}
\newtheorem{prop}[thm]{Proposition}
\newtheorem{lem}[thm]{Lemma}
\newtheorem{DEF}[thm]{Definition}
\newtheorem{rem}[thm]{Remark}
\newenvironment{sol}{\paragraph{Solution:}\color{red}}{\hfill$\square$}
\newenvironment{grade}{\paragraph{grading:}\color{blue}}{\hfill$\square$}



\def\real{{\mathbb R}}
\def\Natural{\mathbb{N}}
\def\Z{\mathbb{Z}}
\def\integer{\mathbb{Z}}
\def\dx{\textnormal{dx}}
\def\dy{\textnormal{dy}}
\def\dz{\textnormal{dz}}
\def\dt{\textnormal{dt}}
\def\ds{\textnormal{ds}}
\def\dw{\textnormal{dw}}
\def\Re{\textnormal{Re}}
\def\Im{\textnormal{Im}}
\def\Id{\textnormal{Id}}
\def\exp{\textnormal{exp}}
\def\dim{\textnormal{dim}}
\def\interior{\textnormal{interior}}
\def\NullLA{\textnormal{null}}
\def\SpanLA{\textnormal{span}}
\def\range{\textnormal{range}}
\def\al{\alpha}
\def\del{\delta}
\def\Del{\Delta}
\def\gam{\gamma}
\def\Gam{\Gamma}
\def\Om{\Omega}
\def\ep{\varepsilon}
\def\lam{\lambda}
\def\rational{{\mathbb Q}}
\def\integer{{\mathbb Z}}
\def\grad{\nabla}


\def\B{\mathcal B}
\def\C{\mathcal C}
\def\I{\mathcal I}
\def\M{\mathcal M}
\def\N{\mathbb N}
\def\P{\mathcal P}
\def\R{\mathbb R}
\def\T{\mathcal T}
\def\Z{\mathbb Z}

\newcommand{\abs}[1]{\left| #1 \right|}
\newcommand{\ColVecTwo}[2]{\begin{pmatrix} #1\\ #2\end{pmatrix}}
\newcommand{\ColVecThree}[3]{\begin{pmatrix} #1\\ #2\\ #3\end{pmatrix}}
\newcommand{\ColVecFour}[4]{\begin{pmatrix} #1\\ #2\\#3 \\ #4\end{pmatrix}}
\newcommand{\MatrixTwoTwo}[4]{\begin{pmatrix} #1 &#2\\ #3 &#4 \end{pmatrix}}
\newcommand{\MatrixThreeThree}[9]{\begin{pmatrix} #1 &#2 &#3\\ #4 &#5 &#6\\ #7 &#8 &#9 \end{pmatrix}}
\newcommand{\inner}[2]{\langle #1 , #2 \rangle}
\newcommand{\norm}[1]{\left\lVert#1\right\rVert}
\newcommand{\spanvect}{\textnormal{span}}
\newcommand{\union}{\cup}
\newcommand{\Union}{\bigcup}
\newcommand{\intersect}{\cap}
\newcommand{\Intersect}{\bigcap}

\DeclareMathOperator*{\Limsup}{LIMSUP}



\pagestyle{empty}
\begin{document}

	\begin{LARGE}
	\begin{center}
		
		
		
	
	\textbf{Homework 8; due Thursday, 11/09/2023}
	

	(MTH 317H, Honors Linear Algebra;  Fall 2023)
	\end{center}
	\end{LARGE}
	\vspace{0.15in}
	
	



	
\section{Commentary}

This assignment covers row reduction, solving equations, inverse functions and matrices, and some basic computations involving 2x2 and 3x3 determinants.  

\subsection{required reading}

\begin{itemize}
	\item LADW (Linear Algebra Done Wrong)     sections: 2.7, 2.8, 3.1, 3.5  
\end{itemize}




\section{Definitions}\label{defs}
Here is some notation we will use in this homework regarding the bookkeeping for coordinates in a basis.  This is a slightly more precise variation on the definition given in LADR 3.62, as we specifically note the basis in the notation.

\begin{DEF}[variation on LADR 3.62]

Let $\B = [v_1, \dots v_m]$ be a basis for a vector space, $V$.  We say that $x\in V$ has the coordinates $a_1,\dots,a_m$ in the basis, $\B$, if 
\begin{align*}
	x = a_1 v_1 + \cdots + a_m v_m.
\end{align*}
We use the following shorthand notation for this as
\begin{align}\label{eq:Coordinates}
	x=\ColVecFour{a_1}{a_2}{\vdots}{a_m}_{\B}\ \ \text{or}\ \
	x=\ColVecFour{a_1}{a_2}{\vdots}{a_m}_{[v_i]} \ \ \ \text{is defined as}\ \ \
	x = a_1 v_1 + \cdots + a_m v_m.
\end{align}

\end{DEF}


Note, when working on $\real^m$ with the \emph{canonical} basis, $[e_1,\dots, e_m]$, we will often suppress the specific indication of the basis.  That is to say, it is implied that when no basis is noted, it means the canonical one is used, 
\begin{align*}
	x=\ColVecFour{a_1}{a_2}{\vdots}{a_m} = \ColVecFour{a_1}{a_2}{\vdots}{a_m}_{[e_i]}.
\end{align*}







\section{Questions}


\begin{question}\label{que:MatrixInverseShowSteps}
	\normalfont
	Use the algorithm of LADW Chp 2, Sec 4 to compute the inverse of the following matrix $A$.  Show your steps and confirm that your answer is correct.  To confirm, this means that if you claim $B$ is the inverse of $A$, then you must show that $AB=BA=\Id$ (where $\Id$ is the identity matrix).
	\begin{align*}
		A = 
		\MatrixThreeThree{1}{0}{1}{1}{3}{0}{1}{2}{1}
	\end{align*}
 
 
\end{question}





\begin{question}
	\normalfont
	
	Let $A$ be the matrix,
	\begin{align*}
		A = 
		\MatrixThreeThree{1}{0}{1}{1}{3}{0}{3}{3}{2}
	\end{align*}
	
	\begin{enumerate}[(i)]
		\item Compute the determinant of the matrix, $A$. Show your work. 
		
		\item Is it possible to solve, for all $b\in\real^3$, the equation
		\begin{align*}
			Ax=b?
		\end{align*}
		If you answer is ``yes'', then prove that there is a solution for each $b$, and if your answer is ``no'', then you must give an example of a particular $b$ so that there is no vector $x$ that solves the equation, and prove that there is no solution. 
	\end{enumerate}
	
	
	
 
\end{question}

\vspace{1cm}

\begin{question}
	\normalfont
	
	
	Let $A$ be the $2\times2$ matrix given by
	\begin{align*}
		A = \MatrixTwoTwo{a}{b}{c}{d}.
	\end{align*}
	Let $B_i$ be the matrices given by
			\begin{align*}
				B_1 = \MatrixTwoTwo{0}{1}{1}{0},\ \ \ 
				B_2 = \MatrixTwoTwo{\cos\theta}{-\sin\theta}{\sin\theta}{\cos\theta},\ \ \
				B_3 = \MatrixTwoTwo{1}{0}{5}{1},\ \ \
				B_4 = \MatrixTwoTwo{2}{0}{0}{3}.
			\end{align*}
	For each of the $B_i$ matrices, above, complete the following:
	\begin{enumerate}[(i)]
		\item compute $B_iA$;
		\item draw a picture of $\real^2$, showing $e_1, e_2$ and $B_ie_1, B_ie_2$, and describe what has changed geometrically from the list $[e_1, e_2]$ to $[B_ie_1, B_ie_2]$;
		\item compute $\det(B_iA)$ in terms of the value of $\det(A)$.  (You are meant to do this calculation directly, from the definition of $2\times2$ determinant. Do not invoke the properties of determinant, such as LADW Theorem 3.5.)
	\end{enumerate}
\end{question}



\vspace{1cm}

\begin{question}
	\normalfont
	
	Let the matrices $A$ and $B$ be given as 
	\begin{align*}
		A=
		\begin{pmatrix}
			0 & 1 & 5 & 0\\
			1 & 1 & 0 & 1\\
			1 & 1 & 5 & 0
		\end{pmatrix}
		\ \ \ \text{and}\ \ \ 
		B=
		\begin{pmatrix}
			1 & 1 & 5 & 0\\
			0 & 1 & 5 & 0\\
			0 & 0 & -5 & 1
		\end{pmatrix}
	\end{align*}
	
	
	\begin{enumerate}[(i)]
		\item Show that the matrix $A$ can be row-reduced to the matrix $B$, using row operations.  Show your steps.
		
		\item Write down an invertible matrix, $T$, so that $TA=B$.  Do not supply a proof, just write down a matrix.  (Hint, keep track of the elementary matrices-- LADW p. 41, 42-- that correspond to your steps in part (i))
		
		\item Prove that $T$ is invertible.  (There are actually many ways one could argue this.)
	\end{enumerate}
	 
	
	
\end{question}

\vspace{1cm}

\begin{question}
	\normalfont
	
	Let $A$ be the following matrix:
	\begin{align*}
		A = 
		\begin{pmatrix}
			1 & 0 & 3 & 1\\
			1 & 1 & 0 & 2\\
			1 & 0 & 3 & 1
		\end{pmatrix}.
	\end{align*}
	
	\begin{enumerate}[(i)]
		\item Let $R$ be the matrix that is the reduced echelon form of $A$.  Write down $R$.  You do not need to show your steps. 
		   
		
		
		\item Define the linear function, $L_A:\real^4\to\real^3$, given by 
		\begin{align*}
			L_A(x) = Ax\ \ \ \text{as matrix multiplication in the canonical bases}.
		\end{align*}
		Use the reduction algorithm of LADR 2.31, starting with the ordered list, 
		\begin{align*}
			[L_A(e_1), L_A(e_2), L_A(e_3), L_A(e_4)],
		\end{align*}
		to give a basis for $\range(L_A)$.  You \textbf{do not} need to show your steps in the algorithm, just state the resulting basis.
		
		
		\item Let $b\in\real^3$ be given as
		\begin{align*}
			b=\ColVecThree{4}{-1}{4}.
		\end{align*} 
		Represent $b$ uniquely using the basis vectors for $\range(L_A)$ from part (ii).  Use this to solve for $x$, in the equation
		\begin{align*}
			L_A(x) = b.
		\end{align*}
		
		\item For the same $b$ as in part (iii), use the row reduction algorithm in LADW section 2.2 to solve for one solution, $x$, in the equation
		\begin{align*}
			Ax=b.
		\end{align*}
		(Because $\NullLA(L_A)\not=\{0\}$, there are infinitely many solutions, but you just need to give one here.)
		
		
		\item For the matrix, $R$, from part (i), define the linear function, $L_R:\real^4\to\real^3$, given by 
		\begin{align*}
			L_R(x) = Rx\ \ \ \text{as matrix multiplication in the canonical bases}.
		\end{align*}
		
		Why will solving for $x$ in the equation
		\begin{align*}
			L_R(x)=b
		\end{align*}
		not give the correct solution to $L_A(x)=b$?  Give a new vector, $z$, so that
		
		\begin{align*}
			L_A(x)=b\ \ \ \iff\ \ \ L_R(x)=z.
		\end{align*}
	\end{enumerate}
	
	
\end{question}

\vspace{1cm}

\begin{question}\label{que:PrimePlusPrimePrime}
	\normalfont
	Define the vector space $V$ as
	\begin{align*}
		V = \{ p\in\P_4\ :\ p(0)=0 \}.
	\end{align*}
	Define the function, $L:V\to\P_3$ as
	\begin{align*}
		L(f) = f'+f''.
	\end{align*}
	
	\begin{enumerate}[(i)]
		\item Briefly observe / justify that
		\begin{align*}
			V = \{ f\in\P_4,\ :\ f(x) = a_1x + a_2x^2 +a_3x^3 + a_4 x^4,\ \text{for some},\ a_1,\dots,a_4 \in\real\}.
		\end{align*}
		
		\item Compute $\NullLA(L)$.  Prove that your answer is correct. 
		
		\item Prove that $L$ is a bijection.
		
		\item By a direct calculation, write down a formula for $L^{-1}$.  That is to say, given a generic $q\in \P_3$, with 
		\begin{align*}
			q(x)= b_0 + b_1x + b_2x^2 + b_3x^3,
		\end{align*}
		if $L^{-1}(q)=p$, with 
		\begin{align*}
			p(x) = a_1x + a_2x^2 +a_3x^3 + a_4 x^4,
		\end{align*}
		you need to specify the coefficients $a_1,\dots, a_4$ in terms of the given $b_0,\dots, b_3$. 
		
		(Hint: you know that $L^{-1}(q)=p$ if and only if $q=L(p)$.  This should allow you to set up 4 equations in 4 unknowns-- the $a_i$-- and you should be able to solve directly for $a_i$ in terms of $b_i$.)
		
		\item Confirm that your answer is correct.  That is to say, you must demonstrate that $L(L^{-1}(q))=q$ and $L^{-1}(L(p))=p$.
		
	\end{enumerate}
 
 
\end{question}

\vspace{1cm}

\begin{question}
	\normalfont
	
	Assume that $V$ and $L$ are as in Question \ref{que:PrimePlusPrimePrime}. Let $p_0,\dots, p_3$ be the canonical basis polynomials for $\P_3$
	\begin{align*}
		p_0(x) = 1,\ \ \ 
		p_1(x) = x,\ \ \
		p_2(x) = x^2,\ \ \ 
		p_3(x) = x^3.
	\end{align*}
	Let the following $q_1,\dots, q_4$ be a choice of basis for $V$:
	\begin{align*}
		q_1(x) = x,\ \ \
		q_2(x) = x^2,\ \ \ 
		q_3(x) = x^3,\ \ \ 
		q_4(x) = x^4.
	\end{align*}
	
	\begin{enumerate}[(i)]
		\item Write down a matrix, $A$, so that in the bases for $V$ and $\P_3$ given above,
		\begin{align*}
			&\text{if}\ \ 
			p = a_1q_1+ \cdots + a_4 q_4,\ \ \text{i.e.}\ \ p=\ColVecThree{a_1}{\vdots}{a_4}_{[q_i]},\\
			&\ \ \text{and}\ \ 
			L(p) = b_0p_0 + \cdots + b_3p_3 = \ColVecThree{b_0}{\vdots}{b_3}_{[p_i]},\\
			&\text{then}
			\ColVecThree{b_0}{\vdots}{b_3}
			= A\ColVecThree{a_1}{\vdots}{a_4}\ \ \text{as matrix multiplication}
		\end{align*}
		
		\item Using the algorithm of LADW Chp 2, Sec 4, compute the inverse matrix, $A^{-1}$.  Confirm by matrix multiplication that your answer is correct.  No need to show your steps, as you have done a similar computation in Question \ref{que:MatrixInverseShowSteps}.  Just write the matrix and confirm.
		
		\item  For the sake of comparing methods, do this part pretending you have not done Question 3.6 yet.
		
		
		 Using the matrix $A^{-1}$, write down the formula for $L^{-1}$.  Given that a linear function is uniquely determined by its action on a basis, it is OK to specify $L^{-1}$ by specifying $L^{-1}(p_i)$.  Or, you can just write down a formula for $L^{-1}(q)$ for a generic $q$.
	\end{enumerate}
\end{question}






%%%%%%%%%%%%%%%%%%%%%%%%%%%%%%%%%%%%%%%%%%%%%%%%%%%%%%%%%%%%%%%%%%%%%%%%%%%%%%%%%%%%%%%%%%%

%%%%%%%%%%%%%%%%%%%%%%%%%%%%%%%%%%%%%%%%%%%%%%%%%%%%%%%%%%%%%%%%%%%%%%%%%%%%%%%%%%%%%%%%%%%
\end{document}





















