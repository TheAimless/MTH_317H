\documentclass[12pt]{article}
\usepackage{amssymb}
\usepackage{epsfig}
\usepackage{psfrag}
\usepackage{fullpage}
\usepackage{color}
\usepackage{amsfonts,epsf}
\usepackage{amsmath,amssymb,
amscd,amsbsy, bbm, amsthm, enumerate, url}
\usepackage{wasysym}


\newtheorem{thm}{Theorem}[section]
\newtheorem{question}[thm]{Question}
\newtheorem{prop}[thm]{Proposition}
\newtheorem{lem}[thm]{Lemma}
\newtheorem{DEF}[thm]{Definition}
\newtheorem{rem}[thm]{Remark}
\newenvironment{sol}{\paragraph{Solution:}\color{red}}{\hfill$\square$}
\newenvironment{grade}{\paragraph{grading:}\color{blue}}{\hfill$\square$}



\def\real{{\mathbb R}}
\def\Natural{\mathbb{N}}
\def\Z{\mathbb{Z}}
\def\integer{\mathbb{Z}}
\def\dx{\textnormal{dx}}
\def\dy{\textnormal{dy}}
\def\dz{\textnormal{dz}}
\def\dt{\textnormal{dt}}
\def\ds{\textnormal{ds}}
\def\dw{\textnormal{dw}}
\def\Re{\textnormal{Re}}
\def\Im{\textnormal{Im}}
\def\Id{\textnormal{Id}}
\def\exp{\textnormal{exp}}
\def\dim{\textnormal{dim}}
\def\interior{\textnormal{interior}}
\def\NullLA{\textnormal{null}}
\def\SpanLA{\textnormal{span}}
\def\range{\textnormal{range}}
\def\al{\alpha}
\def\del{\delta}
\def\Del{\Delta}
\def\gam{\gamma}
\def\Gam{\Gamma}
\def\Om{\Omega}
\def\ep{\varepsilon}
\def\lam{\lambda}
\def\rational{{\mathbb Q}}
\def\integer{{\mathbb Z}}
\def\grad{\nabla}


\def\B{\mathcal B}
\def\C{\mathcal C}
\def\I{\mathcal I}
\def\M{\mathcal M}
\def\N{\mathbb N}
\def\P{\mathcal P}
\def\R{\mathbb R}
\def\T{\mathcal T}
\def\Z{\mathbb Z}

\newcommand{\abs}[1]{\left| #1 \right|}
\newcommand{\ColVecTwo}[2]{\begin{pmatrix} #1\\ #2\end{pmatrix}}
\newcommand{\ColVecThree}[3]{\begin{pmatrix} #1\\ #2\\ #3\end{pmatrix}}
\newcommand{\ColVecFour}[4]{\begin{pmatrix} #1\\ #2\\#3 \\ #4\end{pmatrix}}
\newcommand{\MatrixTwoTwo}[4]{\begin{pmatrix} #1 &#2\\ #3 &#4 \end{pmatrix}}
\newcommand{\inner}[2]{\langle #1 , #2 \rangle}
\newcommand{\norm}[1]{\left\lVert#1\right\rVert}
\newcommand{\spanvect}{\textnormal{span}}
\newcommand{\union}{\cup}
\newcommand{\Union}{\bigcup}
\newcommand{\intersect}{\cap}
\newcommand{\Intersect}{\bigcap}

\DeclareMathOperator*{\Limsup}{LIMSUP}



\pagestyle{empty}
\begin{document}

	\begin{LARGE}
	\begin{center}
		
		

		
	
	\textbf{Homework 10; Due date: 11/30/2023}
	

	(MTH 317H, Honors Linear Algebra;  Spring 2023)
	\end{center}
	\end{LARGE}
	\vspace{0.15in}
	
	
	

	
\section{Commentary}

This homework assignment explores eigenvectors and eigenvalues.

\subsection{required reading}

\begin{itemize}
	\item LADW (Linear Algebra Done Wrong) sections 4.1 and 4.2.
\end{itemize}




\section{Questions}

\vspace{.5cm}




\begin{question}
\normalfont
Find the characteristic polynomials, eigenvalues, and eigenvectors of each of the following matrices. For each eigenvalue, find its algebraic multiplicity and geometric multiplicity.
\begin{enumerate}[(i)]
\item
\[
A= \begin{pmatrix}
			4 & -5\\
			2 & -3\\
		\end{pmatrix}
\]

\item
\[
B= \begin{pmatrix}
			2 & -1 & 0\\
			0 & 2 & 1\\
			0 & 0 & 1
		\end{pmatrix}
\]
\item
\[
C= \begin{pmatrix}
			1 & 3 & 3\\
			-3 & -5 & -3\\
			3 & 3 & 1 \\
		\end{pmatrix}
\]
\end{enumerate}

\end{question}
\vspace{1cm}



\vspace{1cm}

\begin{question}
\normalfont
Determine whether each of the following statements is True or False. If it is true, provide a proof. If it is false, provide a counterexample. 
\begin{enumerate}[(i)]

\item If a matrix has one eigenvector, it has infinitely many eigenvectors.

\item Similar matrices always have the same eigenvectors.

\item A non-zero sum of two eigenvectors of a matrix $A$ is always an eigenvector.

\item A non-zero sum of two eigenvectors of a matrix $A$ corresponding to the same eigenvalue $\lambda$ is always an eigenvector. 

\end{enumerate}

\end{question}
\vspace{1cm}


\begin{question}\normalfont

Recall that a matrix $A$ is called \emph{nilpotent} if $A^k=0$ for some $k$. 

\begin{enumerate}[(i)]
\item Show that the matrix 
\[
B =\begin{pmatrix}
			2 & 2 & -2\\
			5 & 1 & -3\\
			1 & 5 & -3 \\
		\end{pmatrix}
\]
is nilpotent.

\item Find the eigenvalues of the matrix $B$. You do not need to find the associated eigenvectors. 

\item Prove that for any nilpotent matrix $A$, the only eigenvalue of $A$ is 0. 
\end{enumerate}
	
	\end{question}



\vspace{1cm}

\begin{question} 
\normalfont 
\begin{enumerate}[(i)]
 
 \item Find the eigenvalues of the following matrix. You do not need to find the eigenvectors.
 \[
 B = \begin{pmatrix}
			1 & 2 & 3 & 2\\
			0 & 2 & 7 & -2\\
			0 & 0 & -2 & 5\\
			0 & 0 & 0 & 3	
		\end{pmatrix}
 \]
 \vspace{.2cm}

\item Let $A$ be an $n\times n$ triangular matrix. Prove that the eigenvalues of $A$ (repeated with multiplicity) are the diagonal entries of $A$: $a_{1,1}, a_{2,2}, \ldots a_{n,n}$. 
	\normalfont		
\end{enumerate}	
	
	\end{question}




\vspace{1cm}

\begin{question}\label{degree}	\normalfont
Let $A$ be an $n\times n$ matrix, and let $f(\lambda) = \textup{det}(A - \lambda I)$ be its characteristic polynomial. Prove that $f(\lambda)$ is a polynomial of degree $n$, and the coefficient of $\lambda^n$ in $f(\lambda)$ is $(-1)^n$. \\
\emph{Hint: Use cofactor expansion and induction.}	
\end{question}
	


\vspace{1cm}

\begin{question}
\normalfont In this problem you will prove that the determinant of an $n\times n$ matrix $A$ is the product of its eigenvalues (repeated with multiplicity), i.e. det$(A) =\lambda_1\lambda_2...\lambda_n$. You will do this in two steps:
\begin{enumerate}[(i)]
\item First, show that 
\[
\textup{det}(A-\lambda I) = (\lambda_1 - \lambda)(\lambda_2 - \lambda) \ldots (\lambda_n-\lambda),
\]
where $\lambda_1, \lambda_2,...\lambda_n$ are the eigenvalues of $A$ (repeated with multiplicity). \\ \\
\emph{Hint: Use the fact that the eigenvalues are the roots of the characteristic polynomial, plus your result from Question \ref{degree} about the coefficient of $\lambda^n$ in the characteristic polynomial.}

\item Then, plug $\lambda=0$ into your formula from part (i) to conclude that the determinant of $A$ is the product of the eigenvalues. 
\end{enumerate}



\end{question}




\vspace{1cm}


	
	\vspace{.5cm}
	


\begin{question}	
	\normalfont
Let $R$ be any polygon in the plane. Prove that it is possible to divide $R$ into triangles, all of whose vertices are vertices of $R$.
Your inductive proof must include the following steps:
	\begin{enumerate}[(i)]
		\item identify and state what is the conditional statement, $P(n)$, for $n\in \Natural, n \geq 3$.
		\item state which natural number you will use as the base case, and prove that the conditional statement is true for the base case.
		\item for a fixed $k\in\Natural$, state the inductive assumption.
		\item prove the inductive step.  That is to say, prove that $P(k+1)$ is also true.
		\item conclude.
	\end{enumerate}
\end{question}







%%%%%%%%%%%%%%%%%%%%%%%%%%%%%%%%%%%%%%%%%%%%%%%%%%%%%%%%%%%%%%%%%%%%%%%%%%%%%%%%%%%%%%%%%%%

%%%%%%%%%%%%%%%%%%%%%%%%%%%%%%%%%%%%%%%%%%%%%%%%%%%%%%%%%%%%%%%%%%%%%%%%%%%%%%%%%%%%%%%%%%%
\end{document}





















