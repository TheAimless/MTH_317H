\documentclass[12pt]{article}
\usepackage{amssymb}
\usepackage{epsfig}
\usepackage{psfrag}
\usepackage{fullpage}
\usepackage{color}
\usepackage{amsfonts,epsf}
\usepackage{amsmath,amssymb,
amscd,amsbsy, bbm, amsthm, enumerate, url}
\usepackage{wasysym}


\newtheorem{thm}{Theorem}[section]
\newtheorem{question}[thm]{Question}
\newtheorem{prop}[thm]{Proposition}
\newtheorem{lem}[thm]{Lemma}
\newtheorem{DEF}[thm]{Definition}
\newtheorem{rem}[thm]{Remark}
\newenvironment{sol}{\paragraph{Solution:}\color{red}}{\hfill$\square$}
\newenvironment{grade}{\paragraph{grading:}\color{blue}}{\hfill$\square$}



\def\real{{\mathbb R}}
\def\Natural{\mathbb{N}}
\def\Z{\mathbb{Z}}
\def\integer{\mathbb{Z}}
\def\dx{\textnormal{dx}}
\def\dy{\textnormal{dy}}
\def\dz{\textnormal{dz}}
\def\dt{\textnormal{dt}}
\def\ds{\textnormal{ds}}
\def\dw{\textnormal{dw}}
\def\Re{\textnormal{Re}}
\def\Im{\textnormal{Im}}
\def\Id{\textnormal{Id}}
\def\exp{\textnormal{exp}}
\def\dim{\textnormal{dim}}
\def\interior{\textnormal{interior}}
\def\NullLA{\textnormal{null}}
\def\SpanLA{\textnormal{span}}
\def\range{\textnormal{range}}
\def\al{\alpha}
\def\del{\delta}
\def\Del{\Delta}
\def\gam{\gamma}
\def\Gam{\Gamma}
\def\Om{\Omega}
\def\ep{\varepsilon}
\def\lam{\lambda}
\def\rational{{\mathbb Q}}
\def\integer{{\mathbb Z}}
\def\grad{\nabla}

\def\C{\mathcal C}
\def\I{\mathcal I}
\def\M{\mathcal M}
\def\N{\mathbb N}
\def\P{\mathcal P}
\def\R{\mathbb R}
\def\T{\mathcal T}
\def\Z{\mathbb Z}

\newcommand{\abs}[1]{\left| #1 \right|}
\newcommand{\ColVecTwo}[2]{\begin{pmatrix} #1\\ #2\end{pmatrix}}
\newcommand{\ColVecThree}[3]{\begin{pmatrix} #1\\ #2\\ #3\end{pmatrix}}
\newcommand{\ColVecFour}[4]{\begin{pmatrix} #1\\ #2\\#3 \\ #4\end{pmatrix}}
\newcommand{\MatrixTwoTwo}[4]{\begin{pmatrix} #1 &#2\\ #3 &#4 \end{pmatrix}}
\newcommand{\inner}[2]{\langle #1 , #2 \rangle}
\newcommand{\norm}[1]{\left\lVert#1\right\rVert}
\newcommand{\spanvect}{\textnormal{span}}
\newcommand{\union}{\cup}
\newcommand{\Union}{\bigcup}
\newcommand{\intersect}{\cap}
\newcommand{\Intersect}{\bigcap}

\DeclareMathOperator*{\Limsup}{LIMSUP}



\pagestyle{empty}
\begin{document}

\begin{LARGE}
    \begin{center}




        \textbf{Homework 3; Due THURSDAY, 09/21/2023}


        (MTH 317H, Honors Linear Algebra;  Fall 2023)
    \end{center}
\end{LARGE}
\vspace{0.15in}




\section{Commentary}

In this homework you will practice techniques related to span, linear combination, and linear independence.



\subsection{required reading}





\noindent
``LADR'' -- Linear Algebra done right


\noindent
``BOP'' -- Book of Proof

\begin{itemize}
    \item LADR, sections 2A, 2B, 2C.  For the moment, we will exclusively use the scalar field to be the real numbers, $\real$.  That is to say, any time you encounter $\mathbf{F}$, or $\mathbf{F^n}$, you are free to assume $\mathbf{F}=\real$.  (This will change later in the semester.)


    \item BOP, sections 2.8--2.12, chp 4  (and review sections 1.1 -- 1.7, 2.1--2.7)
\end{itemize}





\section{Questions}



\vspace{.5cm}

\begin{question}\label{que:LinDepInR3}
    \normalfont


    Let $v_1, v_2, v_3, v_4$ be the following vectors in $\real^3$.
    \begin{align*}
        v_1 = \ColVecThree{1}{1}{1}\ \ \
        v_2 = \ColVecThree{1}{2}{3}\ \ \
        v_3 = \ColVecThree{1}{0}{1}\ \ \
        v_4 = \ColVecThree{3}{2}{5}.
    \end{align*}


    \begin{enumerate}[(i)]
        \item Prove that $v_1, v_2, v_3, v_4$ \hspace{.1cm} is a linearly dependent list.
        \item Prove that $v_4\in\SpanLA(v_1,v_2,v_3)$.
        \item Prove that $v_1\not\in\SpanLA(v_2, v_3, v_4)$.
    \end{enumerate}
\end{question}

\vspace{1cm}

\begin{question}
    \normalfont


    Consider the same list of vectors $v_1, v_2, v_3, v_4$ \hspace{.1cm} as in Question \ref{que:LinDepInR3}.
    \begin{enumerate}[(i)]
        \item Prove that $\SpanLA(v_1, v_2, v_3, v_4) = \SpanLA(v_1, v_2, v_3)$.

        \item Prove that $\SpanLA(v_1, v_2, v_3) = \real^3$.

              Here, you must use the basic set equality proof structure, namely to show as sets that $A=B$, you must demonstrate $A\subseteq B$ and $B\subset A$.

        \item Prove that $\SpanLA(v_1, v_2)\not=\real^3$.
    \end{enumerate}
\end{question}
\vspace{1cm}




\begin{question}
    \normalfont
    Let $u_1, u_2, v_1, v_2$ be vectors in $\real^3$, and let
    \begin{align*}
        U=\SpanLA(u_1,u_2)\ \ \ \textup{and} \ \ \
        V=\SpanLA(v_1,v_2).
    \end{align*}
    Give an example of specific vectors $u_1, u_2, v_1, v_2$ in $\real^3$ so that
    \begin{align*}
        \{u_1, u_2\}\intersect \{ v_1, v_2 \} = \emptyset,
    \end{align*}
    and yet, $U=V$.  \textbf{No proof needed.}


\end{question}


\vspace{1cm}




\begin{question}
    \normalfont

    Assume that $V$ is a vector space, $v_1,v_2,v_3,w\in V$, and
    \begin{enumerate}[(a)]
        \item $v_1,v_2,v_3$ \hspace{.1cm} is a linearly independent list of vectors.
        \item $v_1,v_2,v_3,w$ \hspace{.1cm}  is a linearly dependent list of vectors.
    \end{enumerate}
    Prove that there is a unique choice of $\al_1,\al_2,\al_3\in\real$ so that
    \begin{align*}
        w=\sum_{i=1}^3 \al_iv_i.
    \end{align*}


    You must include the following two parts in your proof.
    \begin{enumerate}[(i)]
        \item   Explain (prove) to your reader why there must exist at least one choice of $\al_1,\al_2,\al_3$.

        \item   Explain (prove) to your reader why this choice must be unique.  As for right now, all uniqueness proofs of this type should have the following structure:

              \begin{enumerate}[(1)]
                  \item Assume there exist two collections, $\al_1,\al_2,\al_3$ and $\beta_1,\beta_2,\beta_3$ that give the desired construction of $w$.
                  \item  Prove that for $i=1,2,3$, $\al_i=\beta_i$. (Hint: you have two equations that give $w$ as a result. What happens if you subtract one of these equations from the other?)

              \end{enumerate}
    \end{enumerate}



\end{question}

\vspace{1cm}

\begin{question}\label{que:SpanIntersection}
    \normalfont

    \begin{enumerate}[(i)]
        \item Let $V$ be a vector space, and let $A$ = $\{v_1,\dots, v_m\}$ and $B=\{ w_1, \dots, w_n\}$.  Prove that
              \begin{align*}
                  \SpanLA(A\intersect B) \subseteq \SpanLA(A)\intersect\SpanLA(B).
              \end{align*}


        \item Give an example for $V=\real^3$ of two \emph{non-empty} sets $A$ and $B$ as above, in which
              \begin{align*}
                  \SpanLA(A)\intersect \SpanLA(B) \not\subseteq \SpanLA(A\intersect B).
              \end{align*}


    \end{enumerate}
\end{question}

\vspace{1cm}

\begin{question}\label{que:SpanAgain}
    \normalfont

    Assume that  $u,v,w_1,\dots,w_n$ are all distinct elements of a vector space, $V$.  Define the lists of vectors, $B$ and $C$, as
    \begin{align*}
        B =  u, w_1,\dots, w_n \ \ \ \text{and}\ \ \
        C =  v, w_1,\dots, w_n .
    \end{align*}
    Prove the following implication:
    \begin{align*}
        \text{if}\ B\ \text{is a linearly independent list and}\  u\in\SpanLA(C),\ \ \text{then}\ \ \SpanLA(B)=\SpanLA(C).
    \end{align*}
\end{question}

\vspace{1cm}



\vspace{.5cm}


\begin{question}
    \normalfont
    Consider the following vectors in $\R^3$

    $$ v_1 =  \left( \begin{array}{c}
                1  \\
                2  \\
                -1 \\
            \end{array} \right),\ \ \   v_2 =  \left( \begin{array}{c}
                0  \\
                1  \\
                -2 \\
            \end{array}\right), \ \ \
        v_3 =  \left( \begin{array}{c}
                3 \\
                0 \\
                9 \\
            \end{array} \right). $$
    Is the list of vectors $v_1, v_2, v_3$ \hspace{.1cm} a basis for $\R^3$? Justify your answer.

\end{question}
\vspace{1cm}
\begin{question}
    \normalfont

    Consider the polynomials
    \begin{align*}
        p_1(x) = 1-x,\ \ \ \ p_2(x)=x^2+x,\ \ \ \ p_3(x)=x^3+x^2,\ \ \ \ p_4(x)=x^3.
    \end{align*}

    Prove that the list of polynomials $p_1, p_2, p_3, p_4$ \hspace{.1cm} is a basis for $\P_3$.


\end{question}





%%%%%%%%%%%%%%%%%%%%%%%%%%%%%%%%%%%%%%%%%%%%%%%%%%%%%%%%%%%%%%%%%%%%%%%%%%%%%%%%%%%%%%%%%%%

%%%%%%%%%%%%%%%%%%%%%%%%%%%%%%%%%%%%%%%%%%%%%%%%%%%%%%%%%%%%%%%%%%%%%%%%%%%%%%%%%%%%%%%%%%%
\end{document}





















