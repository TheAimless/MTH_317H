\documentclass[12pt]{article}
\usepackage{amssymb}
\usepackage{epsfig}
\usepackage{psfrag}
\usepackage{fullpage}
\usepackage{color}
\usepackage{amsfonts,epsf}
\usepackage{amsmath,amssymb,
amscd,amsbsy, bbm, amsthm, enumerate, url}
\usepackage{wasysym}


\newtheorem{thm}{Theorem}[section]
\newtheorem{question}[thm]{Question}
\newtheorem{prop}[thm]{Proposition}
\newtheorem{lem}[thm]{Lemma}
\newtheorem{DEF}[thm]{Definition}
\newtheorem{rem}[thm]{Remark}
\newenvironment{sol}{\paragraph{Solution:}\color{red}}{\hfill$\square$}
\newenvironment{grade}{\paragraph{grading:}\color{blue}}{\hfill$\square$}



\def\real{{\mathbb R}}
\def\Natural{\mathbb{N}}
\def\Z{\mathbb{Z}}
\def\integer{\mathbb{Z}}
\def\dx{\textnormal{dx}}
\def\dy{\textnormal{dy}}
\def\dz{\textnormal{dz}}
\def\dt{\textnormal{dt}}
\def\ds{\textnormal{ds}}
\def\dw{\textnormal{dw}}
\def\Re{\textnormal{Re}}
\def\Im{\textnormal{Im}}
\def\Id{\textnormal{Id}}
\def\exp{\textnormal{exp}}
\def\dim{\textnormal{dim}}
\def\interior{\textnormal{interior}}
\def\NullLA{\textnormal{null}}
\def\SpanLA{\textnormal{span}}
\def\range{\textnormal{range}}
\def\al{\alpha}
\def\del{\delta}
\def\Del{\Delta}
\def\gam{\gamma}
\def\Gam{\Gamma}
\def\Om{\Omega}
\def\ep{\varepsilon}
\def\lam{\lambda}
\def\rational{{\mathbb Q}}
\def\integer{{\mathbb Z}}
\def\grad{\nabla}

\def\C{\mathcal C}
\def\I{\mathcal I}
\def\M{\mathcal M}
\def\N{\mathbb N}
\def\P{\mathcal P}
\def\R{\mathbb R}
\def\T{\mathcal T}
\def\Z{\mathbb Z}

\newcommand{\abs}[1]{\left| #1 \right|}
\newcommand{\ColVecTwo}[2]{\begin{pmatrix} #1\\ #2\end{pmatrix}}
\newcommand{\ColVecThree}[3]{\begin{pmatrix} #1\\ #2\\ #3\end{pmatrix}}
\newcommand{\ColVecFour}[4]{\begin{pmatrix} #1\\ #2\\#3 \\ #4\end{pmatrix}}
\newcommand{\MatrixTwoTwo}[4]{\begin{pmatrix} #1 &#2\\ #3 &#4 \end{pmatrix}}
\newcommand{\inner}[2]{\langle #1 , #2 \rangle}
\newcommand{\norm}[1]{\left\lVert#1\right\rVert}
\newcommand{\spanvect}{\textnormal{span}}
\newcommand{\union}{\cup}
\newcommand{\Union}{\bigcup}
\newcommand{\intersect}{\cap}
\newcommand{\Intersect}{\bigcap}

\DeclareMathOperator*{\Limsup}{LIMSUP}



\pagestyle{empty}
\begin{document}

	\begin{LARGE}
	\begin{center}
		
	
	\textbf{Homework 4; Due THURSDAY, 09/28/2023}
	

	(MTH 317H, Honors Linear Algebra;  Fall 2023)
	\end{center}
	\end{LARGE}
	\vspace{0.15in}
	
	

\section{Commentary}

In this homework you will continue to understand the techniques related to linear dependence / independence, span, and basis.  Additionally, you will start to use the notion of dimension.



\subsection{required reading}


\noindent
``LADR'' -- Linear Algebra done right



\noindent
``BOP'' -- Book of Proof

\begin{itemize}
	\item LADR, sections  3A, 3B (and review 2A, 2B, 2C).  For the moment, we will exclusively use the scalar field to be the real numbers, $\real$.  That is to say, any time you encounter $\mathbf{F}$, or $\mathbf{F^n}$, you are free to assume $\mathbf{F}=\real$.  (This will change later in the semester.)
	
	
	\item BOP, chp 5, chp 6.
\end{itemize}





\section{Questions}





\begin{question}\label{que:ReductionAlgorithm}
	\normalfont
	
	
	Let $\ell=v_1,\dots, v_6$ and $s=w_1\dots,w_6$ be ordered lists of vectors in $\real^3$, where:
	\begin{align*}
		v_1 = \ColVecThree{1}{2}{3}\ \ 
		v_2 = \ColVecThree{1}{1}{1}\ \
		v_3 = \ColVecThree{3}{4}{5}\ \ 
		v_4 = \ColVecThree{2}{1}{2}\ \ 
		v_5 = \ColVecThree{0}{0}{1}\ \ 
		v_6 = \ColVecThree{1}{0}{3}
	\end{align*}
	and
	\begin{align*}
		w_1 = \ColVecThree{1}{0}{3}\ \ 
		w_2 = \ColVecThree{0}{0}{1}\ \
		w_3 = \ColVecThree{1}{2}{3}\ \ 
		w_4 = \ColVecThree{1}{1}{1}\ \ 
		w_5 = \ColVecThree{3}{4}{5}\ \ 
		w_6 = \ColVecThree{2}{1}{2}.
	\end{align*}
	
	\vspace{.2cm}
	
	Recall that the vectors in the canonical basis for $\real^3$ are
	
	\begin{align*}
		e_1 = \ColVecThree{1}{0}{0}\ \
		e_2 = \ColVecThree{0}{1}{0}\ \
		e_3 = \ColVecThree{0}{0}{1}. 
	\end{align*}
	
	\vspace{.2cm}
	\begin{enumerate}[(i)]
		\item Prove that $e_1,e_2,e_3\in\SpanLA(\ell)$.
		\item Prove that $\SpanLA(\ell)=\real^3$.
		\item Use the reduction algorithm in LADR Result 2.31 to reduce the list $\ell$ to a basis.  What is the resulting list?    For this you must show each step of the algorithm, including the calculation of whether or not a vector is in the span of the previous vectors.  Then just state the resulting list.  Give a one sentence explanation for why your answer for the span of the resulting list is what it is.
		\item Use the reduction algorithm in LADR Result 2.31 to reduce the list $s$ to a basis.  What is the resulting list?  Is it the same list as the algorithm applied to $\ell$?  Does order matter in this algorithm?  Does the ordering of the original vectors affect the span of the reduced list?
	\end{enumerate}
\end{question}

\vspace{.3cm}





\begin{question}
	\normalfont
	
	Define the list, $\ell=x_1,x_2,x_3$ \hspace{.1cm} of vectors in $\real^3$, where 
	\begin{align*}
		x_1 = \ColVecThree{1}{2}{3}\ \ 
		x_2 = \ColVecThree{1}{2}{1}\ \
		x_3 = \ColVecThree{1}{0}{1}.
	\end{align*}
	You will prove that $\ell$ is a basis for $\real^3$ via two different methods.
	\begin{enumerate}[(i)]
		\item Prove, from the definitions of linear independence and span that $\ell$ is linearly independent and $\SpanLA(\ell)=\real^3$, hence that $\ell$ is a basis of $\real^3$.  For the proof of the span, you must use the typical set equality proof, in which you will establish that $\SpanLA(\ell)\subseteq \real^3$ and $\real^3\subseteq \SpanLA(\ell)$.  Note, for your span proof, it may be easiest to first show that the canonical basis vectors satisfy $e_1,e_2,e_3\in \SpanLA(\ell)$.
		
		\item Use some combination of the results in LADR 2.C to prove that $\ell$ is a basis for $\real^3$.
		
		\item Do you have a preference for either method?
		
	\end{enumerate}

\end{question}


\vspace{.3cm}








\begin{question}
	\normalfont
	
	
	Define a list $\ell = p_1, p_2$ \hspace{.1cm}of vectors in $\P_3$, where
	\begin{align*}
		p_1(x) = 1+x^2,\ \ \ 
		p_2(x) = 1-x^2.
	\end{align*}
	Add vectors to the list, $\ell$, so that it becomes a basis for $\P_3$.  Prove that your new list is both linearly independent and that it spans $\P_3$.
\end{question}


\vspace{.3cm}

\begin{question}
	\normalfont
	Assume that the list $B= q_1, \dots, q_m $ \hspace{.1cm} is a basis for $\P_4$.
	\begin{enumerate}[(i)]
		\item Is it possible that no polynomial in $B$ has degree 4? Prove your answer.
		\item Is it possible that no polynomial in $B$ has degree 2? Prove your answer.
		\item What must be the value of $m$ (the length of the list, $B$)?  Prove your answer. \\
	\end{enumerate}
\end{question}

\vspace{.3cm}

\noindent \textbf{\emph{Note: in the following questions the vectors, $v_i$, have nothing to do with Question \ref{que:ReductionAlgorithm}. The vector space and vectors in the following questions are generic.}} \\


\begin{question}
	\normalfont
	
		
	\noindent Assume that \hspace{.1cm} $v_1, v_2, v_3, v_4$ \hspace{.1cm} is a linearly independent list of vectors in a vector space, $V$.
	
	\begin{enumerate}[(i)]
		\item Define the vectors, $w_i$ as
		\begin{align*}
			w_1 = v_1 + v_2,\ \ \ 
			w_2 = v_2 + v_3,\ \ \ 
			w_3 = v_3 + v_4,\ \ \
			w_4 = v_4 + v_1.
		\end{align*}
		Determine if the list of vectors \hspace{.1cm} $w_1, w_2, w_3, w_4$ \hspace{.1cm} is linearly dependent or is linearly independent. Justify your answer with a proof.		
		\item Define the vectors, $u_i$, as
		\begin{align*}
			u_1 = v_1 + v_2,\ \ \ 
			u_2 = v_2 + v_3,\ \ \ 
			u_3 = v_3 + v_4,\ \ \
			u_4 = v_4 - v_1.
		\end{align*}
		Determine if the list of vectors \hspace{.1cm}  $ u_1, u_2, u_3, u_4 $  \hspace{.1cm} is linearly dependent or is linearly independent.  Justify your answer with a proof.	
	\end{enumerate}
\end{question}

\vspace{.3cm}

\begin{question}
	\normalfont
	

	
	Assume that $V$ is a vector space, $v_1,v_2,v_3,v_4\in V$, and
	\begin{enumerate}[(a)]
		\item $v_1,v_2,v_3$ \hspace{.1cm} is a linearly dependent list of vectors.
		\item $v_2,v_3,w$ \hspace{.1cm} is a linearly independent list of vectors.
	\end{enumerate}
	Prove that
	\begin{enumerate}[(i)]
		\item $v_1$ is a linear combination of $v_2$ and $v_3$.
		\item $w$ is not a linear combination of $v_1, v_2, v_3$.
	\end{enumerate}

	
\end{question}

\vspace{.3cm}

\begin{question}\label{que:GeneralVectorSpaceNewBasisFromOld}
	\normalfont
	
	Prove that in any vector space, $V$, if the ordered list, $B=v_1, v_2, v_3, v_4$ \hspace{.1cm} is a basis for $V$, then for 
	\begin{align*}
		C= w_1, w_2, w_3, w_4 ,
	\end{align*}
	where
	\begin{align*}
		w_1=v_1+v_2,\ \  w_2=v_2+v_3,\ \  w_3=v_2,\ \  w_4=v_2 + v_4,
	\end{align*}
	$C$ is also a basis for $V$.  \\
	
	\noindent Prove this by two different methods, as follows:
	
	\begin{enumerate}[(i)]
		\item Use the definitions of linear independence and span to show that $C$ is a basis. (Hint: for span, try to show that $v_i\in\SpanLA(C)$.)
		
		\item Use the results of LADR 2.C to show that $C$ is a basis of $V$.
	\end{enumerate}
	
	
\end{question}








%%%%%%%%%%%%%%%%%%%%%%%%%%%%%%%%%%%%%%%%%%%%%%%%%%%%%%%%%%%%%%%%%%%%%%%%%%%%%%%%%%%%%%%%%%%

%%%%%%%%%%%%%%%%%%%%%%%%%%%%%%%%%%%%%%%%%%%%%%%%%%%%%%%%%%%%%%%%%%%%%%%%%%%%%%%%%%%%%%%%%%%
\end{document}





















